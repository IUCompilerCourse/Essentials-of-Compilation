% Why direct style instead of continuation passing style?

%% Student project ideas:
%%   * high-level optimizations like procedure inlining, etc.
%%   * closure optimization
%%   * adding letrec to the language
%%     (Thought: in the book and regular course, replace top-level defines
%%      with letrec.)
%%   * alternative back ends (ARM, LLVM)
%%   * alternative calling convention (a la Dybvig)
%%   * lazy evaluation
%%   * gradual typing
%%   * continuations (frames in heap a la SML or segmented stack a la Dybvig)
%%   * exceptions
%%   * self hosting
%%   * I/O
%%   * foreign function interface
%%   * quasi-quote and unquote
%%   * macros (too difficult?)
%%   * alternative garbage collector
%%   * alternative register allocator
%%   * parametric polymorphism
%%   * type classes (too difficulty?)
%%   * loops (too easy? combine with something else?)
%%   * loop optimization (fusion, etc.)
%%   * deforestation
%%   * records and subtyping
%%   * object-oriented features
%%     - objects, object types, and structural subtyping (e.g. Abadi & Cardelli)
%%     - class-based objects and nominal subtyping (e.g. Featherweight Java)
%%   * multi-threading, fork join, futures, implicit parallelism
%%   * dataflow analysis, type analysis and specialization


\documentclass[11pt]{book}
\usepackage[T1]{fontenc}
\usepackage[utf8]{inputenc}
\usepackage{lmodern}
\usepackage{hyperref}
\usepackage{graphicx}
\usepackage[english]{babel}
\usepackage{listings}
\usepackage{amsmath}
\usepackage{amsthm}
\usepackage{amssymb}
\usepackage{natbib}
\usepackage{stmaryrd}
\usepackage{xypic}
\usepackage{semantic}
\usepackage{wrapfig}
\usepackage{multirow}
\usepackage{color}

\definecolor{lightgray}{gray}{1}
\newcommand{\black}[1]{{\color{black} #1}}
\newcommand{\gray}[1]{{\color{lightgray} #1}}

%% For pictures
\usepackage{tikz}
\usetikzlibrary{arrows.meta}
\tikzset{baseline=(current bounding box.center), >/.tip={Triangle[scale=1.4]}}

% Computer Modern is already the default. -Jeremy
%\renewcommand{\ttdefault}{cmtt}

\definecolor{comment-red}{rgb}{0.8,0,0}
\if{0}
% Peanut gallery comments:
\newcommand{\rn}[1]{{\color{comment-red}{(RRN: #1)}}}
\newcommand{\margincomment}[1]{\marginpar{#1}}
\else
\newcommand{\rn}[1]{}
\newcommand{\margincomment}[1]{}
\fi

\lstset{%
language=Lisp,
basicstyle=\ttfamily\small,
escapechar=|,
columns=flexible,
moredelim=[is][\color{red}]{~}{~}
}

\newtheorem{theorem}{Theorem}
\newtheorem{lemma}[theorem]{Lemma}
\newtheorem{corollary}[theorem]{Corollary}
\newtheorem{proposition}[theorem]{Proposition}
\newtheorem{constraint}[theorem]{Constraint}
\newtheorem{definition}[theorem]{Definition}
\newtheorem{exercise}[theorem]{Exercise}

%%%%%%%%%%%%%%%%%%%%%%%%%%%%%%%%%%%%%%%%%%%%%%%%%%%%%%%%%%%%%%%%%%%%%%%%%%%%%%%%
% 'dedication' environment: To add a dedication paragraph at the start of book %
% Source: http://www.tug.org/pipermail/texhax/2010-June/015184.html            %
%%%%%%%%%%%%%%%%%%%%%%%%%%%%%%%%%%%%%%%%%%%%%%%%%%%%%%%%%%%%%%%%%%%%%%%%%%%%%%%%
\newenvironment{dedication}
{
   \cleardoublepage
   \thispagestyle{empty}
   \vspace*{\stretch{1}}
   \hfill\begin{minipage}[t]{0.66\textwidth}
   \raggedright
}
{
   \end{minipage}
   \vspace*{\stretch{3}}
   \clearpage
}

%%%%%%%%%%%%%%%%%%%%%%%%%%%%%%%%%%%%%%%%%%%%%%%%
% Chapter quote at the start of chapter        %
% Source: http://tex.stackexchange.com/a/53380 %
%%%%%%%%%%%%%%%%%%%%%%%%%%%%%%%%%%%%%%%%%%%%%%%%
\makeatletter
\renewcommand{\@chapapp}{}% Not necessary...
\newenvironment{chapquote}[2][2em]
  {\setlength{\@tempdima}{#1}%
   \def\chapquote@author{#2}%
   \parshape 1 \@tempdima \dimexpr\textwidth-2\@tempdima\relax%
   \itshape}
  {\par\normalfont\hfill--\ \chapquote@author\hspace*{\@tempdima}\par\bigskip}
\makeatother

%%%%%%%%%%%%%%%%%%%%%%%%%%%%%%%%%%%%%%%%%%%%%%%%%%%%%%%%%%%%%%%%%%%%%%%%%%%%%%%%

\newcommand{\itm}[1]{\ensuremath{\mathit{#1}}}
\newcommand{\Stmt}{\itm{stmt}}
\newcommand{\Exp}{\itm{exp}}
\newcommand{\Def}{\itm{def}}
\newcommand{\Type}{\itm{type}}
\newcommand{\FType}{\itm{ftype}}
\newcommand{\Instr}{\itm{instr}}
\newcommand{\Block}{\itm{block}}
\newcommand{\Prog}{\itm{prog}}
\newcommand{\Arg}{\itm{arg}}
\newcommand{\Reg}{\itm{reg}}
\newcommand{\Int}{\itm{int}}
\newcommand{\Var}{\itm{var}}
\newcommand{\Op}{\itm{op}}
\newcommand{\key}[1]{\texttt{#1}}
\newcommand{\code}[1]{\texttt{#1}}
\newcommand{\READ}{(\key{read})}
\newcommand{\UNIOP}[2]{(\key{#1}~#2)}
\newcommand{\BINOP}[3]{(\key{#1}~#2~#3)}
\newcommand{\LET}[3]{(\key{let}~([#1\;#2])~#3)}

\newcommand{\ASSIGN}[2]{(\key{assign}~#1\;#2)}
\newcommand{\RETURN}[1]{(\key{return}~#1)}

\newcommand{\INT}[1]{(\key{int}\;#1)}
\newcommand{\REG}[1]{(\key{reg}\;#1)}
\newcommand{\VAR}[1]{(\key{var}\;#1)}
\newcommand{\STACKLOC}[1]{(\key{stack}\;#1)}

\newcommand{\IF}[3]{(\key{if}\,#1\;#2\;#3)}

\newcommand{\TTKEY}[1]{{\normalfont\tt #1}}



%%%%%%%%%%%%%%%%%%%%%%%%%%%%%%%%%%%%%%%%%%%%%%%%%%%%%%%%%%%%%%%%%%%%%%%%%%%%%%%%

\title{\Huge \textbf{Essentials of Compilation} \\
  \huge An Incremental Approach}

\author{\textsc{Jeremy G. Siek, Ryan R. Newton} \\
%\thanks{\url{http://homes.soic.indiana.edu/jsiek/}} \\
  Indiana University \\
  \\
  with contributions from: \\
  Carl Factora \\
  Andre Kuhlenschmidt \\
  Michael M. Vitousek \\
  Michael Vollmer \\
  Ryan Scott \\
  Cameron Swords
   }

\begin{document}

\frontmatter
\maketitle

\begin{dedication}
This book is dedicated to the programming language wonks at Indiana
University.
\end{dedication}

\tableofcontents
\listoffigures
%\listoftables

\mainmatter

%%%%%%%%%%%%%%%%%%%%%%%%%%%%%%%%%%%%%%%%%%%%%%%%%%%%%%%%%%%%%%%%%%%%%%%%%%%%%%%%
\chapter*{Preface}

The tradition of compiler writing at Indiana University goes back to
research and courses about programming languages by Daniel Friedman in
the 1970's and 1980's. Dan conducted research on lazy
evaluation~\citep{Friedman:1976aa} in the context of
Lisp~\citep{McCarthy:1960dz} and then studied
continuations~\citep{Felleisen:kx} and
macros~\citep{Kohlbecker:1986dk} in the context of the
Scheme~\citep{Sussman:1975ab}, a dialect of Lisp.  One of the students
of those courses, Kent Dybvig, went on to build Chez
Scheme~\citep{Dybvig:2006aa}, a production-quality and efficient
compiler for Scheme. After completing his Ph.D. at the University of
North Carolina, Kent returned to teach at Indiana University.
Throughout the 1990's and 2000's, Kent continued development of Chez
Scheme and taught the compiler course.

The compiler course evolved to incorporate novel pedagogical ideas
while also including elements of effective real-world compilers.  One
of Dan's ideas was to split the compiler into many small ``passes'' so
that the code for each pass would be easy to understood in isolation.
(In contrast, most compilers of the time were organized into only a
few monolithic passes for reasons of compile-time efficiency.)  Kent,
with later help from his students Dipanwita Sarkar and Andrew Keep,
developed infrastructure to support this approach and evolved the
course, first to use micro-sized passes and then into even smaller
nano passes~\citep{Sarkar:2004fk,Keep:2012aa}. Jeremy Siek was a
student in this compiler course in the early 2000's, as part of his
Ph.D. studies at Indiana University. Needless to say, Jeremy enjoyed
the course immensely!

During that time, another student named Abdulaziz Ghuloum observed
that the front-to-back organization of the course made it difficult
for students to understand the rationale for the compiler
design. Abdulaziz proposed an incremental approach in which the
students build the compiler in stages; they start by implementing a
complete compiler for a very small subset of the input language and in
each subsequent stage they add a language feature and add or modify
passes to handle the new feature~\citep{Ghuloum:2006bh}.  In this way,
the students see how the language features motivate aspects of the
compiler design.

After graduating from Indiana University in 2005, Jeremy went on to
teach at the University of Colorado. He adapted the nano pass and
incremental approaches to compiling a subset of the Python
language~\citep{Siek:2012ab}.  Python and Scheme are quite different
on the surface but there is a large overlap in the compiler techniques
required for the two languages. Thus, Jeremy was able to teach much of
the same content from the Indiana compiler course. He very much
enjoyed teaching the course organized in this way, and even better,
many of the students learned a lot and got excited about compilers.

Jeremy returned to teach at Indiana University in 2013.  In his
absence the compiler course had switched from the front-to-back
organization to a back-to-front organization. Seeing how well the
incremental approach worked at Colorado, he started porting and
adapting the structure of the Colorado course back into the land of
Scheme. In the meantime Indiana had moved on from Scheme to Racket, so
the course is now about compiling a subset of Racket (and Typed
Racket) to the x86 assembly language. The compiler is implemented in
Racket 7.1~\citep{plt-tr}.

This is the textbook for the incremental version of the compiler
course at Indiana University (Spring 2016 - present) and it is the
first open textbook for an Indiana compiler course.  With this book we
hope to make the Indiana compiler course available to people that have
not had the chance to study in Bloomington in person.  Many of the
compiler design decisions in this book are drawn from the assignment
descriptions of \cite{Dybvig:2010aa}. We have captured what we think
are the most important topics from \cite{Dybvig:2010aa} but we have
omitted topics that we think are less interesting conceptually and we
have made simplifications to reduce complexity.  In this way, this
book leans more towards pedagogy than towards the efficiency of the
generated code. Also, the book differs in places where we saw the
opportunity to make the topics more fun, such as in relating register
allocation to Sudoku (Chapter~\ref{ch:register-allocation-r1}).

\section*{Prerequisites}

The material in this book is challenging but rewarding. It is meant to
prepare students for a lifelong career in programming languages.  We do
not recommend this book for students who want to dabble in programming
languages.

The book uses the Racket language both for the implementation of the
compiler and for the language that is compiled, so a student should be
proficient with Racket (or Scheme) prior to reading this book. There
are many excellent resources for learning Scheme and
Racket~\citep{Dybvig:1987aa,Abelson:1996uq,Friedman:1996aa,Felleisen:2001aa,Felleisen:2013aa,Flatt:2014aa}. It
is helpful but not necessary for the student to have prior exposure to
the x86 (or x86-64) assembly language~\citep{Intel:2015aa}, as one might
obtain from a computer systems
course~\citep{Bryant:2005aa,Bryant:2010aa}.  This book introduces the
parts of x86-64 assembly language that are needed.

%\section*{Structure of book}
% You might want to add short description about each chapter in this book.

%\section*{About the companion website}
%The website\footnote{\url{https://github.com/amberj/latex-book-template}} for %this file contains:
%\begin{itemize}
%  \item A link to (freely downlodable) latest version of this document.
%  \item Link to download LaTeX source for this document.
%  \item Miscellaneous material (e.g. suggested readings etc).
%\end{itemize}

\section*{Acknowledgments}

Many people have contributed to the ideas, techniques, organization,
and teaching of the materials in this book. We especially thank the
following people.

\begin{itemize}
\item Bor-Yuh Evan Chang
\item Kent Dybvig
\item Daniel P. Friedman
\item Ronald Garcia
\item Abdulaziz Ghuloum
\item Jay McCarthy
\item Dipanwita Sarkar
\item Andrew Keep
\item Oscar Waddell
\item Michael Wollowski
\end{itemize}

\mbox{}\\
\noindent Jeremy G. Siek \\
\noindent \url{http://homes.soic.indiana.edu/jsiek} \\
%\noindent Spring 2016


%%%%%%%%%%%%%%%%%%%%%%%%%%%%%%%%%%%%%%%%%%%%%%%%%%%%%%%%%%%%%%%%%%%%%%%%%%%%%%%%
\chapter{Preliminaries}
\label{ch:trees-recur}

In this chapter we review the basic tools that are needed to implement
a compiler. We use \emph{abstract syntax trees} (ASTs), which are data
structures in computer memory, rather than programs as they are
typically stored in text files on disk, as \emph{concrete syntax}.
%
ASTs can be represented in many different ways, depending on the programming
language used to write the compiler.
%
Because this book uses Racket (\url{http://racket-lang.org}), a
descendant of Lisp, we use S-expressions to conveniently represent
ASTs (Section~\ref{sec:ast}). We use grammars to defined the abstract
syntax of programming languages (Section~\ref{sec:grammar}) and
pattern matching to inspect individual nodes in an AST
(Section~\ref{sec:pattern-matching}).  We use recursion to construct
and deconstruct entire ASTs (Section~\ref{sec:recursion}).  This
chapter provides an brief introduction to these ideas.

\section{Abstract Syntax Trees and S-expressions}
\label{sec:ast}

The primary data structure that is commonly used for representing
programs is the \emph{abstract syntax tree} (AST). When considering
some part of a program, a compiler needs to ask what kind of thing it
is and what sub-parts it contains. For example, the program on the
left, represented by an S-expression, corresponds to the AST on the
right.
\begin{center}
\begin{minipage}{0.4\textwidth}
\begin{lstlisting}
(+ (read) (- 8))
\end{lstlisting}
\end{minipage}
\begin{minipage}{0.4\textwidth}
\begin{equation}
\begin{tikzpicture}
 \node[draw, circle] (plus)  at (0 ,  0) {\key{+}};
 \node[draw, circle] (read)  at (-1, -1.5) {{\footnotesize\key{read}}};
 \node[draw, circle] (minus) at (1 , -1.5) {$\key{-}$};
 \node[draw, circle] (8)     at (1 , -3) {\key{8}};

 \draw[->] (plus) to (read);
 \draw[->] (plus) to (minus);
 \draw[->] (minus) to (8);
\end{tikzpicture}
\label{eq:arith-prog}
\end{equation}
\end{minipage}
\end{center}
We shall use the standard terminology for trees: each circle above is
called a \emph{node}. The arrows connect a node to its \emph{children}
(which are also nodes). The top-most node is the \emph{root}.  Every
node except for the root has a \emph{parent} (the node it is the child
of). If a node has no children, it is a \emph{leaf} node.  Otherwise
it is an \emph{internal} node.

Recall that an \emph{symbolic expression} (S-expression) is either
\begin{enumerate}
\item an atom, or
\item a pair of two S-expressions, written $(e_1 \key{.} e_2)$,
    where $e_1$ and $e_2$ are each an S-expression.
\end{enumerate}
An \emph{atom} can be a symbol, such as \code{`hello}, a number, the
null value \code{'()}, etc.  We can create an S-expression in Racket
simply by writing a backquote (called a quasi-quote in Racket)
followed by the textual representation of the S-expression.  It is
quite common to use S-expressions to represent a list, such as $a, b
,c$ in the following way:
\begin{lstlisting}
    `(a . (b . (c . ())))
\end{lstlisting}
Each element of the list is in the first slot of a pair, and the
second slot is either the rest of the list or the null value, to mark
the end of the list. Such lists are so common that Racket provides
special notation for them that removes the need for the periods
and so many parenthesis:
\begin{lstlisting}
    `(a b c)
\end{lstlisting}
The following expression creates an S-expression that represents AST
\eqref{eq:arith-prog}.
\begin{center}
\texttt{`(+ (read) (- 8))}
\end{center}
When using S-expressions to represent ASTs, the convention is to
represent each AST node as a list and to put the operation symbol at
the front of the list. The rest of the list contains the children.  So
in the above case, the root AST node has operation \code{`+} and its
two children are \code{`(read)} and \code{`(- 8)}, just as in the
diagram \eqref{eq:arith-prog}.

To build larger S-expressions one often needs to splice together
several smaller S-expressions. Racket provides the comma operator to
splice an S-expression into a larger one. For example, instead of
creating the S-expression for AST \eqref{eq:arith-prog} all at once,
we could have first created an S-expression for AST
\eqref{eq:arith-neg8} and then spliced that into the addition
S-expression.
\begin{lstlisting}
   (define ast1.4 `(- 8))
   (define ast1.1 `(+ (read) ,ast1.4))
\end{lstlisting}
In general, the Racket expression that follows the comma (splice)
can be any expression that produces an S-expression.

When deciding how to compile program \eqref{eq:arith-prog}, we need to
know that the operation associated with the root node is addition and
that it has two children: \texttt{read} and a negation. The AST data
structure directly supports these queries, as we shall see in
Section~\ref{sec:pattern-matching}, and hence is a good choice for use
in compilers. In this book, we often write down the S-expression
representation of a program even when we really have in mind the AST
because the S-expression is more concise.  We recommend that, in your
mind, you always think of programs as abstract syntax trees.

\section{Grammars}
\label{sec:grammar}

A programming language can be thought of as a \emph{set} of programs.
The set is typically infinite (one can always create larger and larger
programs), so one cannot simply describe a language by listing all of
the programs in the language. Instead we write down a set of rules, a
\emph{grammar}, for building programs. We shall write our rules in a
variant of Backus-Naur Form (BNF)~\citep{Backus:1960aa,Knuth:1964aa}.
As an example, we describe a small language, named $R_0$, of
integers and arithmetic operations. The first rule says that any
integer is an expression, $\Exp$, in the language:
\begin{equation}
\Exp ::= \Int  \label{eq:arith-int}
\end{equation}
%
Each rule has a left-hand-side and a right-hand-side. The way to read
a rule is that if you have all the program parts on the
right-hand-side, then you can create an AST node and categorize it
according to the left-hand-side.
%
A name such as $\Exp$ that is
defined by the grammar rules is a \emph{non-terminal}.
%
The name $\Int$ is a also a non-terminal, however,
we do not define $\Int$ because the
reader already knows what an integer is.
%
Further, we make the simplifying design decision that all of the languages in
this book only handle machine-representable integers.  On most modern machines
this corresponds to integers represented with 64-bits, i.e., the in range
$-2^{63}$ to $2^{63}-1$.
%
However, we restrict this range further to match the Racket \texttt{fixnum}
datatype, which allows 63-bit integers on a 64-bit machine.

The second grammar rule is the \texttt{read} operation that receives
an input integer from the user of the program.
\begin{equation}
  \Exp ::= (\key{read}) \label{eq:arith-read}
\end{equation}

The third rule says that, given an $\Exp$ node, you can build another
$\Exp$ node by negating it.
\begin{equation}
  \Exp ::= (\key{-} \; \Exp)  \label{eq:arith-neg}
\end{equation}
Symbols such as \key{-} in typewriter font are \emph{terminal} symbols
and must literally appear in the program for the rule to be
applicable.

We can apply the rules to build ASTs in the $R_0$
language. For example, by rule \eqref{eq:arith-int}, \texttt{8} is an
$\Exp$, then by rule \eqref{eq:arith-neg}, the following AST is
an $\Exp$.
\begin{center}
\begin{minipage}{0.25\textwidth}
\begin{lstlisting}
(- 8)
\end{lstlisting}
\end{minipage}
\begin{minipage}{0.25\textwidth}
\begin{equation}
\begin{tikzpicture}
 \node[draw, circle] (minus) at (0, 0)  {$\text{--}$};
 \node[draw, circle] (8)     at (0, -1.2) {$8$};

 \draw[->] (minus) to (8);
\end{tikzpicture}
\label{eq:arith-neg8}
\end{equation}
\end{minipage}
\end{center}

The following grammar rule defines addition expressions:
\begin{equation}
  \Exp ::= (\key{+} \; \Exp \; \Exp) \label{eq:arith-add}
\end{equation}
Now we can see that the AST \eqref{eq:arith-prog} is an $\Exp$ in
$R_0$.  We know that \lstinline{(read)} is an $\Exp$ by rule
\eqref{eq:arith-read} and we have shown that \texttt{(- 8)} is an
$\Exp$, so we can apply rule \eqref{eq:arith-add} to show that
\texttt{(+ (read) (- 8))} is an $\Exp$ in the $R_0$ language.

If you have an AST for which the above rules do not apply, then the
AST is not in $R_0$. For example, the AST \texttt{(- (read) (+ 8))} is
not in $R_0$ because there are no rules for \key{+} with only one
argument, nor for \key{-} with two arguments.  Whenever we define a
language with a grammar, we implicitly mean for the language to be the
smallest set of programs that are justified by the rules. That is, the
language only includes those programs that the rules allow.

The last grammar rule for $R_0$ states that there is a \key{program}
node to mark the top of the whole program:
\[
  R_0 ::= (\key{program} \; \Exp)
\]

The \code{read-program} function provided in \code{utilities.rkt}
reads programs in from a file (the sequence of characters in the
concrete syntax of Racket) and parses them into the abstract syntax
tree. The concrete syntax does not include a \key{program} form; that
is added by the \code{read-program} function as it creates the
AST. See the description of \code{read-program} in
Appendix~\ref{appendix:utilities} for more details.

It is common to have many rules with the same left-hand side, such as
$\Exp$ in the grammar for $R_0$, so there is a vertical bar notation
for gathering several rules, as shown in
Figure~\ref{fig:r0-syntax}. Each clause between a vertical bar is
called an {\em alternative}.

\begin{figure}[tp]
\fbox{
\begin{minipage}{0.96\textwidth}
\[
\begin{array}{rcl}
\Exp &::=& \Int \mid ({\tt \key{read}}) \mid (\key{-} \; \Exp) \mid
   (\key{+} \; \Exp \; \Exp)  \\
R_0  &::=& (\key{program} \; \Exp)
\end{array}
\]
\end{minipage}
}
\caption{The syntax of $R_0$, a language of integer arithmetic.}
\label{fig:r0-syntax}
\end{figure}




\section{Pattern Matching}
\label{sec:pattern-matching}

As mentioned above, one of the operations that a compiler needs to
perform on an AST is to access the children of a node.  Racket
provides the \texttt{match} form to access the parts of an
S-expression. Consider the following example and the output on the
right.
\begin{center}
\begin{minipage}{0.5\textwidth}
\begin{lstlisting}
(match ast1.1
  [`(,op ,child1 ,child2)
    (print op) (newline)
    (print child1) (newline)
    (print child2)])
\end{lstlisting}
\end{minipage}
\vrule
\begin{minipage}{0.25\textwidth}
\begin{lstlisting}


   '+
   '(read)
   '(- 8)
\end{lstlisting}
\end{minipage}
\end{center}
The \texttt{match} form takes AST \eqref{eq:arith-prog} and binds its
parts to the three variables \texttt{op}, \texttt{child1}, and
\texttt{child2}. In general, a match clause consists of a
\emph{pattern} and a \emph{body}. The pattern is a quoted S-expression
that may contain pattern-variables (each one preceded by a comma).
%
The pattern is not the same thing as a quasiquote expression used to
\emph{construct} ASTs, however, the similarity is intentional: constructing and
deconstructing ASTs uses similar syntax.
%
While the pattern uses a restricted syntax,
the body of the match clause may contain any Racket code whatsoever.


A \texttt{match} form may contain several clauses, as in the following
function \texttt{leaf?} that recognizes when an $R_0$ node is
a leaf. The \texttt{match} proceeds through the clauses in order,
checking whether the pattern can match the input S-expression. The
body of the first clause that matches is executed. The output of
\texttt{leaf?} for several S-expressions is shown on the right. In the
below \texttt{match}, we see another form of pattern: the \texttt{(?
  fixnum?)} applies the predicate \texttt{fixnum?} to the input
S-expression to see if it is a machine-representable integer.
\begin{center}
\begin{minipage}{0.5\textwidth}
\begin{lstlisting}
(define (leaf? arith)
  (match arith
    [(? fixnum?) #t]
    [`(read) #t]
    [`(- ,c1) #f]
    [`(+ ,c1 ,c2) #f]))

(leaf? `(read))
(leaf? `(- 8))
(leaf? `(+ (read) (- 8)))
\end{lstlisting}
\end{minipage}
\vrule
\begin{minipage}{0.25\textwidth}
  \begin{lstlisting}







   #t
   #f
   #f
\end{lstlisting}
\end{minipage}
\end{center}


\section{Recursion}
\label{sec:recursion}

Programs are inherently recursive in that an $R_0$ expression ($\Exp$)
is made up of smaller expressions. Thus, the natural way to process an
entire program is with a recursive function.  As a first example of
such a function, we define \texttt{exp?} below, which takes an
arbitrary S-expression, {\tt sexp}, and determines whether or not {\tt
  sexp} is an $R_0$ expression. Note that each match clause
corresponds to one grammar rule the body of each clause makes a
recursive call for each child node. This pattern of recursive function
is so common that it has a name, \emph{structural recursion}.  In
general, when a recursive function is defined using a sequence of
match clauses that correspond to a grammar, and each clause body makes
a recursive call on each child node, then we say the function is
defined by structural recursion. Below we also define a second
function, named \code{R0?}, determines whether an S-expression is an
$R_0$ program.
%
\begin{center}
\begin{minipage}{0.7\textwidth}
\begin{lstlisting}
(define (exp? sexp)
  (match sexp
    [(? fixnum?) #t]
    [`(read) #t]
    [`(- ,e) (exp? e)]
    [`(+ ,e1 ,e2)
      (and (exp? e1) (exp? e2))]
    [else #f]))

(define (R0? sexp)
  (match sexp
    [`(program ,e) (exp? e)]
    [else #f]))

(R0? `(program (+ (read) (- 8))))
(R0? `(program (- (read) (+ 8))))
\end{lstlisting}
\end{minipage}
\vrule
\begin{minipage}{0.25\textwidth}
\begin{lstlisting}














   #t
   #f
\end{lstlisting}
\end{minipage}
\end{center}

Indeed, the structural recursion follows the grammar itself.  We can
generally expect to write a recursive function to handle each
non-terminal in the grammar.\footnote{This principle of structuring
  code according to the data definition is advocated in the book
  \emph{How to Design Programs}
  \url{http://www.ccs.neu.edu/home/matthias/HtDP2e/}.}

You may be tempted to write the program with just one function, like this:
\begin{center}
\begin{minipage}{0.5\textwidth}
\begin{lstlisting}
(define (R0? sexp)
  (match sexp
    [(? fixnum?) #t]
    [`(read) #t]
    [`(- ,e) (R0? e)]
    [`(+ ,e1 ,e2) (and (R0? e1) (R0? e2))]
    [`(program ,e) (R0? e)]
    [else #f]))
\end{lstlisting}
\end{minipage}
\end{center}
%
Sometimes such a trick will save a few lines of code, especially when it comes
to the {\tt program} wrapper.  Yet this style is generally \emph{not}
recommended because it can get you into trouble.
%
For instance, the above function is subtly wrong:
\lstinline{(R0? `(program (program 3)))} will return true, when it
should return false.

%% NOTE FIXME - must check for consistency on this issue throughout.


\section{Interpreters}
\label{sec:interp-R0}

The meaning, or semantics, of a program is typically defined in the
specification of the language. For example, the Scheme language is
defined in the report by \cite{SPERBER:2009aa}. The Racket language is
defined in its reference manual~\citep{plt-tr}. In this book we use an
interpreter to define the meaning of each language that we consider,
following Reynolds' advice in this
regard~\citep{reynolds72:_def_interp}. Here we warm up by writing an
interpreter for the $R_0$ language, which serves as a second example
of structural recursion. The \texttt{interp-R0} function is defined in
Figure~\ref{fig:interp-R0}. The body of the function is a match on the
input program \texttt{p} and then a call to the \lstinline{interp-exp}
helper function, which in turn has one match clause per grammar rule
for $R_0$ expressions.

\begin{figure}[tbp]
\begin{lstlisting}
   (define (interp-exp e)
     (match e
       [(? fixnum?) e]
       [`(read)
        (let ([r (read)])
          (cond [(fixnum? r) r]
                [else (error 'interp-R0 "input not an integer" r)]))]
       [`(- ,e1)     (fx- 0 (interp-exp e1))]
       [`(+ ,e1 ,e2) (fx+ (interp-exp e1) (interp-exp e2))]
       ))

   (define (interp-R0 p)
     (match p
       [`(program ,e) (interp-exp e)]))
\end{lstlisting}
\caption{Interpreter for the $R_0$ language.}
\label{fig:interp-R0}
\end{figure}

Let us consider the result of interpreting a few $R_0$ programs. The
following program simply adds two integers.
\begin{lstlisting}
   (+ 10 32)
\end{lstlisting}
The result is \key{42}, as you might have expected.  Here we have written the
program in concrete syntax, whereas the parsed abstract syntax would be the
slightly different: \lstinline{(program (+ 10 32))}.

The next example demonstrates that expressions may be nested within
each other, in this case nesting several additions and negations.
\begin{lstlisting}
   (+ 10 (- (+ 12 20)))
\end{lstlisting}
What is the result of the above program?

As mentioned previously, the $R0$ language does not support
arbitrarily-large integers, but only $63$-bit integers, so we
interpret the arithmetic operations of $R0$ using fixnum arithmetic.
What happens when we run the following program?
\begin{lstlisting}
   (define large 999999999999999999)
   (interp-R0 `(program (+ (+ (+ ,large ,large) (+ ,large ,large))
                           (+ (+ ,large ,large) (+ ,large ,large)))))
\end{lstlisting}
It produces an error:
\begin{lstlisting}
   fx+: result is not a fixnum
\end{lstlisting}
We shall use the convention that if the interpreter for a language
produces an error when run on a program, then the meaning of the
program is unspecified. The compiler for the language is under no
obligation for such a program; it can produce an executable that does
anything.

\noindent
Moving on, the \key{read} operation prompts the user of the program
for an integer. If we interpret the AST \eqref{eq:arith-prog} and give
it the input \texttt{50}
\begin{lstlisting}
   (interp-R0 ast1.1)
\end{lstlisting}
we get the answer to life, the universe, and everything:
\begin{lstlisting}
   42
\end{lstlisting}
We include the \key{read} operation in $R_0$ so a clever student
cannot implement a compiler for $R_0$ simply by running the
interpreter at compilation time to obtain the output and then
generating the trivial code to return the output.  (A clever student
did this in a previous version of the course.)

The job of a compiler is to translate a program in one language into a
program in another language so that the output program behaves the
same way as the input program. This idea is depicted in the following
diagram. Suppose we have two languages, $\mathcal{L}_1$ and
$\mathcal{L}_2$, and an interpreter for each language.  Suppose that
the compiler translates program $P_1$ in language $\mathcal{L}_1$ into
program $P_2$ in language $\mathcal{L}_2$.  Then interpreting $P_1$
and $P_2$ on their respective interpreters with input $i$ should yield
the same output $o$.
\begin{equation} \label{eq:compile-correct}
\begin{tikzpicture}[baseline=(current  bounding  box.center)]
 \node (p1) at (0,  0) {$P_1$};
 \node (p2) at (3,  0) {$P_2$};
 \node (o)  at (3, -2.5) {$o$};

 \path[->] (p1) edge [above] node {compile} (p2);
 \path[->] (p2) edge [right] node {interp-$\mathcal{L}_2$($i$)} (o);
 \path[->] (p1) edge [left]  node {interp-$\mathcal{L}_1$($i$)} (o);
\end{tikzpicture}
\end{equation}
In the next section we see our first example of a compiler, which is
another example of structural recursion.


\section{Example Compiler: a Partial Evaluator}
\label{sec:partial-evaluation}

In this section we consider a compiler that translates $R_0$
programs into $R_0$ programs that are more efficient, that is,
this compiler is an optimizer. Our optimizer will accomplish this by
trying to eagerly compute the parts of the program that do not depend
on any inputs. For example, given the following program
\begin{lstlisting}
   (+ (read) (- (+ 5 3)))
\end{lstlisting}
our compiler will translate it into the program
\begin{lstlisting}
   (+ (read) -8)
\end{lstlisting}

Figure~\ref{fig:pe-arith} gives the code for a simple partial
evaluator for the $R_0$ language. The output of the partial evaluator
is an $R_0$ program, which we build up using a combination of
quasiquotes and commas. (Though no quasiquote is necessary for
integers.) In Figure~\ref{fig:pe-arith}, the normal structural
recursion is captured in the main \texttt{pe-arith} function whereas
the code for partially evaluating negation and addition is factored
into two separate helper functions: \texttt{pe-neg} and
\texttt{pe-add}. The input to these helper functions is the output of
partially evaluating the children nodes.

\begin{figure}[tbp]
\begin{lstlisting}
   (define (pe-neg r)
     (cond [(fixnum? r) (fx- 0 r)]
           [else `(- ,r)]))

   (define (pe-add r1 r2)
     (cond [(and (fixnum? r1) (fixnum? r2)) (fx+ r1 r2)]
           [else `(+ ,r1 ,r2)]))

   (define (pe-arith e)
     (match e
       [(? fixnum?) e]
       [`(read) `(read)]
       [`(- ,e1)
         (pe-neg (pe-arith e1))]
       [`(+ ,e1 ,e2)
         (pe-add (pe-arith e1) (pe-arith e2))]))
\end{lstlisting}
\caption{A partial evaluator for $R_0$ expressions.}
\label{fig:pe-arith}
\end{figure}

Our code for \texttt{pe-neg} and \texttt{pe-add} implements the simple
idea of checking whether their arguments are integers and if they are,
to go ahead and perform the arithmetic.  Otherwise, we use quasiquote
to create an AST node for the appropriate operation (either negation
or addition) and use comma to splice in the child nodes.

To gain some confidence that the partial evaluator is correct, we can
test whether it produces programs that get the same result as the
input program. That is, we can test whether it satisfies Diagram
\eqref{eq:compile-correct}. The following code runs the partial
evaluator on several examples and tests the output program.  The
\texttt{assert} function is defined in Appendix~\ref{appendix:utilities}.
\begin{lstlisting}
(define (test-pe p)
  (assert "testing pe-arith"
     (equal? (interp-R0 p) (interp-R0 (pe-arith p)))))

(test-pe `(+ (read) (- (+ 5 3))))
(test-pe `(+ 1 (+ (read) 1)))
(test-pe `(- (+ (read) (- 5))))
\end{lstlisting}

\rn{Do we like the explicit whitespace?  I've never been fond of it, in part
  because it breaks copy/pasting.  But, then again, so do most of the quotes.}

\begin{exercise}
\normalfont % I don't like the italics for exercises. -Jeremy
We challenge the reader to improve on the simple partial evaluator in
Figure~\ref{fig:pe-arith} by replacing the \texttt{pe-neg} and
\texttt{pe-add} helper functions with functions that know more about
arithmetic. For example, your partial evaluator should translate
\begin{lstlisting}
   (+ 1 (+ (read) 1))
\end{lstlisting}
into
\begin{lstlisting}
   (+ 2 (read))
\end{lstlisting}
To accomplish this, we recommend that your partial evaluator produce
output that takes the form of the $\itm{residual}$ non-terminal in the
following grammar.
\[
\begin{array}{lcl}
\Exp &::=& \Int \mid (\key{read}) \mid (\key{-} \;(\key{read})) \mid (\key{+} \; \Exp \; \Exp)\\
\itm{residual} &::=& \Int \mid (\key{+}\; \Int\; \Exp) \mid \Exp
\end{array}
\]
\end{exercise}


%%%%%%%%%%%%%%%%%%%%%%%%%%%%%%%%%%%%%%%%%%%%%%%%%%%%%%%%%%%%%%%%%%%%%%%%%%%%%%%%
\chapter{Integers and Variables}
\label{ch:int-exp}

This chapter concerns the challenge of compiling a subset of Racket
that includes integer arithmetic and local variable binding, which we
name $R_1$, to x86-64 assembly code~\citep{Intel:2015aa}.  Henceforth
we shall refer to x86-64 simply as x86.  The chapter begins with a
description of the $R_1$ language (Section~\ref{sec:s0}) followed by a
description of x86 (Section~\ref{sec:x86}). The x86 assembly language
is quite large, so we only discuss what is needed for compiling
$R_1$. We introduce more of x86 in later chapters. Once we have
introduced $R_1$ and x86, we reflect on their differences and come up
with a plan to break down the translation from $R_1$ to x86 into a
handful of steps (Section~\ref{sec:plan-s0-x86}).  The rest of the
sections in this Chapter give detailed hints regarding each step
(Sections~\ref{sec:uniquify-s0} through \ref{sec:patch-s0}).  We hope
to give enough hints that the well-prepared reader can implement a
compiler from $R_1$ to x86 while at the same time leaving room for
some fun and creativity.

\section{The $R_1$ Language}
\label{sec:s0}

The $R_1$ language extends the $R_0$ language
(Figure~\ref{fig:r0-syntax}) with variable definitions.  The syntax of
the $R_1$ language is defined by the grammar in
Figure~\ref{fig:r1-syntax}.  The non-terminal \Var{} may be any Racket
identifier. As in $R_0$, \key{read} is a nullary operator, \key{-} is
a unary operator, and \key{+} is a binary operator.  Similar to $R_0$,
the $R_1$ language includes the \key{program} construct to mark the
top of the program, which is helpful in parts of the compiler.  The
$\itm{info}$ field of the \key{program} construct contains an
association list that is used to communicate auxiliary data from one
step of the compiler to the next.

The $R_1$ language is rich enough to exhibit several compilation
techniques but simple enough so that the reader, together with couple
friends, can implement a compiler for it in a week or two of part-time
work.  To give the reader a feeling for the scale of this first
compiler, the instructor solution for the $R_1$ compiler is less than
500 lines of code.

\begin{figure}[btp]
\centering
\fbox{
\begin{minipage}{0.96\textwidth}
\[
\begin{array}{rcl}
\Exp &::=& \Int \mid (\key{read}) \mid (\key{-}\;\Exp) \mid (\key{+} \; \Exp\;\Exp)  \\
     &\mid&  \Var \mid \LET{\Var}{\Exp}{\Exp} \\
R_1  &::=& (\key{program} \;\itm{info}\; \Exp)
\end{array}
\]
\end{minipage}
}
\caption{The syntax of $R_1$, a language of integers and variables.}
\label{fig:r1-syntax}
\end{figure}

Let us dive into the description of the $R_1$ language.  The \key{let}
construct defines a variable for use within its body and initializes
the variable with the value of an expression.  So the following
program initializes \code{x} to \code{32} and then evaluates the body
\code{(+ 10 x)}, producing \code{42}.
\begin{lstlisting}
   (program ()
      (let ([x (+ 12 20)]) (+ 10 x)))
\end{lstlisting}
When there are multiple \key{let}'s for the same variable, the closest
enclosing \key{let} is used. That is, variable definitions overshadow
prior definitions. Consider the following program with two \key{let}'s
that define variables named \code{x}. Can you figure out the result?
\begin{lstlisting}
   (program ()
      (let ([x 32]) (+ (let ([x 10]) x) x)))
\end{lstlisting}
For the purposes of showing which variable uses correspond to which
definitions, the following shows the \code{x}'s annotated with subscripts
to distinguish them. Double check that your answer for the above is
the same as your answer for this annotated version of the program.
\begin{lstlisting}
   (program ()
      (let ([x|$_1$| 32]) (+ (let ([x|$_2$| 10]) x|$_2$|) x|$_1$|)))
\end{lstlisting}
The initializing expression is always evaluated before the body of the
\key{let}, so in the following, the \key{read} for \code{x} is
performed before the \key{read} for \code{y}. Given the input
\code{52} then \code{10}, the following produces \code{42} (and not
\code{-42}).
\begin{lstlisting}
   (program ()
     (let ([x (read)]) (let ([y (read)]) (+ x (- y)))))
\end{lstlisting}

Figure~\ref{fig:interp-R1} shows the interpreter for the $R_1$
language. It extends the interpreter for $R_0$ with two new
\key{match} clauses for variables and for \key{let}.  For \key{let},
we will need a way to communicate the initializing value of a variable
to all the uses of a variable. To accomplish this, we maintain a
mapping from variables to values, which is traditionally called an
\emph{environment}. For simplicity, here we use an association list to
represent the environment. The \code{interp-R1} function takes the
current environment, \code{env}, as an extra parameter.  When the
interpreter encounters a variable, it finds the corresponding value
using the \code{lookup} function (Appendix~\ref{appendix:utilities}).
When the interpreter encounters a \key{let}, it evaluates the
initializing expression, extends the environment with the result bound
to the variable, then evaluates the body of the \key{let}.

\begin{figure}[tbp]
\begin{lstlisting}
(define (interp-exp env)
  (lambda (e)
    (match e
      [(? fixnum?) e]
      [`(read)
       (define r (read))
       (cond [(fixnum? r) r]
             [else (error 'interp-R1 "expected an integer" r)])]
      [`(- ,e)
       (define v ((interp-exp env) e))
       (fx- 0 v)]
      [`(+ ,e1 ,e2)
       (define v1 ((interp-exp env) e1))
       (define v2 ((interp-exp env) e2))
       (fx+ v1 v2)]
      [(? symbol?) (lookup e env)]
      [`(let ([,x ,e]) ,body)
       (define new-env (cons (cons x ((interp-exp env) e)) env))
       ((interp-exp new-env) body)]
      )))

   (define (interp-R1 env)
     (lambda (p)
       (match p
         [`(program ,info ,e) ((interp-exp '()) e)])))
\end{lstlisting}
\caption{Interpreter for the $R_1$ language.}
\label{fig:interp-R1}
\end{figure}

The goal for this chapter is to implement a compiler that translates
any program $P_1$ in the $R_1$ language into an x86 assembly
program $P_2$ such that $P_2$ exhibits the same behavior on an x86
computer as the $R_1$ program running in a Racket implementation.
That is, they both output the same integer $n$.
\[
\begin{tikzpicture}[baseline=(current  bounding  box.center)]
 \node (p1) at (0,  0)   {$P_1$};
 \node (p2) at (4,  0)   {$P_2$};
 \node (o)  at (4, -2) {$n$};

 \path[->] (p1) edge [above] node {\footnotesize compile} (p2);
 \path[->] (p1) edge [left]  node {\footnotesize interp-$R_1$} (o);
 \path[->] (p2) edge [right] node {\footnotesize interp-x86} (o);
\end{tikzpicture}
\]
In the next section we introduce enough of the x86 assembly
language to compile $R_1$.

\section{The x86 Assembly Language}
\label{sec:x86}

An x86 program is a sequence of instructions. The program is stored in the
computer's memory and the \emph{program counter} points to the address of the
next instruction to be executed. For most instructions, once the instruction is
executed, the program counter is incremented to point to the immediately
following instruction in memory.  Each instruction may refer to integer
constants (called \emph{immediate values}), variables called \emph{registers},
and instructions may load and store values into memory.  For our purposes, we
can think of the computer's memory as a mapping of 64-bit addresses to 64-bit
values%
\footnote{This simple story suffices for describing how sequential
  programs access memory but is not sufficient for multi-threaded
  programs. However, multi-threaded execution is beyond the scope of
  this book.}.
%
Figure~\ref{fig:x86-a} defines the syntax for the
subset of the x86 assembly language needed for this chapter.
%
We use the AT\&T syntax expected by the GNU assembler, which comes
with the \key{gcc} compiler that we use for compiling assembly code to
machine code.
%
Also, Appendix~\ref{sec:x86-quick-reference} includes a quick-reference of all
the x86 instructions used in this book and a short explanation of what they do.


% to do: finish treatment of imulq
% it's needed for vector's in R6/R7

\newcommand{\allregisters}{\key{rsp} \mid \key{rbp} \mid \key{rax} \mid \key{rbx} \mid \key{rcx}
              \mid \key{rdx} \mid \key{rsi} \mid \key{rdi} \mid \\
              && \key{r8} \mid \key{r9} \mid \key{r10}
              \mid \key{r11} \mid \key{r12} \mid \key{r13}
              \mid \key{r14} \mid \key{r15}}

\begin{figure}[tp]
\fbox{
\begin{minipage}{0.96\textwidth}
\[
\begin{array}{lcl}
\Reg &::=& \allregisters{} \\
\Arg &::=&  \key{\$}\Int \mid \key{\%}\Reg \mid \Int(\key{\%}\Reg) \\
\Instr &::=& \key{addq} \; \Arg, \Arg \mid
      \key{subq} \; \Arg, \Arg \mid
      \key{negq} \; \Arg \mid \key{movq} \; \Arg, \Arg \mid \\
  &&  \key{callq} \; \mathit{label} \mid
      \key{pushq}\;\Arg \mid \key{popq}\;\Arg \mid \key{retq} \mid \itm{label}\key{:}\; \Instr \\
\Prog &::= & \key{.globl main}\\
      &    & \key{main:} \; \Instr^{+}
\end{array}
\]
\end{minipage}
}
\caption{A subset of the x86 assembly language (AT\&T syntax).}
\label{fig:x86-a}
\end{figure}

An immediate value is written using the notation \key{\$}$n$ where $n$
is an integer.
%
A register is written with a \key{\%} followed by the register name,
such as \key{\%rax}.
%
An access to memory is specified using the syntax $n(\key{\%}r)$,
which obtains the address stored in register $r$ and then
offsets the address by $n$ bytes
(8 bits). The address is then used to either load or store to memory
depending on whether it occurs as a source or destination argument of
an instruction.

An arithmetic instruction, such as $\key{addq}\,s,\,d$, reads from the
source $s$ and destination $d$, applies the arithmetic operation, then
writes the result in $d$.
%
The move instruction, $\key{movq}\,s\,d$ reads from $s$ and stores the
result in $d$.
%
The $\key{callq}\,\mathit{label}$ instruction executes the procedure
specified by the label.

Figure~\ref{fig:p0-x86} depicts an x86 program that is equivalent
to \code{(+ 10 32)}. The \key{globl} directive says that the
\key{main} procedure is externally visible, which is necessary so
that the operating system can call it. The label \key{main:}
indicates the beginning of the \key{main} procedure which is where
the operating system starts executing this program.  The instruction
\lstinline{movq $10, %rax} puts $10$ into register \key{rax}. The
following instruction \lstinline{addq $32, %rax} adds $32$ to the
$10$ in \key{rax} and puts the result, $42$, back into
  \key{rax}.

The last instruction, \key{retq}, finishes the \key{main} function by
returning the integer in \key{rax} to the operating system. The
operating system interprets this integer as the program's exit
code. By convention, an exit code of 0 indicates the program was
successful, and all other exit codes indicate various errors.
Nevertheless, we return the result of the program as the exit code.

%\begin{wrapfigure}{r}{2.25in}
\begin{figure}[tbp]
\begin{lstlisting}
	.globl main
main:
	movq	$10, %rax
	addq	$32, %rax
	retq
\end{lstlisting}
\caption{An x86 program equivalent to $\BINOP{+}{10}{32}$.}
\label{fig:p0-x86}
%\end{wrapfigure}
\end{figure}

Unfortunately, x86 varies in a couple ways depending on what operating
system it is assembled in. The code examples shown here are correct on
Linux and most Unix-like platforms, but when assembled on Mac OS X,
labels like \key{main} must be prefixed with an underscore, as in
\key{\_main}.

We exhibit the use of memory for storing intermediate results in the
next example.  Figure~\ref{fig:p1-x86} lists an x86 program that is
equivalent to $\BINOP{+}{52}{ \UNIOP{-}{10} }$. This program uses a
region of memory called the \emph{procedure call stack} (or
\emph{stack} for short). The stack consists of a separate \emph{frame}
for each procedure call. The memory layout for an individual frame is
shown in Figure~\ref{fig:frame}.  The register \key{rsp} is called the
\emph{stack pointer} and points to the item at the top of the
stack. The stack grows downward in memory, so we increase the size of
the stack by subtracting from the stack pointer. The frame size is
required to be a multiple of 16 bytes. In the context of a procedure
call, the \emph{return address} is the next instruction on the caller
side that comes after the call instruction. During a function call,
the return address is pushed onto the stack.  The register \key{rbp}
is the \emph{base pointer} which serves two purposes: 1) it saves the
location of the stack pointer for the calling procedure and 2) it is
used to access variables associated with the current procedure.  The
base pointer of the calling procedure is pushed onto the stack after
the return address. We number the variables from $1$ to $n$. Variable
$1$ is stored at address $-8\key{(\%rbp)}$, variable $2$ at
$-16\key{(\%rbp)}$, etc.

\begin{figure}[tbp]
\begin{lstlisting}
start:
	movq	$10, -8(%rbp)
	negq	-8(%rbp)
	movq	-8(%rbp), %rax
	addq	$52, %rax
	jmp conclusion

	.globl main
main:
	pushq	%rbp
	movq	%rsp, %rbp
	subq	$16, %rsp
	jmp start
conclusion:
	addq	$16, %rsp
	popq	%rbp
	retq
\end{lstlisting}
\caption{An x86 program equivalent to $\BINOP{+}{52}{\UNIOP{-}{10} }$.}
\label{fig:p1-x86}
\end{figure}


\begin{figure}[tbp]
\centering
\begin{tabular}{|r|l|} \hline
Position & Contents \\ \hline
8(\key{\%rbp}) & return address \\
0(\key{\%rbp}) & old \key{rbp} \\
-8(\key{\%rbp}) & variable $1$ \\
-16(\key{\%rbp}) & variable $2$ \\
 \ldots  & \ldots \\
0(\key{\%rsp}) & variable $n$\\ \hline
\end{tabular}

\caption{Memory layout of a frame.}
\label{fig:frame}
\end{figure}

Getting back to the program in Figure~\ref{fig:p1-x86}, the first
three instructions are the typical \emph{prelude} for a procedure.
The instruction \key{pushq \%rbp} saves the base pointer for the
procedure that called the current one onto the stack and subtracts $8$
from the stack pointer. The second instruction \key{movq \%rsp, \%rbp}
changes the base pointer to the top of the stack. The instruction
\key{subq \$16, \%rsp} moves the stack pointer down to make enough
room for storing variables.  This program just needs one variable ($8$
bytes) but because the frame size is required to be a multiple of 16
bytes, it rounds to 16 bytes.

The four instructions under the label \code{start} carry out the work
of computing $\BINOP{+}{52}{\UNIOP{-}{10} }$. The first instruction
\key{movq \$10, -8(\%rbp)} stores $10$ in variable $1$. The
instruction \key{negq -8(\%rbp)} changes variable $1$ to $-10$. The
\key{movq \$52, \%rax} places $52$ in the register \key{rax} and
\key{addq -8(\%rbp), \%rax} adds the contents of variable $1$ to
\key{rax}, at which point \key{rax} contains $42$.

The three instructions under the label \code{conclusion} are the
typical finale of a procedure.  The first two are necessary to get the
state of the machine back to where it was at the beginning of the
procedure.  The \key{addq \$16, \%rsp} instruction moves the stack
pointer back to point at the old base pointer. The amount added here
needs to match the amount that was subtracted in the prelude of the
procedure. Then \key{popq \%rbp} returns the old base pointer to
\key{rbp} and adds $8$ to the stack pointer.  The last instruction,
\key{retq}, jumps back to the procedure that called this one and adds
8 to the stack pointer, which returns the stack pointer to where it
was prior to the procedure call.

The compiler will need a convenient representation for manipulating
x86 programs, so we define an abstract syntax for x86 in
Figure~\ref{fig:x86-ast-a}. We refer to this language as $x86_0$ with
a subscript $0$ because later we introduce extended versions of this
assembly language. The main difference compared to the concrete syntax
of x86 (Figure~\ref{fig:x86-a}) is that it does nto allow labelled
instructions to appear anywhere, but instead organizes instructions
into groups called \emph{blocks} and a label is associated with every
block, which is why the \key{program} form includes an association
list mapping labels to blocks. The reason for this organization
becomes apparent in Chapter~\ref{ch:bool-types}.

\begin{figure}[tp]
\fbox{
\begin{minipage}{0.96\textwidth}
\[
\begin{array}{lcl}
\itm{register} &::=& \allregisters{} \\
\Arg &::=&  \INT{\Int} \mid \REG{\itm{register}}
    \mid (\key{deref}\;\itm{register}\;\Int) \\
\Instr &::=& (\key{addq} \; \Arg\; \Arg) \mid
             (\key{subq} \; \Arg\; \Arg) \mid
             (\key{movq} \; \Arg\; \Arg) \mid
             (\key{retq})\\
      &\mid& (\key{negq} \; \Arg) \mid
             (\key{callq} \; \mathit{label}) \mid
             (\key{pushq}\;\Arg) \mid
             (\key{popq}\;\Arg) \\
\Block &::= & (\key{block} \;\itm{info}\; \Instr^{+}) \\
x86_0 &::= & (\key{program} \;\itm{info} \; ((\itm{label} \,\key{.}\, \Block)^{+}))
\end{array}
\]
\end{minipage}
}
\caption{Abstract syntax for $x86_0$ assembly.}
\label{fig:x86-ast-a}
\end{figure}

\section{Planning the trip to x86 via the $C_0$ language}
\label{sec:plan-s0-x86}

To compile one language to another it helps to focus on the
differences between the two languages because the compiler will need
to bridge them. What are the differences between $R_1$ and x86
assembly? Here we list some of the most important ones.

\begin{enumerate}
\item[(a)] x86 arithmetic instructions typically have two arguments
  and update the second argument in place. In contrast, $R_1$
  arithmetic operations take two arguments and produce a new value.
  An x86 instruction may have at most one memory-accessing argument.
  Furthermore, some instructions place special restrictions on their
  arguments.

\item[(b)] An argument to an $R_1$ operator can be any expression,
  whereas x86 instructions restrict their arguments to be \emph{simple
    expressions} like integers, registers, and memory locations.  (All
  the other kinds are called \emph{complex expressions}.)

\item[(c)] The order of execution in x86 is explicit in the syntax: a
  sequence of instructions and jumps to labeled positions, whereas in
  $R_1$ it is a left-to-right depth-first traversal of the abstract
  syntax tree.

\item[(d)] An $R_1$ program can have any number of variables whereas
  x86 has 16 registers and the procedure calls stack.

\item[(e)] Variables in $R_1$ can overshadow other variables with the
  same name. The registers and memory locations of x86 all have unique
  names or addresses.
\end{enumerate}

We ease the challenge of compiling from $R_1$ to x86 by breaking down
the problem into several steps, dealing with the above differences one
at a time.  Each of these steps is called a \emph{pass} of the
compiler, because step traverses (passes over) the AST of the program.
%
We begin by giving a sketch about how we might implement each pass,
and give them names.  We shall then figure out an ordering of the
passes and the input/output language for each pass. The very first
pass has $R_1$ as its input language and the last pass has x86 as its
output language. In between we can choose whichever language is most
convenient for expressing the output of each pass, whether that be
$R_1$, x86, or new \emph{intermediate languages} of our own design.
Finally, to implement the compiler, we shall write one function,
typically a structural recursive function, per pass.

\begin{description}
\item[Pass \key{select-instructions}] To handle the difference between
  $R_1$ operations and x86 instructions we shall convert each $R_1$
  operation to a short sequence of instructions that accomplishes the
  same task.

\item[Pass \key{remove-complex-opera*}] To ensure that each
  subexpression (i.e. operator and operand, and hence \key{opera*}) is
  a simple expression, we shall introduce temporary variables to hold
  the results of subexpressions.
  
\item[Pass \key{explicate-control}] To make the execution order of the
  program explicit, we shall convert from the abstract syntax tree
  representation into a graph representation in which each node
  contains a sequence of actions and the edges say where to go after
  the sequence is complete.

\item[Pass \key{assign-homes}] To handle the difference between the
  variables in $R_1$ versus the registers and stack location in x86,
  we shall come up with an assignment of each variable to its
  \emph{home}, that is, to a register or stack location.

\item[Pass \key{uniquify}] This pass deals with the shadowing of variables
  by renaming every variable to a unique name, so that shadowing no
  longer occurs.

\end{description}

The next question is: in what order should we apply these passes? This
question can be a challenging one to answer because it is difficult to
know ahead of time which orders will be better (easier to implement,
produce more efficient code, etc.) so often some trial-and-error is
involved. Nevertheless, we can try to plan ahead and make educated
choices regarding the orderings.

Let us consider the ordering of \key{uniquify} and
\key{remove-complex-opera*}. The assignment of subexpressions to
temporary variables involves introducing new variables and moving
subexpressions, which might change the shadowing of variables and
inadvertently change the behavior of the program.  But if we apply
\key{uniquify} first, this will not be an issue. Of course, this means
that in \key{remove-complex-opera*}, we need to ensure that the
temporary variables that it creates are unique.

Next we shall consider the ordering of the \key{explicate-control}
pass and \key{select-instructions}. It is clear that
\key{explicate-control} must come first because the control-flow graph
that it generates is needed when determining where to place the x86
label and jump instructions.
%
Regarding the ordering of \key{explicate-control} with respect to
\key{uniquify}, it is important to apply \key{uniquify} first because
in \key{explicate-control} we change all the \key{let}-bound variables
to become local variables whose scope is the entire program.
%
With respect to \key{remove-complex-opera*}, it perhaps does not
matter very much, but it works well to place \key{explicate-control}
after removing complex subexpressions.

The \key{assign-homes} pass should come after
\key{remove-complex-opera*} and \key{explicate-control}.  The
\key{remove-complex-opera*} pass generates temporary variables, which
also need to be assigned homes. The \key{explicate-control} pass
deletes branches that will never be executed, which can remove
variables. Thus it is good to place \key{explicate-control} prior to
\key{assign-homes} so that there are fewer variables that need to be
assigned homes. This is important because the \key{assign-homes} pass
has the highest time complexity.

Last, we need to decide on the ordering of \key{select-instructions}
and \key{assign-homes}.  These two issues are intertwined, creating a
bit of a Gordian Knot. To do a good job of assigning homes, it is
helpful to have already determined which instructions will be used,
because x86 instructions have restrictions about which of their
arguments can be registers versus stack locations. For example, one
can give preferential treatment to variables that occur in
register-argument positions. On the other hand, it may turn out to be
impossible to make sure that all such variables are assigned to
registers, and then one must redo the selection of instructions. Some
compilers handle this problem by iteratively repeating these two
passes until a good solution is found.  We shall use a simpler
approach in which \key{select-instructions} comes first, followed by
the \key{assign-homes}, followed by a third pass, named
\key{patch-instructions}, that uses a reserved register (\key{rax}) to
patch-up outstanding problems regarding instructions with too many
memory accesses.

\begin{figure}[tbp]
\begin{tikzpicture}[baseline=(current  bounding  box.center)]
\node (R1) at (0,2)  {\large $R_1$};
\node (R1-2) at (3,2)  {\large $R_1$};
\node (R1-3) at (6,2)  {\large $R_1$};
\node (C0-1) at (6,0)  {\large $C_0$};
\node (C0-2) at (3,0)  {\large $C_0$};

\node (x86-2) at (3,-2)  {\large $\text{x86}^{*}_0$};
\node (x86-3) at (6,-2)  {\large $\text{x86}^{*}_0$};
\node (x86-4) at (9,-2) {\large $\text{x86}_0$};
\node (x86-5) at (12,-2) {\large $\text{x86}^{\dagger}_0$};

\path[->,bend left=15] (R1) edge [above] node {\ttfamily\footnotesize uniquify} (R1-2);
\path[->,bend left=15] (R1-2) edge [above] node {\ttfamily\footnotesize remove-complex.} (R1-3);
\path[->,bend left=15] (R1-3) edge [right] node {\ttfamily\footnotesize explicate-control} (C0-1);
\path[->,bend right=15] (C0-1) edge [above] node {\ttfamily\footnotesize uncover-locals} (C0-2);
\path[->,bend right=15] (C0-2) edge [left] node {\ttfamily\footnotesize select-instr.} (x86-2);
\path[->,bend left=15] (x86-2) edge [above] node {\ttfamily\footnotesize assign-homes} (x86-3);
\path[->,bend left=15] (x86-3) edge [above] node {\ttfamily\footnotesize patch-instr.} (x86-4);
\path[->,bend left=15] (x86-4) edge [above] node {\ttfamily\footnotesize print-x86} (x86-5);
\end{tikzpicture}

\caption{Overview of the passes for compiling $R_1$. }
\label{fig:R1-passes}
\end{figure}

Figure~\ref{fig:R1-passes} presents the ordering of the compiler
passes in the form of a graph. Each pass is an edge and the
input/output language of each pass is a node in the graph.  The output
of \key{uniquify} and \key{remove-complex-opera*} are programs that
are still in the $R_1$ language, but the output of the pass
\key{explicate-control} is in a different language that is designed to
make the order of evaluation explicit in its syntax, which we
introduce in the next section. Also, there are two passes of lesser
importance in Figure~\ref{fig:R1-passes} that we have not yet talked
about, \key{uncover-locals} and \key{print-x86}. We shall discuss them
later in this Chapter.

\subsection{The $C_0$ Intermediate Language}

It so happens that the output of \key{explicate-control} is vaguely
similar to the $C$ language~\citep{Kernighan:1988nx}, so we name it
$C_0$.  The syntax for $C_0$ is defined in Figure~\ref{fig:c0-syntax}.
%
The $C_0$ language supports the same operators as $R_1$ but the
arguments of operators are now restricted to just variables and
integers, thanks to the \key{remove-complex-opera*} pass.  In the
literature this style of intermediate language is called
administrative normal form, or ANF for
short~\citep{Danvy:1991fk,Flanagan:1993cg}.  Instead of \key{let}
expressions, $C_0$ has assignment statements which can be executed in
sequence using the \key{seq} construct. A sequence of statements
always ends with \key{return}, a guarantee that is baked into the
grammar rules for the \itm{tail} non-terminal. The naming of this
non-terminal comes from the term \emph{tail position}, which refers to
an expression that is the last one to execute within a function. (A
expression in tail position may contain subexpressions, and those may
or may not be in tail position depending on the kind of expression.)

A $C_0$ program consists of an association list mapping labels to
tails. This is overkill for the present Chapter, as we do not yet need
to introduce \key{goto} for jumping to labels, but it saves us from
having to change the syntax of the program construct in
Chapter~\ref{ch:bool-types}.  For now there will be just one label,
\key{start}, and the whole program will be it's tail.
%
The $\itm{info}$ field of the program construt, after the
\key{uncover-locals} pass, will contain a mapping from the symbol
\key{locals} to a list of variables, that is, a list of all the
variables used in the program. At the start of the program, these
variables are uninitialized (they contain garbage) and each variable
becomes initialized on its first assignment.

\begin{figure}[tbp]
\fbox{
\begin{minipage}{0.96\textwidth}
\[
\begin{array}{lcl}
\Arg &::=& \Int \mid \Var \\
\Exp &::=& \Arg \mid (\key{read}) \mid (\key{-}\;\Arg) \mid (\key{+} \; \Arg\;\Arg)\\
\Stmt &::=& \ASSIGN{\Var}{\Exp} \\
\Tail &::= & \RETURN{\Exp} \mid (\key{seq}\; \Stmt\; \Tail) \\
C_0 & ::= & (\key{program}\;\itm{info}\;((\itm{label}\,\key{.}\,\Tail)^{+}))
\end{array}
\]
\end{minipage}
}
\caption{The $C_0$ intermediate language.}
\label{fig:c0-syntax}
\end{figure}


%% The \key{select-instructions} pass is optimistic in the sense that it
%% treats variables as if they were all mapped to registers. The
%% \key{select-instructions} pass generates a program that consists of
%% x86 instructions but that still uses variables, so it is an
%% intermediate language that is technically different than x86, which
%% explains the asterisks in the diagram above.

%% In this Chapter we shall take the easy road to implementing
%% \key{assign-homes} and simply map all variables to stack locations.
%% The topic of Chapter~\ref{ch:register-allocation-r1} is implementing a
%% smarter approach in which we make a best-effort to map variables to
%% registers, resorting to the stack only when necessary.

%% Once variables have been assigned to their homes, we can finalize the
%% instruction selection by dealing with an idiosyncrasy of x86
%% assembly. Many x86 instructions have two arguments but only one of the
%% arguments may be a memory reference (and the stack is a part of
%% memory).  Because some variables may get mapped to stack locations,
%% some of our generated instructions may violate this restriction.  The
%% purpose of the \key{patch-instructions} pass is to fix this problem by
%% replacing every violating instruction with a short sequence of
%% instructions that use the \key{rax} register. Once we have implemented
%% a good register allocator (Chapter~\ref{ch:register-allocation-r1}), the
%% need to patch instructions will be relatively rare.

\subsection{The dialects of x86}

The x86$^{*}_0$ language, pronounced ``pseudo-x86'', is the output of
the pass \key{select-instructions}. It extends $x86_0$ with variables
and looser rules regarding instruction arguments. The x86$^{\dagger}$
language, the output of \key{print-x86}, is the concrete syntax for
x86.


\section{Uniquify Variables}
\label{sec:uniquify-s0}

The purpose of this pass is to make sure that each \key{let} uses a
unique variable name. For example, the \code{uniquify} pass should
translate the program on the left into the program on the right. \\
\begin{tabular}{lll}
\begin{minipage}{0.4\textwidth}
\begin{lstlisting}
 (program ()
   (let ([x 32])
     (+ (let ([x 10]) x) x)))
\end{lstlisting}
\end{minipage}
&
$\Rightarrow$
&
\begin{minipage}{0.4\textwidth}
\begin{lstlisting}
(program ()
  (let ([x.1 32])
    (+ (let ([x.2 10]) x.2) x.1)))
\end{lstlisting}
\end{minipage}
\end{tabular} \\
%
The following is another example translation, this time of a program
with a \key{let} nested inside the initializing expression of another
\key{let}.\\
\begin{tabular}{lll}
\begin{minipage}{0.4\textwidth}
\begin{lstlisting}
(program ()
  (let ([x (let ([x 4])
             (+ x 1))])
    (+ x 2)))
\end{lstlisting}
\end{minipage}
&
$\Rightarrow$
&
\begin{minipage}{0.4\textwidth}
\begin{lstlisting}
(program ()
  (let ([x.2 (let ([x.1 4])
               (+ x.1 1))])
    (+ x.2 2)))
\end{lstlisting}
\end{minipage}
\end{tabular}

We recommend implementing \code{uniquify} as a structurally recursive
function that mostly copies the input program. However, when
encountering a \key{let}, it should generate a unique name for the
variable (the Racket function \code{gensym} is handy for this) and
associate the old name with the new unique name in an association
list. The \code{uniquify} function will need to access this
association list when it gets to a variable reference, so we add
another parameter to \code{uniquify} for the association list. It is
quite common for a compiler pass to need a map to store extra
information about variables. Such maps are often called \emph{symbol
  tables}.

The skeleton of the \code{uniquify} function is shown in
Figure~\ref{fig:uniquify-s0}.  The function is curried so that it is
convenient to partially apply it to an association list and then apply
it to different expressions, as in the last clause for primitive
operations in Figure~\ref{fig:uniquify-s0}. In the last \key{match}
clause for the primitive operators, note the use of the comma-@
operator to splice a list of S-expressions into an enclosing
S-expression.

\begin{exercise}
\normalfont % I don't like the italics for exercises. -Jeremy

Complete the \code{uniquify} pass by filling in the blanks, that is,
implement the clauses for variables and for the \key{let} construct.
\end{exercise}

\begin{figure}[tbp]
\begin{lstlisting}
   (define (uniquify-exp alist)
     (lambda (e)
       (match e
         [(? symbol?) ___]
         [(? integer?) e]
         [`(let ([,x ,e]) ,body) ___]
         [`(,op ,es ...)
          `(,op ,@(for/list ([e es]) ((uniquify-exp alist) e)))]
         )))

   (define (uniquify alist)
     (lambda (e)
       (match e
         [`(program ,info ,e)
          `(program ,info ,((uniquify-exp alist) e))]
         )))
\end{lstlisting}
\caption{Skeleton for the \key{uniquify} pass.}
\label{fig:uniquify-s0}
\end{figure}

\begin{exercise}
\normalfont % I don't like the italics for exercises. -Jeremy

Test your \key{uniquify} pass by creating five example $R_1$ programs
and checking whether the output programs produce the same result as
the input programs. The $R_1$ programs should be designed to test the
most interesting parts of the \key{uniquify} pass, that is, the
programs should include \key{let} constructs, variables, and variables
that overshadow each other.  The five programs should be in a
subdirectory named \key{tests} and they should have the same file name
except for a different integer at the end of the name, followed by the
ending \key{.rkt}.  Use the \key{interp-tests} function
(Appendix~\ref{appendix:utilities}) from \key{utilities.rkt} to test
your \key{uniquify} pass on the example programs.

\end{exercise}

\section{Remove Complex Operators and Operands}
\label{sec:remove-complex-opera-r1}

The \code{remove-complex-opera*} pass will transform $R_1$ programs so
that the arguments of operations are simple expressions.  Put another
way, this pass removes complex subexpressions, such as the expression
\code{(- 10)} in the program below. This is accomplished by
introducing a new \key{let}-bound variable, binding the complex
subexpression to the new variable, and then using the new variable in
place of the complex expression, as shown in the output of
\code{remove-complex-opera*} on the right.\\
\begin{tabular}{lll}
\begin{minipage}{0.4\textwidth}
% s0_19.rkt
\begin{lstlisting}
 (program ()
   (+ 52 (- 10)))
\end{lstlisting}
\end{minipage}
&
$\Rightarrow$
&
\begin{minipage}{0.4\textwidth}
\begin{lstlisting}
(program ()
  (let ([tmp.1 (- 10)])
    (+ 52 tmp.1)))
\end{lstlisting}
\end{minipage}
\end{tabular}

We recommend implementing this pass with two mutually recursive
functions, \code{rco-arg} and \code{rco-exp}. The idea is to apply
\code{rco-arg} to subexpressions that need to become simple and to
apply \code{rco-exp} to subexpressions can stay complex.  
Both functions take an expression in $R_1$ as input.
The \code{rco-exp} function returns an expression.
The \code{rco-arg} function returns two things:
a simple expression and association list mapping temporary variables
to complex subexpressions. You can return multiple things from a
function using Racket's \key{values} form and you can receive multiple
things from a function call using the \key{define-values} form. If you
are not familiar with these constructs, the Racket documentation will
be of help.  Also, the \key{for/lists} construct is useful for
applying a function to each element of a list, in the case where the
function returns multiple values.

\begin{tabular}{lll}
\begin{minipage}{0.4\textwidth}
\begin{lstlisting}
(rco-arg `(- 10))
\end{lstlisting}
\end{minipage}
&
$\Rightarrow$
&
\begin{minipage}{0.4\textwidth}
\begin{lstlisting}
  (values `tmp.1
           `((tmp.1 . (- 10))))
\end{lstlisting}
\end{minipage}
\end{tabular}

Take special care of programs such as the following that
\key{let}-bind variables with integers or other variables. It should
leave them unchanged, as shown in to the program on the right \\
\begin{tabular}{lll}
\begin{minipage}{0.4\textwidth}
\begin{lstlisting}
(program ()
  (let ([a 42])
    (let ([b a])
      b)))
\end{lstlisting}
\end{minipage}
&
$\Rightarrow$
&
\begin{minipage}{0.4\textwidth}
\begin{lstlisting}
(program ()
  (let ([a 42])
    (let ([b a])
      b)))
\end{lstlisting}
\end{minipage}
\end{tabular} \\
and not translate them to the following, which might result from a
careless implementation of \key{rco-exp} and \key{rco-arg}.

\begin{minipage}{0.4\textwidth}
\begin{lstlisting}
   (program ()
     (let ([tmp.1 42])
       (let ([a tmp.1])
         (let ([tmp.2 a])
           (let ([b tmp.2])
             b)))))
\end{lstlisting}
\end{minipage}

\begin{exercise}
\normalfont Implement the \code{remove-complex-opera*} pass and test
it on all of the example programs that you created to test the
\key{uniquify} pass and create three new example programs that are
designed to exercise all of the interesting code in the
\code{remove-complex-opera*} pass. Use the \key{interp-tests} function
(Appendix~\ref{appendix:utilities}) from \key{utilities.rkt} to test
your passes on the example programs.
\end{exercise}


\section{Explicate Control}
\label{sec:explicate-control-r1}

The \code{explicate-control} pass makes the order of execution
explicit in the syntax of the program. For $R_1$, this amounts to
flattening \key{let} constructs into a sequence of assignment
statements. For example, consider the following $R_1$ program.
% s0_11.rkt
\begin{lstlisting}
(program ()
  (let ([y (let ([x 20])
	   (+ x (let ([x 22]) x)))])
    y))
\end{lstlisting}
%
The output of \code{remove-complex-opera*} is shown below, on the
left.  The right-hand-side of a \key{let} executes before its body, so
the order of evaluation for this program is to assign \code{20} to
\code{x.1}, assign \code{22} to \code{x.2}, assign \code{(+ x.1 x.2)}
to \code{y}, then return \code{y}. Indeed, the result of
\code{explicate-control} produces code in the $C_0$ language that
makes this explicit.\\
\begin{tabular}{lll}
\begin{minipage}{0.4\textwidth}
\begin{lstlisting}
(program ()
  (let ([y (let ([x.1 20]) 
             (let ([x.2 22])
               (+ x.1 x.2)))])
   y))
\end{lstlisting}
\end{minipage}
&
$\Rightarrow$
&
\begin{minipage}{0.4\textwidth}
\begin{lstlisting}
(program ()
  ((start . 
   (seq (assign x.1 20)
   (seq (assign x.2 22)
   (seq (assign y (+ x.1 x.2))
   (return y)))))))
\end{lstlisting}
\end{minipage}
\end{tabular}

We recommend implementing \code{explicate-control} using two mutually
recursive functions: \code{explicate-control-tail} and
\code{explicate-control-assign}.  The \code{explicate-control-tail}
function should be applied to expressions in tail position, whereas
\code{explicate-control-assign} should be applied to expressions that
occur on the right-hand-side of a \code{let}.  The function
\code{explicate-control-tail} takes an $R_1$ expression as input and
produces a $C_0$ $\Tail$ (see the grammar in
Figure~\ref{fig:c0-syntax}).  The \code{explicate-control-assign}
function takes an $R_1$ expression, the variable that it is to be
assigned to, and $C_0$ code (a $\Tail$) that should come after the
assignment (e.g., the code generated for the body of the \key{let}).

\section{Uncover Locals}
\label{sec:uncover-locals-r1}

The pass \code{uncover-locals} simply collects all of the variables in
the program and places then in the $\itm{info}$ of the program
construct. Here is the output for the example program of the last
section.

\begin{minipage}{0.4\textwidth}
\begin{lstlisting}
(program ((locals . (x.1 x.2 y)))
  ((start . 
   (seq (assign x.1 20)
   (seq (assign x.2 22)
   (seq (assign y (+ x.1 x.2))
   (return y)))))))
\end{lstlisting}
\end{minipage}

\section{Select Instructions}
\label{sec:select-r1}

In the \code{select-instructions} pass we begin the work of
translating from $C_0$ to x86. The target language of this pass is a
pseudo-x86 language that still uses variables, so we add an AST node
of the form $\VAR{\itm{var}}$ to the x86 abstract syntax.  We
recommend implementing the \code{select-instructions} in terms of
three auxilliary functions, one for each of the non-terminals of
$C_0$: $\Arg$, $\Stmt$, and $\Tail$.

The cases for $\itm{arg}$ are straightforward, simply putting
variables and integer literals into the s-expression format expected
of pseudo-x86, \code{(var $x$)} and \code{(int $n$)}, respectively.

Next we discuss some of the cases for $\itm{stmt}$, starting with
arithmetic operations. For example, in $C_0$ an addition operation can
take the form below.  To translate to x86, we need to use the
\key{addq} instruction which does an in-place update. So we must first
move \code{10} to \code{x}. \\
\begin{tabular}{lll}
\begin{minipage}{0.4\textwidth}
\begin{lstlisting}
 (assign x (+ 10 32))
\end{lstlisting}
\end{minipage}
&
$\Rightarrow$
&
\begin{minipage}{0.4\textwidth}
\begin{lstlisting}
   (movq (int 10) (var x))
   (addq (int 32) (var x))
\end{lstlisting}
\end{minipage}
\end{tabular} \\
%
There are some cases that require special care to avoid generating
needlessly complicated code. If one of the arguments is the same as
the left-hand side of the assignment, then there is no need for the
extra move instruction.  For example, the following assignment
statement can be translated into a single \key{addq} instruction.\\
\begin{tabular}{lll}
\begin{minipage}{0.4\textwidth}
\begin{lstlisting}
 (assign x (+ 10 x))
\end{lstlisting}
\end{minipage}
&
$\Rightarrow$
&
\begin{minipage}{0.4\textwidth}
\begin{lstlisting}
(addq (int 10) (var x))
\end{lstlisting}
\end{minipage}
\end{tabular} \\

The \key{read} operation does not have a direct counterpart in x86
assembly, so we have instead implemented this functionality in the C
language, with the function \code{read\_int} in the file
\code{runtime.c}. In general, we refer to all of the functionality in
this file as the \emph{runtime system}, or simply the \emph{runtime}
for short. When compiling your generated x86 assembly code, you
will need to compile \code{runtime.c} to \code{runtime.o} (an ``object
file'', using \code{gcc} option \code{-c}) and link it into the final
executable. For our purposes of code generation, all you need to do is
translate an assignment of \key{read} to some variable $\itm{lhs}$
(for left-hand side) into a call to the \code{read\_int} function
followed by a move from \code{rax} to the left-hand side.  The move
from \code{rax} is needed because the return value from
\code{read\_int} goes into \code{rax}, as is the case in general.  \\
\begin{tabular}{lll}
\begin{minipage}{0.4\textwidth}
\begin{lstlisting}
 (assign |$\itm{lhs}$| (read))
\end{lstlisting}
\end{minipage}
&
$\Rightarrow$
&
\begin{minipage}{0.4\textwidth}
\begin{lstlisting}
(callq read_int)
(movq (reg rax) (var |$\itm{lhs}$|))
\end{lstlisting}
\end{minipage}
\end{tabular} \\

There are two cases for the $\Tail$ non-terminal: \key{return} and
\key{seq}. Regarding \RETURN{e}, we recommend treating it as an
assignment to the \key{rax} register followed by a jump to the
conclusion of the program (so the conclusion needs to be labeled).
For $(\key{seq}\,s\,t)$, we simply process the statement $s$ and tail
$t$ recursively and append the resulting instructions.

\begin{exercise}
\normalfont
Implement the \key{select-instructions} pass and test it on all of the
example programs that you created for the previous passes and create
three new example programs that are designed to exercise all of the
interesting code in this pass. Use the \key{interp-tests} function
(Appendix~\ref{appendix:utilities}) from \key{utilities.rkt} to test
your passes on the example programs.
\end{exercise}

\section{Assign Homes}
\label{sec:assign-r1}

As discussed in Section~\ref{sec:plan-s0-x86}, the
\key{assign-homes} pass places all of the variables on the stack.
Consider again the example $R_1$ program \code{(+ 52 (- 10))},
which after \key{select-instructions} looks like the following.
\begin{lstlisting}
   (movq (int 10) (var tmp.1))
   (negq (var tmp.1))
   (movq (var tmp.1) (var tmp.2))
   (addq (int 52) (var tmp.2))
   (movq (var tmp.2) (reg rax)))
\end{lstlisting}
The variable \code{tmp.1} is assigned to stack location
\code{-8(\%rbp)}, and \code{tmp.2} is assign to \code{-16(\%rbp)}, so
the \code{assign-homes} pass translates the above to
\begin{lstlisting}
   (movq (int 10) (deref rbp -8))
   (negq (deref rbp -8))
   (movq (deref rbp -8) (deref rbp -16))
   (addq (int 52) (deref rbp -16))
   (movq (deref rbp -16) (reg rax)))
\end{lstlisting}

In the process of assigning stack locations to variables, it is
convenient to compute and store the size of the frame (in bytes) in
the $\itm{info}$ field of the \key{program} node, with the key
\code{stack-space}, which will be needed later to generate the
procedure conclusion.  Some operating systems place restrictions on
the frame size. For example, Mac OS X requires the frame size to be a
multiple of 16 bytes.

\begin{exercise}
\normalfont Implement the \key{assign-homes} pass and test it on all
of the example programs that you created for the previous passes pass.
We recommend that \key{assign-homes} take an extra parameter that is a
mapping of variable names to homes (stack locations for now).  Use the
\key{interp-tests} function (Appendix~\ref{appendix:utilities}) from
\key{utilities.rkt} to test your passes on the example programs.
\end{exercise}

\section{Patch Instructions}
\label{sec:patch-s0}

The purpose of this pass is to make sure that each instruction adheres
to the restrictions regarding which arguments can be memory
references. For most instructions, the rule is that at most one
argument may be a memory reference.

Consider again the following example.
\begin{lstlisting}
   (let ([a 42])
     (let ([b a])
       b))
\end{lstlisting}
After \key{assign-homes} pass, the above has been translated to
\begin{lstlisting}
   (movq (int 42) (deref rbp -8))
   (movq (deref rbp -8) (deref rbp -16))
   (movq (deref rbp -16) (reg rax))
   (jmp conclusion)
\end{lstlisting}
The second \key{movq} instruction is problematic because both
arguments are stack locations. We suggest fixing this problem by
moving from the source to the register \key{rax} and then from
\key{rax} to the destination, as follows.
\begin{lstlisting}
   (movq (int 42) (deref rbp -8))
   (movq (deref rbp -8) (reg rax))
   (movq (reg rax) (deref rbp -16))
   (movq (deref rbp -16) (reg rax))
\end{lstlisting}

\begin{exercise}
\normalfont
Implement the \key{patch-instructions} pass and test it on all of the
example programs that you created for the previous passes and create
three new example programs that are designed to exercise all of the
interesting code in this pass. Use the \key{interp-tests} function
(Appendix~\ref{appendix:utilities}) from \key{utilities.rkt} to test
your passes on the example programs.
\end{exercise}


\section{Print x86}
\label{sec:print-x86}

The last step of the compiler from $R_1$ to x86 is to convert the x86
AST (defined in Figure~\ref{fig:x86-ast-a}) to the string
representation (defined in Figure~\ref{fig:x86-a}). The Racket
\key{format} and \key{string-append} functions are useful in this
regard. The main work that this step needs to perform is to create the
\key{main} function and the standard instructions for its prelude and
conclusion, as shown in Figure~\ref{fig:p1-x86} of
Section~\ref{sec:x86}. You need to know the number of stack-allocated
variables, so we suggest computing it in the \key{assign-homes} pass
(Section~\ref{sec:assign-r1}) and storing it in the $\itm{info}$ field
of the \key{program} node.

%% Your compiled code should print the result of the program's execution
%% by using the \code{print\_int} function provided in
%% \code{runtime.c}. If your compiler has been implemented correctly so
%% far, this final result should be stored in the \key{rax} register.
%% We'll talk more about how to perform function calls with arguments in
%% general later on, but for now, place the following after the compiled
%% code for the $R_1$ program but before the conclusion:

%% \begin{lstlisting}
%%     movq %rax, %rdi
%%     callq print_int
%% \end{lstlisting}

%% These lines move the value in \key{rax} into the \key{rdi} register, which
%% stores the first argument to be passed into \key{print\_int}.

If you want your program to run on Mac OS X, your code needs to
determine whether or not it is running on a Mac, and prefix
underscores to labels like \key{main}.  You can determine the platform
with the Racket call \code{(system-type 'os)}, which returns
\code{'macosx}, \code{'unix}, or \code{'windows}.  
%% In addition to
%% placing underscores on \key{main}, you need to put them in front of
%% \key{callq} labels (so \code{callq print\_int} becomes \code{callq
%%   \_print\_int}).

\begin{exercise}
\normalfont Implement the \key{print-x86} pass and test it on all of
the example programs that you created for the previous passes. Use the
\key{compiler-tests} function (Appendix~\ref{appendix:utilities}) from
\key{utilities.rkt} to test your complete compiler on the example
programs.
% The following is specific to P423/P523. -Jeremy
%Mac support is optional, but your compiler has to output
%valid code for Unix machines.
\end{exercise}


\margincomment{\footnotesize To do: add a challenge section. Perhaps
  extending the partial evaluation to $R_0$? \\ --Jeremy}

%%%%%%%%%%%%%%%%%%%%%%%%%%%%%%%%%%%%%%%%%%%%%%%%%%%%%%%%%%%%%%%%%%%%%%%%%%%%%%%%
\chapter{Register Allocation}
\label{ch:register-allocation-r1}

In Chapter~\ref{ch:int-exp} we simplified the generation of x86
assembly by placing all variables on the stack. We can improve the
performance of the generated code considerably if we instead place as
many variables as possible into registers.  The CPU can access a
register in a single cycle, whereas accessing the stack takes many
cycles to go to cache or many more to access main memory.
Figure~\ref{fig:reg-eg} shows a program with four variables that
serves as a running example. We show the source program and also the
output of instruction selection. At that point the program is almost
x86 assembly but not quite; it still contains variables instead of
stack locations or registers.

\begin{figure}
\begin{minipage}{0.45\textwidth}
$R_1$ program:
% s0_22.rkt
\begin{lstlisting}
(program ()
  (let ([v 1])
  (let ([w 46])
  (let ([x (+ v 7)])
  (let ([y (+ 4 x)])
  (let ([z (+ x w)])
       (+ z (- y))))))))
\end{lstlisting}
\end{minipage}
\begin{minipage}{0.45\textwidth}
After instruction selection:
\begin{lstlisting}
(program
 ((locals . (v w x y z t.1)))
 ((start .
   (block ()
     (movq (int 1) (var v))
     (movq (int 46) (var w))
     (movq (var v) (var x))
     (addq (int 7) (var x))
     (movq (var x) (var y))
     (addq (int 4) (var y))
     (movq (var x) (var z))
     (addq (var w) (var z))
     (movq (var y) (var t.1))
     (negq (var t.1))
     (movq (var z) (reg rax))
     (addq (var t.1) (reg rax))
     (jmp conclusion)))))
\end{lstlisting}
\end{minipage}
\caption{An example program for register allocation.}
\label{fig:reg-eg}
\end{figure}

The goal of register allocation is to fit as many variables into
registers as possible. It is often the case that we have more
variables than registers, so we cannot map each variable to a
different register. Fortunately, it is common for different variables
to be needed during different periods of time, and in such cases
several variables can be mapped to the same register.  Consider
variables \code{x} and \code{y} in Figure~\ref{fig:reg-eg}.  After the
variable \code{x} is moved to \code{z} it is no longer needed.
Variable \code{y}, on the other hand, is used only after this point,
so \code{x} and \code{y} could share the same register. The topic of
Section~\ref{sec:liveness-analysis-r1} is how we compute where a variable
is needed.  Once we have that information, we compute which variables
are needed at the same time, i.e., which ones \emph{interfere}, and
represent this relation as graph whose vertices are variables and
edges indicate when two variables interfere with eachother
(Section~\ref{sec:build-interference}). We then model register
allocation as a graph coloring problem, which we discuss in
Section~\ref{sec:graph-coloring}.

In the event that we run out of registers despite these efforts, we
place the remaining variables on the stack, similar to what we did in
Chapter~\ref{ch:int-exp}. It is common to say that when a variable
that is assigned to a stack location, it has been \emph{spilled}. The
process of spilling variables is handled as part of the graph coloring
process described in \ref{sec:graph-coloring}.

\section{Registers and Calling Conventions}
\label{sec:calling-conventions}

As we perform register allocation, we will need to be aware of the
conventions that govern the way in which registers interact with
function calls. The convention for x86 is that the caller is
responsible for freeing up some registers, the \emph{caller-saved
  registers}, prior to the function call, and the callee is
responsible for saving and restoring some other registers, the
\emph{callee-saved registers}, before and after using them. The
caller-saved registers are
\begin{lstlisting}
  rax rdx rcx rsi rdi r8 r9 r10 r11
\end{lstlisting}
while the callee-saved registers are
\begin{lstlisting}
  rsp rbp rbx r12 r13 r14 r15
\end{lstlisting}
Another way to think about this caller/callee convention is the
following. The caller should assume that all the caller-saved registers
get overwritten with arbitrary values by the callee.  On the other
hand, the caller can safely assume that all the callee-saved registers
contain the same values after the call that they did before the call.
The callee can freely use any of the caller-saved registers.  However,
if the callee wants to use a callee-saved register, the callee must
arrange to put the original value back in the register prior to
returning to the caller, which is usually accomplished by saving and
restoring the value from the stack.


\section{Liveness Analysis}
\label{sec:liveness-analysis-r1}

A variable is \emph{live} if the variable is used at some later point
in the program and there is not an intervening assignment to the
variable.
%
To understand the latter condition, consider the following code
fragment in which there are two writes to \code{b}. Are \code{a} and
\code{b} both live at the same time?
\begin{lstlisting}[numbers=left,numberstyle=\tiny]
   (movq (int 5) (var a))
   (movq (int 30) (var b))
   (movq (var a) (var c))
   (movq (int 10) (var b))
   (addq (var b) (var c))
\end{lstlisting}
The answer is no because the value \code{30} written to \code{b} on
line 2 is never used. The variable \code{b} is read on line 5 and
there is an intervening write to \code{b} on line 4, so the read on
line 5 receives the value written on line 4, not line 2.

The live variables can be computed by traversing the instruction
sequence back to front (i.e., backwards in execution order).  Let
$I_1,\ldots, I_n$ be the instruction sequence. We write
$L_{\mathsf{after}}(k)$ for the set of live variables after
instruction $I_k$ and $L_{\mathsf{before}}(k)$ for the set of live
variables before instruction $I_k$. The live variables after an
instruction are always the same as the live variables before the next
instruction.
\begin{equation*}
  L_{\mathsf{after}}(k) = L_{\mathsf{before}}(k+1)
\end{equation*}
To start things off, there are no live variables after the last
instruction, so
\begin{equation*}
  L_{\mathsf{after}}(n) = \emptyset
\end{equation*}
We then apply the following rule repeatedly, traversing the
instruction sequence back to front.
\begin{equation*}
  L_{\mathtt{before}}(k) = (L_{\mathtt{after}}(k) - W(k)) \cup R(k),
\end{equation*}
where $W(k)$ are the variables written to by instruction $I_k$ and
$R(k)$ are the variables read by instruction $I_k$.
Figure~\ref{fig:live-eg} shows the results of live variables analysis
for the running example, with each instruction aligned with its
$L_{\mathtt{after}}$ set to make the figure easy to read.

\margincomment{JM: I think you should walk through the explanation of this formula,
   connecting it back to the example from before. \\
   JS: Agreed.}

\begin{figure}[tbp]
\hspace{20pt}
\begin{minipage}{0.45\textwidth}
\begin{lstlisting}[numbers=left]
(block ()
  (movq (int 1) (var v))
  (movq (int 46) (var w))
  (movq (var v) (var x))
  (addq (int 7) (var x))
  (movq (var x) (var y))
  (addq (int 4) (var y))
  (movq (var x) (var z))
  (addq (var w) (var z))
  (movq (var y) (var t.1))
  (negq (var t.1))
  (movq (var z) (reg rax))
  (addq (var t.1) (reg rax))
  (jmp conclusion))
\end{lstlisting}
\end{minipage}
\vrule\hspace{10pt}
\begin{minipage}{0.45\textwidth}
\begin{lstlisting}
|$\{\}$|
|$\{v \}$|
|$\{v,w\}$|
|$\{w,x\}$|
|$\{w,x\}$|
|$\{w,x,y\}$|
|$\{w,x,y\}$|
|$\{w,y,z\}$|
|$\{y,z\}$|
|$\{z,t.1\}$|
|$\{z,t.1\}$|
|$\{t.1\}$|
|$\{\}$|
|$\{\}$|
\end{lstlisting}
\end{minipage}

\caption{An example block annotated with live-after sets.}
\label{fig:live-eg}
\end{figure}

\begin{exercise}\normalfont
Implement the compiler pass named \code{uncover-live} that computes
the live-after sets. We recommend storing the live-after sets (a list
of lists of variables) in the $\itm{info}$ field of the \key{block}
construct.
%
We recommend organizing your code to use a helper function that takes
a list of instructions and an initial live-after set (typically empty)
and returns the list of live-after sets.
%
We recommend creating helper functions to 1) compute the set of
variables that appear in an argument (of an instruction), 2) compute
the variables read by an instruction which corresponds to the $R$
function discussed above, and 3) the variables written by an
instruction which corresponds to $W$.
\end{exercise}

\section{Building the Interference Graph}
\label{sec:build-interference}

Based on the liveness analysis, we know where each variable is needed.
However, during register allocation, we need to answer questions of
the specific form: are variables $u$ and $v$ live at the same time?
(And therefore cannot be assigned to the same register.)  To make this
question easier to answer, we create an explicit data structure, an
\emph{interference graph}.  An interference graph is an undirected
graph that has an edge between two variables if they are live at the
same time, that is, if they interfere with each other.

The most obvious way to compute the interference graph is to look at
the set of live variables between each statement in the program, and
add an edge to the graph for every pair of variables in the same set.
This approach is less than ideal for two reasons. First, it can be
rather expensive because it takes $O(n^2)$ time to look at every pair
in a set of $n$ live variables. Second, there is a special case in
which two variables that are live at the same time do not actually
interfere with each other: when they both contain the same value
because we have assigned one to the other.

A better way to compute the interference graph is to focus on the
writes. That is, for each instruction, create an edge between the
variable being written to and all the \emph{other} live variables.
(One should not create self edges.) For a \key{callq} instruction,
think of all caller-saved registers as being written to, so and edge
must be added between every live variable and every caller-saved
register. For \key{movq}, we deal with the above-mentioned special
case by not adding an edge between a live variable $v$ and destination
$d$ if $v$ matches the source of the move. So we have the following
three rules.

\begin{enumerate}
\item If instruction $I_k$ is an arithmetic instruction such as
  (\key{addq} $s$\, $d$), then add the edge $(d,v)$ for every $v \in
  L_{\mathsf{after}}(k)$ unless $v = d$.

\item If instruction $I_k$ is of the form (\key{callq}
  $\mathit{label}$), then add an edge $(r,v)$ for every caller-saved
  register $r$ and every variable $v \in L_{\mathsf{after}}(k)$.

\item If instruction $I_k$ is a move: (\key{movq} $s$\, $d$), then add
  the edge $(d,v)$ for every $v \in L_{\mathsf{after}}(k)$ unless $v =
  d$ or $v = s$.
\end{enumerate}
\margincomment{JM: I think you could give examples of each one of these
  using the example program and use those to help explain why these
  rules are correct.\\
  JS: Agreed.}

Working from the top to bottom of Figure~\ref{fig:live-eg}, we obtain
the following interference for the instruction at the specified line
number.
\begin{quote}
Line 2: no interference,\\
Line 3: $w$ interferes with $v$,\\
Line 4: $x$ interferes with $w$,\\
Line 5: $x$ interferes with $w$,\\
Line 6: $y$ interferes with $w$,\\
Line 7: $y$ interferes with $w$ and $x$,\\
Line 8: $z$ interferes with $w$ and $y$,\\
Line 9: $z$ interferes with $y$, \\
Line 10: $t.1$ interferes with $z$, \\
Line 11: $t.1$ interferes with $z$, \\
Line 12: no interference, \\
Line 13: no interference. \\
Line 14: no interference.
\end{quote}
The resulting interference graph is shown in
Figure~\ref{fig:interfere}.

\begin{figure}[tbp]
\large
\[
\begin{tikzpicture}[baseline=(current  bounding  box.center)]
\node (v) at (0,0)   {$v$};
\node (w) at (2,0)   {$w$};
\node (x) at (4,0)   {$x$};
\node (t1) at (6,-2)   {$t.1$};
\node (y) at (2,-2)  {$y$};
\node (z) at (4,-2)  {$z$};

\draw (v) to (w);
\foreach \i in {w,x,y}
{
  \foreach \j in {w,x,y}
  {
    \draw (\i) to (\j);
  }
}
\draw (z) to (w);
\draw (z) to (y);
\draw (t1) to (z);
\end{tikzpicture}
\]
\caption{The interference graph of the example program.}
\label{fig:interfere}
\end{figure}

%% Our next concern is to choose a data structure for representing the
%% interference graph. There are many choices for how to represent a
%% graph, for example, \emph{adjacency matrix}, \emph{adjacency list},
%% and \emph{edge set}~\citep{Cormen:2001uq}. The right way to choose a
%% data structure is to study the algorithm that uses the data structure,
%% determine what operations need to be performed, and then choose the
%% data structure that provide the most efficient implementations of
%% those operations. Often times the choice of data structure can have an
%% effect on the time complexity of the algorithm, as it does here. If
%% you skim the next section, you will see that the register allocation
%% algorithm needs to ask the graph for all of its vertices and, given a
%% vertex, it needs to known all of the adjacent vertices. Thus, the
%% correct choice of graph representation is that of an adjacency
%% list. There are helper functions in \code{utilities.rkt} for
%% representing graphs using the adjacency list representation:
%% \code{make-graph}, \code{add-edge}, and \code{adjacent}
%% (Appendix~\ref{appendix:utilities}).
%% %
%% \margincomment{\footnotesize To do: change to use the
%%     Racket graph library. \\ --Jeremy}
%% %
%% In particular, those functions use a hash table to map each vertex to
%% the set of adjacent vertices, and the sets are represented using
%% Racket's \key{set}, which is also a hash table.

\begin{exercise}\normalfont
Implement the compiler pass named \code{build-interference} according
to the algorithm suggested above. We recommend using the Racket
\code{graph} package to create and inspect the interference graph.
The output graph of this pass should be stored in the $\itm{info}$
field of the program, under the key \code{conflicts}.
\end{exercise}

\section{Graph Coloring via Sudoku}
\label{sec:graph-coloring}

We now come to the main event, mapping variables to registers (or to
stack locations in the event that we run out of registers).  We need
to make sure not to map two variables to the same register if the two
variables interfere with each other.  In terms of the interference
graph, this means that adjacent vertices must be mapped to different
registers.  If we think of registers as colors, the register
allocation problem becomes the widely-studied graph coloring
problem~\citep{Balakrishnan:1996ve,Rosen:2002bh}.

The reader may be more familiar with the graph coloring problem than he
or she realizes; the popular game of Sudoku is an instance of the
graph coloring problem. The following describes how to build a graph
out of an initial Sudoku board.
\begin{itemize}
\item There is one vertex in the graph for each Sudoku square.
\item There is an edge between two vertices if the corresponding squares
  are in the same row, in the same column, or if the squares are in
  the same $3\times 3$ region.
\item Choose nine colors to correspond to the numbers $1$ to $9$.
\item Based on the initial assignment of numbers to squares in the
  Sudoku board, assign the corresponding colors to the corresponding
  vertices in the graph.
\end{itemize}
If you can color the remaining vertices in the graph with the nine
colors, then you have also solved the corresponding game of Sudoku.
Figure~\ref{fig:sudoku-graph} shows an initial Sudoku game board and
the corresponding graph with colored vertices.  We map the Sudoku
number 1 to blue, 2 to yellow, and 3 to red.  We only show edges for a
sampling of the vertices (those that are colored) because showing
edges for all of the vertices would make the graph unreadable.

\begin{figure}[tbp]
\includegraphics[width=0.45\textwidth]{figs/sudoku}
\includegraphics[width=0.5\textwidth]{figs/sudoku-graph}
\caption{A Sudoku game board and the corresponding colored graph.}
\label{fig:sudoku-graph}
\end{figure}


Given that Sudoku is an instance of graph coloring, one can use Sudoku
strategies to come up with an algorithm for allocating registers. For
example, one of the basic techniques for Sudoku is called Pencil
Marks. The idea is that you use a process of elimination to determine
what numbers no longer make sense for a square, and write down those
numbers in the square (writing very small). For example, if the number
$1$ is assigned to a square, then by process of elimination, you can
write the pencil mark $1$ in all the squares in the same row, column,
and region. Many Sudoku computer games provide automatic support for
Pencil Marks. 
%
The Pencil Marks technique corresponds to the notion of color
\emph{saturation} due to \cite{Brelaz:1979eu}.  The saturation of a
vertex, in Sudoku terms, is the set of colors that are no longer
available. In graph terminology, we have the following definition:
\begin{equation*}
  \mathrm{saturation}(u) = \{ c \;|\; \exists v. v \in \mathrm{neighbors}(u)
     \text{ and } \mathrm{color}(v) = c \}
\end{equation*}
where $\mathrm{neighbors}(u)$ is the set of vertices that share an
edge with $u$.

Using the Pencil Marks technique leads to a simple strategy for
filling in numbers: if there is a square with only one possible number
left, then write down that number! But what if there are no squares
with only one possibility left? One brute-force approach is to just
make a guess. If that guess ultimately leads to a solution, great.  If
not, backtrack to the guess and make a different guess.  One good
thing about Pencil Marks is that it reduces the degree of branching in
the search tree. Nevertheless, backtracking can be horribly time
consuming. One way to reduce the amount of backtracking is to use the
most-constrained-first heuristic. That is, when making a guess, always
choose a square with the fewest possibilities left (the vertex with
the highest saturation).  The idea is that choosing highly constrained
squares earlier rather than later is better because later there may
not be any possibilities.

In some sense, register allocation is easier than Sudoku because we
can always cheat and add more numbers by mapping variables to the
stack. We say that a variable is \emph{spilled} when we decide to map
it to a stack location. We would like to minimize the time needed to
color the graph, and backtracking is expensive. Thus, it makes sense
to keep the most-constrained-first heuristic but drop the backtracking
in favor of greedy search (guess and just keep going).
Figure~\ref{fig:satur-algo} gives the pseudo-code for this simple
greedy algorithm for register allocation based on saturation and the
most-constrained-first heuristic, which is roughly equivalent to the
DSATUR algorithm of \cite{Brelaz:1979eu} (also known as saturation
degree ordering~\citep{Gebremedhin:1999fk,Omari:2006uq}).  Just
as in Sudoku, the algorithm represents colors with integers, with the
first $k$ colors corresponding to the $k$ registers in a given machine
and the rest of the integers corresponding to stack locations.

\begin{figure}[btp]
  \centering
\begin{lstlisting}[basicstyle=\rmfamily,deletekeywords={for,from,with,is,not,in,find},morekeywords={while},columns=fullflexible]
Algorithm: DSATUR
Input: a graph |$G$|
Output: an assignment |$\mathrm{color}[v]$| for each vertex |$v \in G$|

|$W \gets \mathit{vertices}(G)$|
while |$W \neq \emptyset$| do
    pick a vertex |$u$| from |$W$| with the highest saturation,
        breaking ties randomly
    find the lowest color |$c$| that is not in |$\{ \mathrm{color}[v] \;:\; v \in \mathrm{adjacent}(u)\}$|
    |$\mathrm{color}[u] \gets c$|
    |$W \gets W - \{u\}$|
\end{lstlisting}
  \caption{The saturation-based greedy graph coloring algorithm.}
  \label{fig:satur-algo}
\end{figure}

With this algorithm in hand, let us return to the running example and
consider how to color the interference graph in
Figure~\ref{fig:interfere}. We shall not use register \key{rax} for
register allocation because we use it to patch instructions, so we
remove that vertex from the graph.  Initially, all of the vertices are
not yet colored and they are unsaturated, so we annotate each of them
with a dash for their color and an empty set for the saturation.
\[
\begin{tikzpicture}[baseline=(current  bounding  box.center)]
\node (v) at (0,0)    {$v:-,\{\}$};
\node (w) at (3,0)    {$w:-,\{\}$};
\node (x) at (6,0)    {$x:-,\{\}$};
\node (y) at (3,-1.5) {$y:-,\{\}$};
\node (z) at (6,-1.5) {$z:-,\{\}$};
\node (t1) at (9,-1.5)   {$t.1:-,\{\}$};

\draw (v) to (w);
\foreach \i in {w,x,y}
{
  \foreach \j in {w,x,y}
  {
    \draw (\i) to (\j);
  }
}
\draw (z) to (w);
\draw (z) to (y);
\draw (t1) to (z);
\end{tikzpicture}
\]
We select a maximally saturated vertex and color it $0$. In this case we
have a 7-way tie, so we arbitrarily pick $t.1$. The then mark color $0$
as no longer available for $z$ because it interferes
with $t.1$.
\[
\begin{tikzpicture}[baseline=(current  bounding  box.center)]
\node (v) at (0,0)    {$v:-,\{\}$};
\node (w) at (3,0)    {$w:-,\{\}$};
\node (x) at (6,0)    {$x:-,\{\}$};
\node (y) at (3,-1.5) {$y:-,\{\}$};
\node (z) at (6,-1.5) {$z:-,\{\mathbf{0}\}$};
\node (t1) at (9,-1.5)   {$t.1:\mathbf{0},\{\}$};
\draw (v) to (w);
\foreach \i in {w,x,y}
{
  \foreach \j in {w,x,y}
  {
    \draw (\i) to (\j);
  }
}
\draw (z) to (w);
\draw (z) to (y);
\draw (t1) to (z);
\end{tikzpicture}
\]
Now we repeat the process, selecting another maximally saturated
vertex, which in this case is $z$. We color $z$ with $1$.
\[
\begin{tikzpicture}[baseline=(current  bounding  box.center)]
\node (v) at (0,0)    {$v:-,\{\}$};
\node (w) at (3,0)    {$w:-,\{\mathbf{1}\}$};
\node (x) at (6,0)    {$x:-,\{\}$};
\node (y) at (3,-1.5) {$y:-,\{\mathbf{1}\}$};
\node (z) at (6,-1.5) {$z:\mathbf{1},\{0\}$};
\node (t1) at (9,-1.5)   {$t.1:0,\{\mathbf{1}\}$};
\draw (t1) to (z);
\draw (v) to (w);
\foreach \i in {w,x,y}
{
  \foreach \j in {w,x,y}
  {
    \draw (\i) to (\j);
  }
}
\draw (z) to (w);
\draw (z) to (y);
\end{tikzpicture}
\]
The most saturated vertices are now $w$ and $y$. We color $y$ with the
first available color, which is $0$.
\[
\begin{tikzpicture}[baseline=(current  bounding  box.center)]
\node (v) at (0,0)    {$v:-,\{\}$};
\node (w) at (3,0)    {$w:-,\{\mathbf{0},1\}$};
\node (x) at (6,0)    {$x:-,\{\mathbf{0},\}$};
\node (y) at (3,-1.5) {$y:\mathbf{0},\{1\}$};
\node (z) at (6,-1.5) {$z:1,\{\mathbf{0}\}$};
\node (t1) at (9,-1.5)   {$t.1:0,\{1\}$};
\draw (t1) to (z);
\draw (v) to (w);
\foreach \i in {w,x,y}
{
  \foreach \j in {w,x,y}
  {
    \draw (\i) to (\j);
  }
}
\draw (z) to (w);
\draw (z) to (y);
\end{tikzpicture}
\]
Vertex $w$ is now the most highly saturated, so we color $w$ with $2$.
\[
\begin{tikzpicture}[baseline=(current  bounding  box.center)]
\node (v) at (0,0)   {$v:-,\{2\}$};
\node (w) at (3,0)   {$w:\mathbf{2},\{0,1\}$};
\node (x) at (6,0)   {$x:-,\{0,\mathbf{2}\}$};
\node (y) at (3,-1.5)  {$y:0,\{1,\mathbf{2}\}$};
\node (z) at (6,-1.5)  {$z:1,\{0,\mathbf{2}\}$};
\node (t1) at (9,-1.5)   {$t.1:0,\{\}$};
\draw (t1) to (z);
\draw (v) to (w);
\foreach \i in {w,x,y}
{
  \foreach \j in {w,x,y}
  {
    \draw (\i) to (\j);
  }
}
\draw (z) to (w);
\draw (z) to (y);
\end{tikzpicture}
\]
Now $x$ has the highest saturation, so we color it $1$.
\[
\begin{tikzpicture}[baseline=(current  bounding  box.center)]
\node (v) at (0,0)   {$v:-,\{2\}$};
\node (w) at (3,0)   {$w:2,\{0,\mathbf{1}\}$};
\node (x) at (6,0)   {$x:\mathbf{1},\{0,2\}$};
\node (y) at (3,-1.5)  {$y:0,\{\mathbf{1},2\}$};
\node (z) at (6,-1.5)  {$z:1,\{0,2\}$};
\node (t1) at (9,-1.5)   {$t.1:0,\{\}$};
\draw (t1) to (z);
\draw (v) to (w);
\foreach \i in {w,x,y}
{
  \foreach \j in {w,x,y}
  {
    \draw (\i) to (\j);
  }
}
\draw (z) to (w);
\draw (z) to (y);
\end{tikzpicture}
\]
In the last step of the algorithm, we color $v$ with $0$.
\[
\begin{tikzpicture}[baseline=(current  bounding  box.center)]
\node (v) at (0,0)   {$v:\mathbf{0},\{2\}$};
\node (w) at (3,0)   {$w:2,\{\mathbf{0},1\}$};
\node (x) at (6,0)   {$x:1,\{0,2\}$};
\node (y) at (3,-1.5)  {$y:0,\{1,2\}$};
\node (z) at (6,-1.5)  {$z:1,\{0,2\}$};
\node (t1) at (9,-1.5)   {$t.1:0,\{\}$};
\draw (t1) to (z);
\draw (v) to (w);
\foreach \i in {w,x,y}
{
  \foreach \j in {w,x,y}
  {
    \draw (\i) to (\j);
  }
}
\draw (z) to (w);
\draw (z) to (y);
\end{tikzpicture}
\]

With the coloring complete, we can finalize the assignment of
variables to registers and stack locations. Recall that if we have $k$
registers, we map the first $k$ colors to registers and the rest to
stack locations.  Suppose for the moment that we have just one
register to use for register allocation, \key{rcx}. Then the following
is the mapping of colors to registers and stack allocations.
\[
  \{ 0 \mapsto \key{\%rcx}, \; 1 \mapsto \key{-8(\%rbp)}, \; 2 \mapsto \key{-16(\%rbp)}, \ldots \}
\]
Putting this mapping together with the above coloring of the variables, we
arrive at the assignment:
\begin{gather*}
  \{ v \mapsto \key{\%rcx}, \,
  w \mapsto \key{-16(\%rbp)},  \,
  x \mapsto \key{-8(\%rbp)}, \\
  y \mapsto \key{\%rcx},  \,
  z\mapsto \key{-8(\%rbp)}, 
  t.1\mapsto \key{\%rcx} \}
\end{gather*}
Applying this assignment to our running example, on the left, yields
the program on the right.\\
% why frame size of 32? -JGS
\begin{minipage}{0.4\textwidth}
\begin{lstlisting}
(block ()
  (movq (int 1) (var v))
  (movq (int 46) (var w))
  (movq (var v) (var x))
  (addq (int 7) (var x))
  (movq (var x) (var y))
  (addq (int 4) (var y))
  (movq (var x) (var z))
  (addq (var w) (var z))
  (movq (var y) (var t.1))
  (negq (var t.1))
  (movq (var z) (reg rax))
  (addq (var t.1) (reg rax))
  (jmp conclusion))
\end{lstlisting}
\end{minipage}
$\Rightarrow$
\begin{minipage}{0.45\textwidth}
\begin{lstlisting}
(block ()
  (movq (int 1) (reg rcx))
  (movq (int 46) (deref rbp -16))
  (movq (reg rcx) (deref rbp -8))
  (addq (int 7) (deref rbp -8))
  (movq (deref rbp -8) (reg rcx))
  (addq (int 4) (reg rcx))
  (movq (deref rbp -8) (deref rbp -8))
  (addq (deref rbp -16) (deref rbp -8))
  (movq (reg rcx) (reg rcx))
  (negq (reg rcx))
  (movq (deref rbp -8) (reg rax))
  (addq (reg rcx) (reg rax))
  (jmp conclusion))
\end{lstlisting}
\end{minipage}

The resulting program is almost an x86 program. The remaining step
is to apply the patch instructions pass. In this example, the trivial
move of \code{-8(\%rbp)} to itself is deleted and the addition of
\code{-16(\%rbp)} to \key{-8(\%rbp)} is fixed by going through
\code{rax} as follows.
\begin{lstlisting}
  (movq (deref rbp -16) (reg rax)
  (addq (reg rax) (deref rbp -8))
\end{lstlisting}

An overview of all of the passes involved in register allocation is
shown in Figure~\ref{fig:reg-alloc-passes}.

\begin{figure}[tbp]
\begin{tikzpicture}[baseline=(current  bounding  box.center)]
\node (R1) at (0,2)  {\large $R_1$};
\node (R1-2) at (3,2)  {\large $R_1$};
\node (R1-3) at (6,2)  {\large $R_1$};
\node (C0-1) at (6,0)  {\large $C_0$};
\node (C0-2) at (3,0)  {\large $C_0$};

\node (x86-2) at (3,-2)  {\large $\text{x86}^{*}$};
\node (x86-3) at (6,-2)  {\large $\text{x86}^{*}$};
\node (x86-4) at (9,-2) {\large $\text{x86}$};
\node (x86-5) at (12,-2) {\large $\text{x86}^{\dagger}$};

\node (x86-2-1) at (3,-4)  {\large $\text{x86}^{*}$};
\node (x86-2-2) at (6,-4)  {\large $\text{x86}^{*}$};

\path[->,bend left=15] (R1) edge [above] node {\ttfamily\footnotesize uniquify} (R1-2);
\path[->,bend left=15] (R1-2) edge [above] node {\ttfamily\footnotesize remove-complex.} (R1-3);
\path[->,bend left=15] (R1-3) edge [right] node {\ttfamily\footnotesize explicate-control} (C0-1);
\path[->,bend right=15] (C0-1) edge [above] node {\ttfamily\footnotesize uncover-locals} (C0-2);
\path[->,bend right=15] (C0-2) edge [left] node {\ttfamily\footnotesize select-instr.} (x86-2);
\path[->,bend left=15] (x86-2) edge [right] node {\ttfamily\footnotesize\color{red} uncover-live} (x86-2-1);
\path[->,bend right=15] (x86-2-1) edge [below] node {\ttfamily\footnotesize\color{red} build-inter.} (x86-2-2);
\path[->,bend right=15] (x86-2-2) edge [right] node {\ttfamily\footnotesize\color{red} allocate-reg.} (x86-3);
\path[->,bend left=15] (x86-3) edge [above] node {\ttfamily\footnotesize patch-instr.} (x86-4);
\path[->,bend left=15] (x86-4) edge [above] node {\ttfamily\footnotesize print-x86} (x86-5);
\end{tikzpicture}
\caption{Diagram of the passes for $R_1$ with register allocation.}
\label{fig:reg-alloc-passes}
\end{figure}

\begin{exercise}\normalfont
  Implement the pass \code{allocate-registers}, which should come
  after the \code{build-interference} pass. The three new passes,
  \code{uncover-live}, \code{build-interference}, and
  \code{allocate-registers} replace the \code{assign-homes} pass of
  Section~\ref{sec:assign-r1}.
  
  We recommend that you create a helper function named
  \code{color-graph} that takes an interference graph and a list of
  all the variables in the program. This function should return a
  mapping of variables to their colors (represented as natural
  numbers). By creating this helper function, you will be able to
  reuse it in Chapter~\ref{ch:functions} when you add support for
  functions.

  Once you have obtained the coloring from \code{color-graph}, you can
  assign the variables to registers or stack locations and then reuse
  code from the \code{assign-homes} pass from
  Section~\ref{sec:assign-r1} to replace the variables with their
  assigned location.
  
  Test your updated compiler by creating new example programs that
  exercise all of the register allocation algorithm, such as forcing
  variables to be spilled to the stack.
\end{exercise}


\section{Print x86 and Conventions for Registers}
\label{sec:print-x86-reg-alloc}

Recall the \code{print-x86} pass generates the prelude and
conclusion instructions for the \code{main} function.
%
The prelude saved the values in \code{rbp} and \code{rsp} and the
conclusion returned those values to \code{rbp} and \code{rsp}.  The
reason for this is that our \code{main} function must adhere to the
x86 calling conventions that we described in
Section~\ref{sec:calling-conventions}. In addition, the \code{main}
function needs to restore (in the conclusion) any callee-saved
registers that get used during register allocation. The simplest
approach is to save and restore all of the callee-saved registers. The
more efficient approach is to keep track of which callee-saved
registers were used and only save and restore them. Either way, make
sure to take this use of stack space into account when you are
calculating the size of the frame. Also, don't forget that the size of
the frame needs to be a multiple of 16 bytes.

\section{Challenge: Move Biasing$^{*}$}
\label{sec:move-biasing}

This section describes an optional enhancement to register allocation
for those students who are looking for an extra challenge or who have
a deeper interest in register allocation.

We return to the running example, but we remove the supposition that
we only have one register to use. So we have the following mapping of
color numbers to registers.
\[
  \{ 0 \mapsto \key{\%rbx}, \; 1 \mapsto \key{\%rcx}, \; 2 \mapsto \key{\%rdx}, \ldots \}
\]
Using the same assignment that was produced by register allocator
described in the last section, we get the following program.

\begin{minipage}{0.45\textwidth}
\begin{lstlisting}
(block ()
  (movq (int 1) (var v))
  (movq (int 46) (var w))
  (movq (var v) (var x))
  (addq (int 7) (var x))
  (movq (var x) (var y))
  (addq (int 4) (var y))
  (movq (var x) (var z))
  (addq (var w) (var z))
  (movq (var y) (var t.1))
  (negq (var t.1))
  (movq (var z) (reg rax))
  (addq (var t.1) (reg rax))
  (jmp conclusion))
\end{lstlisting}
\end{minipage}
$\Rightarrow$
\begin{minipage}{0.45\textwidth}
\begin{lstlisting}
(block ()
  (movq (int 1) (reg rbx))
  (movq (int 46) (reg rdx))
  (movq (reg rbx) (reg rcx))
  (addq (int 7) (reg rcx))
  (movq (reg rcx) (reg rbx))
  (addq (int 4) (reg rbx))
  (movq (reg rcx) (reg rcx))
  (addq (reg rdx) (reg rcx))
  (movq (reg rbx) (reg rbx))
  (negq (reg rbx))
  (movq (reg rcx) (reg rax))
  (addq (reg rbx) (reg rax))
  (jmp conclusion))
\end{lstlisting}
\end{minipage}

While this allocation is quite good, we could do better. For example,
the variables \key{v} and \key{x} ended up in different registers, but
if they had been placed in the same register, then the move from
\key{v} to \key{x} could be removed.

We say that two variables $p$ and $q$ are \emph{move related} if they
participate together in a \key{movq} instruction, that is, \key{movq}
  $p$, $q$ or \key{movq} $q$, $p$. When the register allocator chooses a
color for a variable, it should prefer a color that has already been
used for a move-related variable (assuming that they do not
interfere). Of course, this preference should not override the
preference for registers over stack locations, but should only be used
as a tie breaker when choosing between registers or when choosing
between stack locations.

We recommend that you represent the move relationships in a graph,
similar to how we represented interference.  The following is the
\emph{move graph} for our running example.
\[
\begin{tikzpicture}[baseline=(current  bounding  box.center)]
\node (v) at (0,0)    {$v$};
\node (w) at (3,0)    {$w$};
\node (x) at (6,0)    {$x$};
\node (y) at (3,-1.5) {$y$};
\node (z) at (6,-1.5) {$z$};
\node (t1) at (9,-1.5)   {$t.1$};
\draw[bend left=15] (t1) to (y);
\draw[bend left=15] (v) to (x);
\draw (x) to (y);
\draw (x) to (z);
\end{tikzpicture}
\]

Now we replay the graph coloring, pausing to see the coloring of $x$
and $v$. So we have the following coloring and the most saturated
vertex is $x$.
\[
\begin{tikzpicture}[baseline=(current  bounding  box.center)]
\node (v) at (0,0)   {$v:-,\{2\}$};
\node (w) at (3,0)   {$w:2,\{0,1\}$};
\node (x) at (6,0)   {$x:-,\{0,2\}$};
\node (y) at (3,-1.5)  {$y:0,\{1,2\}$};
\node (z) at (6,-1.5)  {$z:1,\{0,2\}$};
\node (t1) at (9,-1.5)   {$t.1:0,\{\}$};
\draw (t1) to (z);
\draw (v) to (w);
\foreach \i in {w,x,y}
{
  \foreach \j in {w,x,y}
  {
    \draw (\i) to (\j);
  }
}
\draw (z) to (w);
\draw (z) to (y);
\end{tikzpicture}
\]
Last time we chose to color $x$ with $1$,
%
which so happens to be the color of $z$, and $x$ is move related to
$z$. This was rather lucky, and if the program had been a little
different, and say $z$ had been already assigned to $2$, then $x$
would still get $1$ and our luck would have run out. With move
biasing, we use the fact that $x$ and $z$ are move related to
influence the choice of color for $x$, in this case choosing $1$
because that's the color of $z$.
\[
\begin{tikzpicture}[baseline=(current  bounding  box.center)]
\node (v) at (0,0)   {$v:-,\{2\}$};
\node (w) at (3,0)   {$w:2,\{0,\mathbf{1}\}$};
\node (x) at (6,0)   {$x:\mathbf{1},\{0,2\}$};
\node (y) at (3,-1.5)  {$y:0,\{\mathbf{1},2\}$};
\node (z) at (6,-1.5)  {$z:1,\{0,2\}$};
\node (t1) at (9,-1.5)   {$t.1:0,\{\}$};
\draw (t1) to (z);
\draw (v) to (w);
\foreach \i in {w,x,y}
{
  \foreach \j in {w,x,y}
  {
    \draw (\i) to (\j);
  }
}
\draw (z) to (w);
\draw (z) to (y);
\end{tikzpicture}
\]

Next we consider coloring the variable $v$, and we just need to avoid
choosing $2$ because of the interference with $w$. Last time we choose
the color $0$, simply because it was the lowest, but this time we know
that $v$ is move related to $x$, so we choose the color $1$.
\[
\begin{tikzpicture}[baseline=(current  bounding  box.center)]
\node (v) at (0,0)   {$v:\mathbf{1},\{2\}$};
\node (w) at (3,0)   {$w:2,\{0,\mathbf{1}\}$};
\node (x) at (6,0)   {$x:1,\{0,2\}$};
\node (y) at (3,-1.5)  {$y:0,\{1,2\}$};
\node (z) at (6,-1.5)  {$z:1,\{0,2\}$};
\node (t1) at (9,-1.5)   {$t.1:0,\{\}$};
\draw (t1) to (z);
\draw (v) to (w);
\foreach \i in {w,x,y}
{
  \foreach \j in {w,x,y}
  {
    \draw (\i) to (\j);
  }
}
\draw (z) to (w);
\draw (z) to (y);
\end{tikzpicture}
\]

We apply this register assignment to the running example, on the left,
to obtain the code on right.

\begin{minipage}{0.45\textwidth}
\begin{lstlisting}
(block ()
  (movq (int 1) (var v))
  (movq (int 46) (var w))
  (movq (var v) (var x))
  (addq (int 7) (var x))
  (movq (var x) (var y))
  (addq (int 4) (var y))
  (movq (var x) (var z))
  (addq (var w) (var z))
  (movq (var y) (var t.1))
  (negq (var t.1))
  (movq (var z) (reg rax))
  (addq (var t.1) (reg rax))
  (jmp conclusion))
\end{lstlisting}
\end{minipage}
$\Rightarrow$
\begin{minipage}{0.45\textwidth}
\begin{lstlisting}
(block ()
  (movq (int 1) (reg rcx))
  (movq (int 46) (reg rbx))
  (movq (reg rcx) (reg rcx))
  (addq (int 7) (reg rcx))
  (movq (reg rcx) (reg rdx))
  (addq (int 4) (reg rdx))
  (movq (reg rcx) (reg rcx))
  (addq (reg rbx) (reg rcx))
  (movq (reg rdx) (reg rbx))
  (negq (reg rbx))
  (movq (reg rcx) (reg rax))
  (addq (reg rbx) (reg rax))
  (jmp conclusion))
\end{lstlisting}
\end{minipage}

The \code{patch-instructions} then removes the trivial moves from
\key{v} to \key{x} and from \key{x} to \key{z} to obtain the following
result.

\begin{minipage}{0.45\textwidth}
  \begin{lstlisting}
(block ()
  (movq (int 1) (reg rcx))
  (movq (int 46) (reg rbx))
  (addq (int 7) (reg rcx))
  (movq (reg rcx) (reg rdx))
  (addq (int 4) (reg rdx))
  (addq (reg rbx) (reg rcx))
  (movq (reg rdx) (reg rbx))
  (negq (reg rbx))
  (movq (reg rcx) (reg rax))
  (addq (reg rbx) (reg rax))
  (jmp conclusion))
\end{lstlisting}
\end{minipage}

\begin{exercise}\normalfont
Change your implementation of \code{allocate-registers} to take move
biasing into account. Make sure that your compiler still passes all of
the previous tests. Create two new tests that include at least one
opportunity for move biasing and visually inspect the output x86
programs to make sure that your move biasing is working properly.
\end{exercise}

\margincomment{\footnotesize To do: another neat challenge would be to do
  live range splitting~\citep{Cooper:1998ly}. \\ --Jeremy}


%%%%%%%%%%%%%%%%%%%%%%%%%%%%%%%%%%%%%%%%%%%%%%%%%%%%%%%%%%%%%%%%%%%%%%%%%%%%%%%%
\chapter{Booleans and Control Flow}
\label{ch:bool-types}

The $R_0$ and $R_1$ languages only had a single kind of value, the
integers. In this Chapter we add a second kind of value, the Booleans,
to create the $R_2$ language. The Boolean values \emph{true} and
\emph{false} are written \key{\#t} and \key{\#f} respectively in
Racket.  We also introduce several operations that involve Booleans
(\key{and}, \key{not}, \key{eq?}, \key{<}, etc.) and the conditional
\key{if} expression. With the addition of \key{if} expressions,
programs can have non-trivial control flow which has an impact on
several parts of the compiler. Also, because we now have two kinds of
values, we need to worry about programs that apply an operation to the
wrong kind of value, such as \code{(not 1)}.

There are two language design options for such situations.  One option
is to signal an error and the other is to provide a wider
interpretation of the operation. The Racket language uses a mixture of
these two options, depending on the operation and the kind of
value. For example, the result of \code{(not 1)} in Racket is
\code{\#f} because Racket treats non-zero integers like \code{\#t}. On
the other hand, \code{(car 1)} results in a run-time error in Racket
stating that \code{car} expects a pair.

The Typed Racket language makes similar design choices as Racket,
except much of the error detection happens at compile time instead of
run time. Like Racket, Typed Racket accepts and runs \code{(not 1)},
producing \code{\#f}. But in the case of \code{(car 1)}, Typed Racket
reports a compile-time error because Typed Racket expects the type of
the argument to be of the form \code{(Listof T)} or \code{(Pairof T1 T2)}.

For the $R_2$ language we choose to be more like Typed Racket in that
we shall perform type checking during compilation. In
Chapter~\ref{ch:type-dynamic} we study the alternative choice, that
is, how to compile a dynamically typed language like Racket.  The
$R_2$ language is a subset of Typed Racket but by no means includes
all of Typed Racket. Furthermore, for many of the operations we shall
take a narrower interpretation than Typed Racket, for example,
rejecting \code{(not 1)}.

This chapter is organized as follows.  We begin by defining the syntax
and interpreter for the $R_2$ language (Section~\ref{sec:r2-lang}). We
then introduce the idea of type checking and build a type checker for
$R_2$ (Section~\ref{sec:type-check-r2}). To compile $R_2$ we need to
enlarge the intermediate language $C_0$ into $C_1$, which we do in
Section~\ref{sec:c1}. The remaining sections of this Chapter discuss
how our compiler passes need to change to accommodate Booleans and
conditional control flow.


\section{The $R_2$ Language}
\label{sec:r2-lang}

The syntax of the $R_2$ language is defined in
Figure~\ref{fig:r2-syntax}. It includes all of $R_1$ (shown in gray),
the Boolean literals \code{\#t} and \code{\#f}, and the conditional
\code{if} expression. Also, we expand the operators to include
subtraction, \key{and}, \key{or} and \key{not}, the \key{eq?}
operations for comparing two integers or two Booleans, and the
\key{<}, \key{<=}, \key{>}, and \key{>=} operations for comparing
integers.

\begin{figure}[tp]
\centering
\fbox{
\begin{minipage}{0.96\textwidth}
\[
\begin{array}{lcl}
  \itm{cmp} &::= & \key{eq?} \mid \key{<} \mid \key{<=} \mid \key{>} \mid \key{>=} \\
  \Exp &::=& \gray{\Int \mid (\key{read}) \mid (\key{-}\;\Exp) \mid (\key{+} \; \Exp\;\Exp)}  \mid (\key{-}\;\Exp\;\Exp) \\
     &\mid&  \gray{\Var \mid \LET{\Var}{\Exp}{\Exp}} \\
     &\mid& \key{\#t} \mid \key{\#f} 
      \mid (\key{and}\;\Exp\;\Exp) \mid (\key{or}\;\Exp\;\Exp)
      \mid (\key{not}\;\Exp) \\
      &\mid& (\itm{cmp}\;\Exp\;\Exp) \mid \IF{\Exp}{\Exp}{\Exp} \\
  R_2 &::=& (\key{program} \; \itm{info}\; \Exp)
\end{array}
\]
\end{minipage}
}
\caption{The syntax of $R_2$, extending $R_1$
  (Figure~\ref{fig:r1-syntax}) with Booleans and conditionals.}
\label{fig:r2-syntax}
\end{figure}

Figure~\ref{fig:interp-R2} defines the interpreter for $R_2$, omitting
the parts that are the same as the interpreter for $R_1$
(Figure~\ref{fig:interp-R1}). The literals \code{\#t} and \code{\#f}
simply evaluate to themselves. The conditional expression $(\key{if}\,
\itm{cnd}\,\itm{thn}\,\itm{els})$ evaluates the Boolean expression
\itm{cnd} and then either evaluates \itm{thn} or \itm{els} depending
on whether \itm{cnd} produced \code{\#t} or \code{\#f}. The logical
operations \code{not} and \code{and} behave as you might expect, but
note that the \code{and} operation is short-circuiting. That is, given
the expression $(\key{and}\,e_1\,e_2)$, the expression $e_2$ is not
evaluated if $e_1$ evaluates to \code{\#f}.

With the addition of the comparison operations, there are quite a few
primitive operations and the interpreter code for them is somewhat
repetitive. In Figure~\ref{fig:interp-R2} we factor out the different
parts into the \code{interp-op} function and the similar parts into
the one match clause shown in Figure~\ref{fig:interp-R2}. We do not
use \code{interp-op} for the \code{and} operation because of the
short-circuiting behavior in the order of evaluation of its arguments.


\begin{figure}[tbp]
\begin{lstlisting}
   (define primitives (set '+ '- 'eq? '< '<= '> '>= 'not 'read))

   (define (interp-op op)
     (match op
       ...
       ['not (lambda (v) (match v [#t #f] [#f #t]))]
       ['eq? (lambda (v1 v2)
               (cond [(or (and (fixnum? v1) (fixnum? v2))
                          (and (boolean? v1) (boolean? v2)))
                      (eq? v1 v2)]))]
       ['< (lambda (v1 v2)
             (cond [(and (fixnum? v1) (fixnum? v2)) (< v1 v2)]))]
       ['<= (lambda (v1 v2)
              (cond [(and (fixnum? v1) (fixnum? v2)) (<= v1 v2)]))]
       ['> (lambda (v1 v2)
             (cond [(and (fixnum? v1) (fixnum? v2)) (> v1 v2)]))]
       ['>= (lambda (v1 v2)
              (cond [(and (fixnum? v1) (fixnum? v2)) (>= v1 v2)]))]
       [else (error 'interp-op "unknown operator")]))

   (define (interp-exp env)
     (lambda (e)
       (define recur (interp-exp env))
       (match e
         ...
         [(? boolean?) e]
         [`(if ,cnd ,thn ,els)
          (define b (recur cnd))
          (match b
            [#t (recur thn)]
            [#f (recur els)])]
         [`(and ,e1 ,e2)
          (define v1 (recur e1))
          (match v1
            [#t (match (recur e2) [#t #t] [#f #f])]
            [#f #f])]
         [`(,op ,args ...)
          #:when (set-member? primitives op)
          (apply (interp-op op) (for/list ([e args]) (recur e)))]
         )))

   (define (interp-R2 env)
     (lambda (p)
       (match p
        [`(program ,info ,e)
         ((interp-exp '()) e)])))
\end{lstlisting}
\caption{Interpreter for the $R_2$ language.}
\label{fig:interp-R2}
\end{figure}


\section{Type Checking $R_2$ Programs}
\label{sec:type-check-r2}

It is helpful to think about type checking in two complementary
ways. A type checker predicts the \emph{type} of value that will be
produced by each expression in the program.  For $R_2$, we have just
two types, \key{Integer} and \key{Boolean}. So a type checker should
predict that
\begin{lstlisting}
   (+ 10 (- (+ 12 20)))
\end{lstlisting}
produces an \key{Integer} while
\begin{lstlisting}
   (and (not #f) #t)
\end{lstlisting}
produces a \key{Boolean}.

As mentioned at the beginning of this chapter, a type checker also
rejects programs that apply operators to the wrong type of value. Our
type checker for $R_2$ will signal an error for the following
expression because, as we have seen above, the expression \code{(+ 10
  ...)} has type \key{Integer}, and we require the argument of a
\code{not} to have type \key{Boolean}.
\begin{lstlisting}
   (not (+ 10 (- (+ 12 20))))
\end{lstlisting}

The type checker for $R_2$ is best implemented as a structurally
recursive function over the AST. Figure~\ref{fig:type-check-R2} shows
many of the clauses for the \code{type-check-exp} function.  Given an
input expression \code{e}, the type checker either returns the type
(\key{Integer} or \key{Boolean}) or it signals an error.  Of course,
the type of an integer literal is \code{Integer} and the type of a
Boolean literal is \code{Boolean}.  To handle variables, the type
checker, like the interpreter, uses an association list. However, in
this case the association list maps variables to types instead of
values. Consider the clause for \key{let}.  We type check the
initializing expression to obtain its type \key{T} and then associate
type \code{T} with the variable \code{x}. When the type checker
encounters the use of a variable, it can find its type in the
association list.

\begin{figure}[tbp]
\begin{lstlisting}
   (define (type-check-exp env)
     (lambda (e)
       (define recur (type-check-exp env))
       (match e
         [(? fixnum?)  'Integer]
         [(? boolean?) 'Boolean]
         [(? symbol? x) (dict-ref env x)]
         [`(read)      'Integer]
         [`(let ([,x ,e]) ,body)
          (define T (recur e))
          (define new-env (cons (cons x T) env))
          (type-check-exp new-env body)]
         ...
         [`(not ,e)
          (match (recur e)
            ['Boolean 'Boolean]
            [else (error 'type-check-exp "'not' expects a Boolean" e)])]
         ...
         )))

   (define (type-check-R2 env)
     (lambda (e)
       (match e
         [`(program ,info ,body)
          (define ty ((type-check-exp '()) body))
          `(program ,info ,body)]
         )))
\end{lstlisting}
\caption{Skeleton of a type checker for the $R_2$ language.}
\label{fig:type-check-R2}
\end{figure}

%% To print the resulting value correctly, the overall type of the
%% program must be threaded through the remainder of the passes. We can
%% store the type within the \key{program} form as shown in Figure
%% \ref{fig:type-check-R2}. Let $R^\dagger_2$ be the name for the
%% intermediate language produced by the type checker, which we define as
%% follows: \\[1ex]
%% \fbox{
%% \begin{minipage}{0.87\textwidth}
%% \[
%% \begin{array}{lcl}
%%   R^\dagger_2 &::=& (\key{program}\;(\key{type}\;\itm{type})\; \Exp)
%% \end{array}
%% \]
%% \end{minipage}
%% }

\begin{exercise}\normalfont
Complete the implementation of \code{type-check-R2} and test it on 10
new example programs in $R_2$ that you choose based on how thoroughly
they test the type checking algorithm. Half of the example programs
should have a type error, to make sure that your type checker properly
rejects them. The other half of the example programs should not have
type errors. Your testing should check that the result of the type
checker agrees with the value returned by the interpreter, that is, if
the type checker returns \key{Integer}, then the interpreter should
return an integer. Likewise, if the type checker returns
\key{Boolean}, then the interpreter should return \code{\#t} or
\code{\#f}. Note that if your type checker does not signal an error
for a program, then interpreting that program should not encounter an
error.  If it does, there is something wrong with your type checker.
\end{exercise}

\section{Shrink the $R_2$ Language}
\label{sec:shrink-r2}

The $R_2$ language includes several operators that are easily
expressible in terms of other operators. For example, subtraction is
expressible in terms of addition and negation.
\[
 (\key{-}\; e_1 \; e_2) \quad \Rightarrow \quad (\key{+} \; e_1 \; (\key{-} \; e_2))
\]
Several of the comparison operations are expressible in terms of
less-than and logical negation.
\[
(\key{<=}\; e_1 \; e_2) \quad \Rightarrow \quad
\LET{t_1}{e_1}{(\key{not}\;(\key{<}\;e_2\;t_1))}
\]
By performing these translations near the front-end of the compiler,
the later passes of the compiler will not need to deal with these
constructs, making those passes shorter. On the other hand, sometimes
these translations make it more difficult to generate the most
efficient code with respect to the number of instructions. However,
these differences typically do not affect the number of accesses to
memory, which is the primary factor that determines execution time on
modern computer architectures.

\begin{exercise}\normalfont
  Implement the pass \code{shrink} that removes subtraction,
  \key{and}, \key{or}, \key{<=}, \key{>}, and \key{>=} from the language
  by translating them to other constructs in $R_2$.  Create tests to
  make sure that the behavior of all of these constructs stays the
  same after translation.
\end{exercise}


\section{XOR, Comparisons, and Control Flow in x86}
\label{sec:x86-1}

To implement the new logical operations, the comparison operations,
and the \key{if} expression, we need to delve further into the x86
language. Figure~\ref{fig:x86-1} defines the abstract syntax for a
larger subset of x86 that includes instructions for logical
operations, comparisons, and jumps.

One small challenge is that x86 does not provide an instruction that
directly implements logical negation (\code{not} in $R_2$ and $C_1$).
However, the \code{xorq} instruction can be used to encode \code{not}.
The \key{xorq} instruction takes two arguments, performs a pairwise
exclusive-or operation on each bit of its arguments, and writes the
results into its second argument.  Recall the truth table for
exclusive-or:
\begin{center}
\begin{tabular}{l|cc}
   & 0 & 1 \\ \hline
0  & 0 & 1 \\
1  & 1 & 0
\end{tabular}
\end{center}
For example, $0011 \mathrel{\mathrm{XOR}} 0101 = 0110$.  Notice that
in row of the table for the bit $1$, the result is the opposite of the
second bit.  Thus, the \code{not} operation can be implemented by
\code{xorq} with $1$ as the first argument: $0001
\mathrel{\mathrm{XOR}} 0000 = 0001$ and $0001 \mathrel{\mathrm{XOR}}
0001 = 0000$.

\begin{figure}[tp]
\fbox{
\begin{minipage}{0.96\textwidth}
\[
\begin{array}{lcl}
\Arg &::=&  \gray{\INT{\Int} \mid \REG{\itm{register}}
    \mid (\key{deref}\,\itm{register}\,\Int)} \\
     &\mid& (\key{byte-reg}\; \itm{register}) \\
\itm{cc} & ::= & \key{e} \mid \key{l} \mid \key{le} \mid \key{g} \mid \key{ge} \\
\Instr &::=& \gray{(\key{addq} \; \Arg\; \Arg) \mid
             (\key{subq} \; \Arg\; \Arg) \mid
             (\key{negq} \; \Arg) \mid (\key{movq} \; \Arg\; \Arg)} \\
      &\mid& \gray{(\key{callq} \; \mathit{label}) \mid
             (\key{pushq}\;\Arg) \mid
             (\key{popq}\;\Arg) \mid
             (\key{retq})} \\
       &\mid& (\key{xorq} \; \Arg\;\Arg)
       \mid (\key{cmpq} \; \Arg\; \Arg) \mid (\key{set}\;\itm{cc} \; \Arg) \\
       &\mid& (\key{movzbq}\;\Arg\;\Arg)
       \mid  (\key{jmp} \; \itm{label})
       \mid (\key{jmp-if}\; \itm{cc} \; \itm{label}) \\
       &\mid& (\key{label} \; \itm{label}) \\
x86_1 &::= & (\key{program} \;\itm{info} \;(\key{type}\;\itm{type})\; \Instr^{+})
\end{array}
\]
\end{minipage}
}
\caption{The x86$_1$ language (extends x86$_0$ of Figure~\ref{fig:x86-ast-a}).}
\label{fig:x86-1}
\end{figure}

Next we consider the x86 instructions that are relevant for
compiling the comparison operations. The \key{cmpq} instruction
compares its two arguments to determine whether one argument is less
than, equal, or greater than the other argument. The \key{cmpq}
instruction is unusual regarding the order of its arguments and where
the result is placed. The argument order is backwards: if you want to
test whether $x < y$, then write \code{cmpq y, x}. The result of
\key{cmpq} is placed in the special EFLAGS register. This register
cannot be accessed directly but it can be queried by a number of
instructions, including the \key{set} instruction. The \key{set}
instruction puts a \key{1} or \key{0} into its destination depending
on whether the comparison came out according to the condition code
\itm{cc} (\key{e} for equal, \key{l} for less, \key{le} for
less-or-equal, \key{g} for greater, \key{ge} for greater-or-equal).
The set instruction has an annoying quirk in that its destination
argument must be single byte register, such as \code{al}, which is
part of the \code{rax} register.  Thankfully, the \key{movzbq}
instruction can then be used to move from a single byte register to a
normal 64-bit register.

For compiling the \key{if} expression, the x86 instructions for
jumping are relevant. The \key{jmp} instruction updates the program
counter to point to the instruction after the indicated label.  The
\key{jmp-if} instruction updates the program counter to point to the
instruction after the indicated label depending on whether the result
in the EFLAGS register matches the condition code \itm{cc}, otherwise
the \key{jmp-if} instruction falls through to the next
instruction. Because the \key{jmp-if} instruction relies on the EFLAGS
register, it is quite common for the \key{jmp-if} to be immediately
preceeded by a \key{cmpq} instruction, to set the EFLAGS regsiter.
Our abstract syntax for \key{jmp-if} differs from the concrete syntax
for x86 to separate the instruction name from the condition code. For
example, \code{(jmp-if le foo)} corresponds to \code{jle foo}.

\section{The $C_1$ Intermediate Language}
\label{sec:c1}

As with $R_1$, we shall compile $R_2$ to a C-like intermediate
language, but we need to grow that intermediate language to handle the
new features in $R_2$: Booleans and conditional expressions.
Figure~\ref{fig:c1-syntax} shows the new features of $C_1$; we add
logic and comparison operators to the $\Exp$ non-terminal, the
literals \key{\#t} and \key{\#f} to the $\Arg$ non-terminal.
Regarding control flow, $C_1$ differs considerably from $R_2$.
Instead of \key{if} expressions, $C_1$ has goto's and conditional
goto's in the grammar for $\Tail$. This means that a sequence of
statements may now end with a \code{goto} or a conditional
\code{goto}, which jumps to one of two labeled pieces of code
depending on the outcome of the comparison. In
Section~\ref{sec:explicate-control-r2} we discuss how to translate
from $R_2$ to $C_1$, bridging this gap between \key{if} expressions
and \key{goto}'s.

\begin{figure}[tp]
\fbox{
\begin{minipage}{0.96\textwidth}
\[
\begin{array}{lcl}
\Arg &::=& \gray{\Int \mid \Var} \mid \key{\#t} \mid \key{\#f} \\
\itm{cmp} &::= & \key{eq?} \mid \key{<}  \\
\Exp &::= & \gray{\Arg \mid (\key{read}) \mid (\key{-}\;\Arg) \mid (\key{+} \; \Arg\;\Arg)}
      \mid (\key{not}\;\Arg) \mid (\itm{cmp}\;\Arg\;\Arg) \\
\Stmt &::=& \gray{ \ASSIGN{\Var}{\Exp} } \\
\Tail &::= & \gray{\RETURN{\Exp} \mid (\key{seq}\;\Stmt\;\Tail)} \\
      &\mid& (\key{goto}\,\itm{label}) \mid \IF{(\itm{cmp}\, \Arg\,\Arg)}{(\key{goto}\,\itm{label})}{(\key{goto}\,\itm{label})} \\
C_1 & ::= & (\key{program}\;\itm{info}\; ((\itm{label}\,\key{.}\,\Tail)^{+}))
\end{array}
\]
\end{minipage}
}
\caption{The $C_1$ language, extending $C_0$ with Booleans and conditionals.}
\label{fig:c1-syntax}
\end{figure}

\section{Explicate Control}
\label{sec:explicate-control-r2}

Recall that the purpose of \code{explicate-control} is to make the
order of evaluation explicit in the syntax of the program.  With the
addition of \key{if} in $R_2$, things get more interesting.

As a motivating example, consider the following program that has an
\key{if} expression nested in the predicate of another \key{if}.
% s1_38.rkt
\begin{lstlisting}
    (program ()
      (if (if (eq? (read) 1)
              (eq? (read) 0)
              (eq? (read) 2))
          (+ 10 32)
          (+ 700 77)))
\end{lstlisting}
%
The naive way to compile \key{if} and \key{eq?} would be to handle
each of them in isolation, regardless of their context.  Each
\key{eq?} would be translated into a \key{cmpq} instruction followed
by a couple instructions to move the result from the EFLAGS register
into a general purpose register or stack location. Each \key{if} would
be translated into the combination of a \key{cmpq} and \key{jmp-if}.
However, if we take context into account we can do better and reduce
the use of \key{cmpq} and EFLAG-accessing instructions.

One idea is to try and reorganize the code at the level of $R_2$,
pushing the outer \key{if} inside the inner one. This would yield the
following code.
\begin{lstlisting}
    (if (eq? (read) 1)
        (if (eq? (read) 0)
                (+ 10 32)
                (+ 700 77))
            (if (eq? (read) 2))
                (+ 10 32)
                (+ 700 77))
\end{lstlisting}
Unfortunately, this approach duplicates the two branches, and a
compiler must never duplicate code!

We need a way to perform the above transformation, but without
duplicating code. The solution is straightforward if we think at the
level of x86 assembly: we can label the code for each of the branches
and insert \key{goto}'s in all the places that need to execute the
branches. Put another way, we need to move away from abstract syntax
\emph{trees} and instead use \emph{graphs}. In particular, we shall
use a standard program representation called a \emph{control flow
  graph} (CFG), due to Frances Elizabeth \citet{Allen:1970uq}.  Each
vertex is a labeled sequence of code, called a \emph{basic block}, and
each edge represents a jump to another block. The \key{program}
construct of $C_0$ and $C_1$ represents a control flow graph as an
association list mapping labels to basic blocks. Each block is
represented by the $\Tail$ non-terminal.

Figure~\ref{fig:explicate-control-s1-38} shows the output of the
\code{remove-complex-opera*} pass and then the
\code{explicate-control} pass on the example program. We shall walk
through the output program and then discuss the algorithm.
%
Following the order of evaluation in the output of
\code{remove-complex-opera*}, we first have the \code{(read)} and
comparison to \code{1} from the predicate of the inner \key{if}.  In
the output of \code{explicate-control}, in the \code{start} block,
this becomes a \code{(read)} followed by a conditional goto to either
\code{block61} or \code{block62}. Each of these contains the
translations of the code \code{(eq? (read) 0)} and \code{(eq? (read)
  1)}, respectively. Regarding \code{block61}, we start with the
\code{(read)} and comparison to \code{0} and then have a conditional
goto, either to \code{block59} or \code{block60}, which indirectly
take us to \code{block55} and \code{block56}, the two branches of the
outer \key{if}, i.e., \code{(+ 10 32)} and \code{(+ 700 77)}. The
story for \code{block62} is similar.

\begin{figure}[tbp]
\begin{tabular}{lll}
\begin{minipage}{0.4\textwidth}
\begin{lstlisting}
(program ()
  (if (if (eq? (read) 1)
          (eq? (read) 0)
          (eq? (read) 2))
      (+ 10 32)
      (+ 700 77)))  
\end{lstlisting}
\hspace{40pt}$\Downarrow$
\begin{lstlisting}
(program ()
  (if (if (let ([tmp52 (read)])
            (eq? tmp52 1))
          (let ([tmp53 (read)]) 
            (eq? tmp53 0))
          (let ([tmp54 (read)]) 
            (eq? tmp54 2)))
   (+ 10 32)
   (+ 700 77)))
\end{lstlisting}
\end{minipage}
&
$\Rightarrow$
&
\begin{minipage}{0.55\textwidth}
\begin{lstlisting}
(program ()
  ((block62 .
     (seq (assign tmp54 (read))
          (if (eq? tmp54 2)
              (goto block59)
              (goto block60))))
   (block61 .
     (seq (assign tmp53 (read))
          (if (eq? tmp53 0)
               (goto block57)
               (goto block58))))
   (block60 . (goto block56))
   (block59 . (goto block55))
   (block58 . (goto block56))
   (block57 . (goto block55))
   (block56 . (return (+ 700 77)))
   (block55 . (return (+ 10 32)))
   (start . 
     (seq (assign tmp52 (read))
          (if (eq? tmp52 1)
               (goto block61)
               (goto block62))))))
\end{lstlisting}
\end{minipage}
\end{tabular} 

\caption{Example translation from $R_2$ to $C_1$
  via the \code{explicate-control}.}
\label{fig:explicate-control-s1-38}
\end{figure}

The nice thing about the output of \code{explicate-control} is that
there are no unnecessary uses of \code{eq?} and every use of
\code{eq?} is part of a conditional jump. The down-side of this output
is that it includes trivial blocks, such as \code{block57} through
\code{block60}, that only jump to another block. We discuss a solution
to this problem in Section~\ref{sec:opt-jumps}.

Recall that in Section~\ref{sec:explicate-control-r1} we implement the
\code{explicate-control} pass for $R_1$ using two mutually recursive
functions, \code{explicate-control-tail} and
\code{explicate-control-assign}.  The former function translated
expressions in tail position whereas the later function translated
expressions on the right-hand-side of a \key{let}. With the addition
of \key{if} expression in $R_2$ we have a new kind of context to deal
with: the predicate position of the \key{if}. So we shall need another
function, \code{explicate-control-pred}, that takes an $R_2$
expression and two pieces of $C_1$ code (two $\Tail$'s) for the
then-branch and else-branch. The output of
\code{explicate-control-pred} is a $C_1$ $\Tail$.  However, these
three functions also need to contruct the control-flow graph, which we
recommend they do via updates to a global variable. Next we consider
the specific additions to the tail and assign functions, and some of
cases for the pred function.

The \code{explicate-control-tail} function needs an additional case
for \key{if}. The branches of the \key{if} inherit the current
context, so they are in tail position.  Let $B_1$ be the result of
\code{explicate-control-tail} on the $\itm{thn}$ branch and $B_2$ be
the result of apply \code{explicate-control-tail} to the $\itm{else}$
branch. Then the \key{if} translates to the block $B_3$ which is the
result of applying \code{explicate-control-pred} to the predicate
$\itm{cnd}$ and the blocks $B_1$ and $B_2$.
\[
    (\key{if}\; \itm{cnd}\; \itm{thn}\; \itm{els}) \quad\Rightarrow\quad B_3
\]

Next we consider the case for \key{if} in the
\code{explicate-control-assign} function. So the context of the
\key{if} is an assignment to some variable $x$ and then the control
continues to some block $B_1$.  The code that we generate for both the
$\itm{thn}$ and $\itm{els}$ branches shall both need to continue to
$B_1$, so we add $B_1$ to the control flow graph with a fresh label
$\ell_1$.  Again, the branches of the \key{if} inherit the current
context, so that are in assignment positions.  Let $B_2$ be the result
of applying \code{explicate-control-assign} to the $\itm{thn}$ branch,
variable $x$, and the block \code{(goto $\ell_1$)}.  Let $B_3$ be the
result of applying \code{explicate-control-assign} to the $\itm{else}$
branch, variable $x$, and the block \code{(goto $\ell_1$)}. The
\key{if} translates to the block $B_4$ which is the result of applying
\code{explicate-control-pred} to the predicate $\itm{cnd}$ and the
blocks $B_2$ and $B_3$.
\[
(\key{if}\; \itm{cnd}\; \itm{thn}\; \itm{els}) \quad\Rightarrow\quad B_4
\]

The function \code{explicate-control-pred} will need a case for every
expression that can have type \code{Boolean}. We detail a few cases
here and leave the rest for the reader. The input to this function is
an expression and two blocks, $B_1$ and $B_2$, for the branches of the
enclosing \key{if}. One of the base cases of this function is when the
expression is a less-than comparision. We translate it to a
conditional \code{goto}. We need labels for the two branches $B_1$ and
$B_2$, so we add them to the control flow graph and obtain some labels
$\ell_1$ and $\ell_2$. The translation of the less-than comparison is
as follows.
\[
(\key{<}\;e_1\;e_2) \quad\Rightarrow\quad
(\key{if}\;(\key{<}\;e_1\;e_2)\;(\key{goto}\;\ell_1)\;(\key{goto}\;\ell_2))
\]

The case for \key{if} in \code{explicate-control-pred} is particularly
illuminating, as it deals with the challenges that we discussed above
regarding the example of the nested \key{if} expressions.  Again, we
add the two input branches $B_1$ and $B_2$ to the control flow graph
and obtain the labels $\ell_1$ and $\ell_2$.  The branches $\itm{thn}$
and $\itm{els}$ of the current \key{if} inherit their context from the
current one, i.e., predicate context. So we apply
\code{explicate-control-pred} to $\itm{thn}$ with the two blocks
\code{(goto $\ell_1$)} and \code{(goto $\ell_2$)}, to obtain $B_3$.
Similarly for the $\itm{els}$ branch, to obtain $B_4$.
Finally, we apply \code{explicate-control-pred} to
the predicate $\itm{cnd}$ and the blocks $B_3$ and $B_4$
to obtain the result $B_5$.
\[
(\key{if}\; \itm{cnd}\; \itm{thn}\; \itm{els})
\quad\Rightarrow\quad
B_5
\]

\begin{exercise}\normalfont
  Implement the pass \code{explicate-code} by adding the cases for
  \key{if} to the functions for tail and assignment contexts, and
  implement the function for predicate contexts. Create test cases
  that exercise all of the new cases in the code for this pass.
\end{exercise}


\section{Select Instructions}
\label{sec:select-r2}

Recall that the \code{select-instructions} pass lowers from our
$C$-like intermediate representation to the pseudo-x86 language, which
is suitable for conducting register allocation. The pass is
implemented using three auxilliary functions, one for each of the
non-terminals $\Arg$, $\Stmt$, and $\Tail$.

For $\Arg$, we have new cases for the Booleans.  We take the usual
approach of encoding them as integers, with true as 1 and false as 0.
\[
\key{\#t} \Rightarrow \key{1}
\qquad
\key{\#f} \Rightarrow \key{0}
\]

For $\Stmt$, we discuss a couple cases.  The \code{not} operation can
be implemented in terms of \code{xorq} as we discussed at the
beginning of this section. Given an assignment \code{(assign
  $\itm{lhs}$ (not $\Arg$))}, if the left-hand side $\itm{lhs}$ is
the same as $\Arg$, then just the \code{xorq} suffices:
\[
(\key{assign}\; x\; (\key{not}\; x))
\quad\Rightarrow\quad
((\key{xorq}\;(\key{int}\;1)\;x'))
\]
Otherwise, a \key{movq} is needed to adapt to the update-in-place
semantics of x86. Let $\Arg'$ be the result of recursively processing
$\Arg$. Then we have
\[
(\key{assign}\; \itm{lhs}\; (\key{not}\; \Arg))
\quad\Rightarrow\quad
((\key{movq}\; \Arg'\; \itm{lhs}') \; (\key{xorq}\;(\key{int}\;1)\;\itm{lhs}'))
\]

Next consider the cases for \code{eq?} and less-than comparison.
Translating these operations to x86 is slightly involved due to the
unusual nature of the \key{cmpq} instruction discussed above.  We
recommend translating an assignment from \code{eq?} into the following
sequence of three instructions. \\
\begin{tabular}{lll}
\begin{minipage}{0.4\textwidth}
\begin{lstlisting}
 (assign |$\itm{lhs}$| (eq? |$\Arg_1$| |$\Arg_2$|))
\end{lstlisting}
\end{minipage}
&
$\Rightarrow$
&
\begin{minipage}{0.4\textwidth}
\begin{lstlisting}
(cmpq |$\Arg'_2$| |$\Arg'_1$|)
(set e (byte-reg al))
(movzbq (byte-reg al) |$\itm{lhs}'$|)
\end{lstlisting}
\end{minipage}
\end{tabular}  \\

Regarding the $\Tail$ non-terminal, we have two new cases, for
\key{goto} and conditional \key{goto}. Both are straightforward
to handle. A \key{goto} becomes a jump instruction.
\[
(\key{goto}\; \ell) \quad \Rightarrow \quad ((\key{jmp} \;\ell))
\]
A conditional \key{goto} becomes a compare instruction followed
by a conditional jump (for ``then'') and the fall-through is
to a regular jump (for ``else'').\\
\begin{tabular}{lll}
\begin{minipage}{0.4\textwidth}
\begin{lstlisting}
  (if (eq? |$\Arg_1$| |$\Arg_2$|)
      (goto |$\ell_1$|)
      (goto |$\ell_2$|))
\end{lstlisting}
\end{minipage}
&
$\Rightarrow$
&
\begin{minipage}{0.4\textwidth}
\begin{lstlisting}
((cmpq |$\Arg'_2$| |$\Arg'_1$|)
 (jmp-if e |$\ell_1$|)
 (jmp |$\ell_2$|))
\end{lstlisting}
\end{minipage}
\end{tabular}  \\

\begin{exercise}\normalfont
Expand your \code{select-instructions} pass to handle the new features
of the $R_2$ language. Test the pass on all the examples you have
created and make sure that you have some test programs that use the
\code{eq?} and \code{<} operators, creating some if necessary. Test
the output using the \code{interp-x86} interpreter
(Appendix~\ref{appendix:interp}).
\end{exercise}

\section{Register Allocation}
\label{sec:register-allocation-r2}

The changes required for $R_2$ affect the liveness analysis, building
the interference graph, and assigning homes, but the graph coloring
algorithm itself does not need to change.

\subsection{Liveness Analysis}
\label{sec:liveness-analysis-r2}

Recall that for $R_1$ we implemented liveness analysis for a single
basic block (Section~\ref{sec:liveness-analysis-r1}). With the
addition of \key{if} expressions to $R_2$, \code{explicate-control}
now produces many basic blocks arranged in a control-flow graph. The
first question we need to consider is in what order should we process
the basic blocks? Recall that to perform liveness analysis, we need to
know the live-after set. If a basic block has no successor blocks,
then it has an empty live-after set and we can immediately apply
liveness analysis to it. If a basic block has some successors, then we
need to complete liveness analysis on those blocks first.
Furthermore, we know that the control flow graph does not contain any
cycles (it is a DAG, that is, a directed acyclic graph)\footnote{If we
  were to add loops to the language, then the CFG could contain cycles
  and we would instead need to use the classic worklist algorithm for
  computing the fixed point of the liveness
  analysis~\citep{Aho:1986qf}.}. What all this amounts to is that we
need to process the basic blocks in reverse topological order. We
recommend using the \code{tsort} and \code{transpose} functions of the
Racket \code{graph} package to obtain this ordering.

The next question is how to compute the live-after set of a block
given the live-before sets of all its successor blocks.  During
compilation we do not know which way the branch will go, so we do not
know which of the successor's live-before set to use.  The solution
comes from the observation that there is no harm in identifying more
variables as live than absolutely necessary. Thus, we can take the
union of the live-before sets from all the successors to be the
live-after set for the block. Once we have computed the live-after
set, we can proceed to perform liveness analysis on the block just as
we did in Section~\ref{sec:liveness-analysis-r1}.

The helper functions for computing the variables in an instruction's
argument and for computing the variables read-from ($R$) or written-to
($W$) by an instruction need to be updated to handle the new kinds of
arguments and instructions in x86$_1$.

\subsection{Build Interference}
\label{sec:build-interference-r2}

Many of the new instructions in x86$_1$ can be handled in the same way
as the instructions in x86$_0$. Thus, if your code was already quite
general, it will not need to be changed to handle the new
instructions. If not, I recommend that you change your code to be more
general. The \key{movzbq} instruction should be handled like the
\key{movq} instruction.

%% \subsection{Assign Homes}
%% \label{sec:assign-homes-r2}

%% The \code{assign-homes} function (Section~\ref{sec:assign-r1}) needs
%% to be updated to handle the \key{if} statement, simply by recursively
%% processing the child nodes.  Hopefully your code already handles the
%% other new instructions, but if not, you can generalize your code.

\begin{exercise}\normalfont
Update the \code{register-allocation} pass so that it works for $R_2$
and test your compiler using your previously created programs on the
\code{interp-x86} interpreter (Appendix~\ref{appendix:interp}).
\end{exercise}


%% \section{Lower Conditionals (New Pass)}
%% \label{sec:lower-conditionals}

%% In the \code{select-instructions} pass we decided to procrastinate in
%% the lowering of the \key{if} statement, thereby making liveness
%% analysis easier. Now we need to make up for that and turn the \key{if}
%% statement into the appropriate instruction sequence.  The following
%% translation gives the general idea. If the condition is true, we need
%% to execute the $\itm{thns}$ branch and otherwise we need to execute
%% the $\itm{elss}$ branch. So we use \key{cmpq} and do a conditional
%% jump to the $\itm{thenlabel}$, choosing the condition code $cc$ that
%% is appropriate for the comparison operator \itm{cmp}.  If the
%% condition is false, we fall through to the $\itm{elss}$ branch. At the
%% end of the $\itm{elss}$ branch we need to take care to not fall
%% through to the $\itm{thns}$ branch. So we jump to the
%% $\itm{endlabel}$. All of the labels in the generated code should be
%% created with \code{gensym}.

%% \begin{tabular}{lll}
%% \begin{minipage}{0.4\textwidth}
%% \begin{lstlisting}
%%  (if (|\itm{cmp}| |$\Arg_1$| |$\Arg_2$|) |$\itm{thns}$| |$\itm{elss}$|)
%% \end{lstlisting}
%% \end{minipage}
%% &
%% $\Rightarrow$
%% &
%% \begin{minipage}{0.4\textwidth}
%% \begin{lstlisting}
%%  (cmpq |$\Arg_2$| |$\Arg_1$|)
%%  (jmp-if |$cc$| |$\itm{thenlabel}$|)
%%  |$\itm{elss}$|
%%  (jmp |$\itm{endlabel}$|)
%%  (label |$\itm{thenlabel}$|)
%%  |$\itm{thns}$|
%%  (label |$\itm{endlabel}$|)
%% \end{lstlisting}
%% \end{minipage}
%% \end{tabular}

%% \begin{exercise}\normalfont
%% Implement the \code{lower-conditionals} pass. Test your compiler using
%% your previously created programs on the \code{interp-x86} interpreter
%% (Appendix~\ref{appendix:interp}).
%% \end{exercise}

\section{Patch Instructions}

The second argument of the \key{cmpq} instruction must not be an
immediate value (such as a literal integer). So if you are comparing
two immediates, we recommend inserting a \key{movq} instruction to put
the second argument in \key{rax}.
%
The second argument of the \key{movzbq} must be a register.
%
There are no special restrictions on the x86 instructions
\key{jmp-if}, \key{jmp}, and \key{label}. 

\begin{exercise}\normalfont
Update \code{patch-instructions} to handle the new x86 instructions.
Test your compiler using your previously created programs on the
\code{interp-x86} interpreter (Appendix~\ref{appendix:interp}).
\end{exercise}


\section{An Example Translation}

Figure~\ref{fig:if-example-x86} shows a simple example program in
$R_2$ translated to x86, showing the results of
\code{explicate-control}, \code{select-instructions}, and the final
x86 assembly code.

\begin{figure}[tbp]
\begin{tabular}{lll}
\begin{minipage}{0.5\textwidth}
% s1_20.rkt
\begin{lstlisting}
(program ()
  (if (eq? (read) 1) 42 0))
\end{lstlisting}
$\Downarrow$
\begin{lstlisting}
(program ()
 ((block32 . (return 0))
  (block31 . (return 42))
  (start .
     (seq (assign tmp30 (read))
          (if (eq? tmp30 1)
              (goto block31)
              (goto block32))))))
\end{lstlisting}
$\Downarrow$
\begin{lstlisting}
(program ((locals . (tmp30)))
 ((block32 .
    (block ()
      (movq (int 0) (reg rax))
      (jmp conclusion)))
  (block31 .
    (block () 
      (movq (int 42) (reg rax))
      (jmp conclusion)))
  (start .
    (block ()
      (callq read_int)
      (movq (reg rax) (var tmp30))
      (cmpq (int 1) (var tmp30))
      (jmp-if e block31)
      (jmp block32)))))
\end{lstlisting}
\end{minipage}
&
$\Rightarrow$
\begin{minipage}{0.4\textwidth}
\begin{lstlisting}
_block31:
	movq	$42, %rax
	jmp _conclusion
_block32:
	movq	$0, %rax
	jmp _conclusion
_start:
	callq	_read_int
	movq	%rax, %rcx
	cmpq	$1, %rcx
	je _block31
	jmp _block32

	.globl _main
_main:
	pushq	%rbp
	movq	%rsp, %rbp
	pushq	%r12
	pushq	%rbx
	pushq	%r13
	pushq	%r14
	subq	$0, %rsp
	jmp _start
_conclusion:
	addq	$0, %rsp
	popq	%r14
	popq	%r13
	popq	%rbx
	popq	%r12
	popq	%rbp
	retq
\end{lstlisting}
\end{minipage}
\end{tabular}
\caption{Example compilation of an \key{if} expression to x86.}
\label{fig:if-example-x86}
\end{figure}


\begin{figure}[p]
\begin{tikzpicture}[baseline=(current  bounding  box.center)]
\node (R2) at (0,2)  {\large $R_2$};
\node (R2-2) at (3,2)  {\large $R_2$};
\node (R2-3) at (6,2)  {\large $R_2$};
\node (R2-4) at (9,2)  {\large $R_2$};
\node (R2-5) at (12,2)  {\large $R_2$};
\node (C1-1) at (6,0)  {\large $C_1$};
\node (C1-2) at (3,0)  {\large $C_1$};

\node (x86-2) at (3,-2)  {\large $\text{x86}^{*}$};
\node (x86-3) at (6,-2)  {\large $\text{x86}^{*}$};
\node (x86-4) at (9,-2) {\large $\text{x86}^{*}$};
\node (x86-5) at (12,-2) {\large $\text{x86}^{\dagger}$};

\node (x86-2-1) at (3,-4)  {\large $\text{x86}^{*}$};
\node (x86-2-2) at (6,-4)  {\large $\text{x86}^{*}$};

\path[->,bend left=15] (R2) edge [above] node {\ttfamily\footnotesize\color{red} typecheck} (R2-2);
\path[->,bend left=15] (R2-2) edge [above] node {\ttfamily\footnotesize\color{red} shrink} (R2-3);
\path[->,bend left=15] (R2-3) edge [above] node {\ttfamily\footnotesize uniquify} (R2-4);
\path[->,bend left=15] (R2-4) edge [above] node {\ttfamily\footnotesize remove-complex.} (R2-5);
\path[->,bend left=15] (R2-5) edge [right] node {\ttfamily\footnotesize\color{red} explicate-control} (C1-1);
\path[->,bend right=15] (C1-1) edge [above] node {\ttfamily\footnotesize uncover-locals} (C1-2);
\path[->,bend right=15] (C1-2) edge [left] node {\ttfamily\footnotesize\color{red} select-instr.} (x86-2);
\path[->,bend left=15] (x86-2) edge [right] node {\ttfamily\footnotesize\color{red} uncover-live} (x86-2-1);
\path[->,bend right=15] (x86-2-1) edge [below] node {\ttfamily\footnotesize build-inter.} (x86-2-2);
\path[->,bend right=15] (x86-2-2) edge [right] node {\ttfamily\footnotesize allocate-reg.} (x86-3);
\path[->,bend left=15] (x86-3) edge [above] node {\ttfamily\footnotesize\color{red} patch-instr.} (x86-4);
\path[->,bend left=15] (x86-4) edge [above] node {\ttfamily\footnotesize\color{red} print-x86 } (x86-5);
\end{tikzpicture}
\caption{Diagram of the passes for $R_2$, a language with conditionals.}
 \label{fig:R2-passes}
\end{figure}

Figure~\ref{fig:R2-passes} lists all the passes needed for the
compilation of $R_2$.

\section{Challenge: Optimize Jumps$^{*}$}
\label{sec:opt-jumps}

UNDER CONSTRUCTION


%% \section{Challenge: Optimizing Conditions$^{*}$}
%% \label{sec:opt-if}

%% A close inspection of the x86 code generated in
%% Figure~\ref{fig:if-example-x86} reveals some redundant computation
%% regarding the condition of the \key{if}. We compare \key{rcx} to $1$
%% twice using \key{cmpq} as follows.

%% % Wierd LaTeX bug if I remove the following. -Jeremy
%% % Does it have to do with page breaks?
%% \begin{lstlisting}
%% \end{lstlisting}

%% \begin{lstlisting}
%% 	cmpq	$1, %rcx
%% 	sete	%al
%% 	movzbq	%al, %rcx
%% 	cmpq	$1, %rcx
%% 	je then21288
%% \end{lstlisting}


%% The reason for this non-optimal code has to do with the \code{flatten}
%% pass earlier in this Chapter. We recommended flattening the condition
%% to an $\Arg$ and then comparing with \code{\#t}. But if the condition
%% is already an \code{eq?} test, then we would like to use that
%% directly. In fact, for many of the expressions of Boolean type, we can
%% generate more optimized code. For example, if the condition is
%% \code{\#t} or \code{\#f}, we do not need to generate an \code{if} at
%% all. If the condition is a \code{let}, we can optimize based on the
%% form of its body. If the condition is a \code{not}, then we can flip
%% the two branches.
%% %
%% \margincomment{\tiny We could do even better by converting to basic
%%   blocks.\\ --Jeremy}
%% %
%% On the other hand, if the condition is a \code{and}
%% or another \code{if}, we should flatten them into an $\Arg$ to avoid
%% code duplication.

%% Figure~\ref{fig:opt-if} shows an example program and the result of
%% applying the above suggested optimizations.

%% \begin{exercise}\normalfont
%%   Change the \code{flatten} pass to improve the code that gets
%%   generated for \code{if} expressions. We recommend writing a helper
%%   function that recursively traverses the condition of the \code{if}.
%% \end{exercise}

%% \begin{figure}[tbp]
%% \begin{tabular}{lll}
%% \begin{minipage}{0.5\textwidth}
%% \begin{lstlisting}
%% (program
%%   (if (let ([x 1])
%%         (not (eq? x (read))))
%%     777
%%     42))
%% \end{lstlisting}
%% $\Downarrow$
%% \begin{lstlisting}
%% (program (x.1 if.2 tmp.3)
%%   (type Integer)
%%   (assign x.1 1)
%%   (assign tmp.3 (read))
%%   (if (eq? x.1 tmp.3)
%%     ((assign if.2 42))
%%     ((assign if.2 777)))
%%   (return if.2))
%% \end{lstlisting}
%% $\Downarrow$
%% \begin{lstlisting}
%% (program (x.1 if.2 tmp.3)
%%   (type Integer)
%%   (movq (int 1) (var x.1))
%%   (callq read_int)
%%   (movq (reg rax) (var tmp.3))
%%   (if (eq? (var x.1) (var tmp.3))
%%     ((movq (int 42) (var if.2)))
%%     ((movq (int 777) (var if.2))))
%%   (movq (var if.2) (reg rax)))
%% \end{lstlisting}
%% \end{minipage}
%% &
%% $\Rightarrow$
%% \begin{minipage}{0.4\textwidth}
%% \begin{lstlisting}
%% 	.globl _main
%% _main:
%% 	pushq	%rbp
%% 	movq	%rsp, %rbp
%% 	pushq	%r13
%% 	pushq	%r14
%% 	pushq	%r12
%% 	pushq	%rbx
%% 	subq	$0, %rsp

%% 	movq	$1, %rbx
%% 	callq	_read_int
%% 	movq	%rax, %rcx
%% 	cmpq	%rcx, %rbx
%% 	je then35989
%% 	movq	$777, %rbx
%% 	jmp if_end35990
%% then35989:
%% 	movq	$42, %rbx
%% if_end35990:
%% 	movq	%rbx, %rax

%% 	movq	%rax, %rdi
%% 	callq	_print_int
%% 	movq	$0, %rax
%% 	addq	$0, %rsp
%% 	popq	%rbx
%% 	popq	%r12
%% 	popq	%r14
%% 	popq	%r13
%% 	popq	%rbp
%% 	retq
%% \end{lstlisting}
%% \end{minipage}
%% \end{tabular}
%% \caption{Example program with optimized conditionals.}
%% \label{fig:opt-if}
%% \end{figure}

%%%%%%%%%%%%%%%%%%%%%%%%%%%%%%%%%%%%%%%%%%%%%%%%%%%%%%%%%%%%%%%%%%%%%%%%%%%%%%%%
\chapter{Tuples and Garbage Collection}
\label{ch:tuples}

\margincomment{\scriptsize To do: look through Andre's code comments for extra
  things to discuss in this chapter. \\ --Jeremy}
\margincomment{\scriptsize To do: Flesh out this chapter, e.g., make sure
  all the IR grammars are spelled out! \\ --Jeremy}
\margincomment{\scriptsize Introduce has-type, but after flatten, remove it,
  but keep type annotations on vector creation and local variables, function
  parameters, etc. \\ --Jeremy}
\margincomment{\scriptsize Be more explicit about how to deal with
  the root stack. \\ --Jeremy}

In this chapter we study the implementation of mutable tuples (called
``vectors'' in Racket). This language feature is the first to use the
computer's \emph{heap} because the lifetime of a Racket tuple is
indefinite, that is, a tuple lives forever from the programmer's
viewpoint. Of course, from an implementor's viewpoint, it is important
to reclaim the space associated with a tuple when it is no longer
needed, which is why we also study \emph{garbage collection}
techniques in this chapter.

Section~\ref{sec:r3} introduces the $R_3$ language including its
interpreter and type checker. The $R_3$ language extends the $R_2$
language of Chapter~\ref{ch:bool-types} with vectors and Racket's
``void'' value. The reason for including the later is that the
\code{vector-set!} operation returns a value of type
\code{Void}\footnote{This may sound contradictory, but Racket's
  \code{Void} type corresponds to what is more commonly called the
  \code{Unit} type. This type is inhabited by a single value that is
  usually written \code{unit} or \code{()}\citep{Pierce:2002hj}.}.

Section~\ref{sec:GC} describes a garbage collection algorithm based on
copying live objects back and forth between two halves of the
heap. The garbage collector requires coordination with the compiler so
that it can see all of the \emph{root} pointers, that is, pointers in
registers or on the procedure call stack.
Sections~\ref{sec:expose-allocation} through \ref{sec:print-x86-gc}
discuss all the necessary changes and additions to the compiler
passes, including a new compiler pass named \code{expose-allocation}.

\section{The $R_3$ Language}
\label{sec:r3}

Figure~\ref{fig:r3-syntax} defines the syntax for $R_3$, which
includes three new forms for creating a tuple, reading an element of a
tuple, and writing to an element of a tuple. The program in
Figure~\ref{fig:vector-eg} shows the usage of tuples in Racket. We
create a 3-tuple \code{t} and a 1-tuple. The 1-tuple is stored at
index $2$ of the 3-tuple, demonstrating that tuples are first-class
values.  The element at index $1$ of \code{t} is \code{\#t}, so the
``then'' branch is taken.  The element at index $0$ of \code{t} is
$40$, to which we add the $2$, the element at index $0$ of the
1-tuple.

\begin{figure}[tbp]
\begin{lstlisting}
    (let ([t (vector 40 #t (vector 2))])
      (if (vector-ref t 1)
          (+ (vector-ref t 0)
             (vector-ref (vector-ref t 2) 0))
          44))
\end{lstlisting}
\caption{Example program that creates tuples and reads from them.}
\label{fig:vector-eg}
\end{figure}

\begin{figure}[tbp]
\centering
\fbox{
\begin{minipage}{0.96\textwidth}
\[
\begin{array}{lcl}
  \Type &::=& \gray{\key{Integer} \mid \key{Boolean}}
  \mid (\key{Vector}\;\Type^{+}) \mid \key{Void}\\
  \itm{cmp} &::= & \gray{  \key{eq?} \mid \key{<} \mid \key{<=} \mid \key{>} \mid \key{>=}  } \\
  \Exp &::=& \gray{  \Int \mid (\key{read}) \mid (\key{-}\;\Exp) \mid (\key{+} \; \Exp\;\Exp) \mid (\key{-}\;\Exp\;\Exp) }  \\
  &\mid&  \gray{  \Var \mid \LET{\Var}{\Exp}{\Exp}  }\\
  &\mid& \gray{ \key{\#t} \mid \key{\#f} 
   \mid (\key{and}\;\Exp\;\Exp) 
   \mid (\key{or}\;\Exp\;\Exp)
   \mid (\key{not}\;\Exp) } \\
  &\mid& \gray{  (\itm{cmp}\;\Exp\;\Exp) 
   \mid \IF{\Exp}{\Exp}{\Exp}  } \\
  &\mid& (\key{vector}\;\Exp^{+}) 
   \mid (\key{vector-ref}\;\Exp\;\Int) \\
  &\mid& (\key{vector-set!}\;\Exp\;\Int\;\Exp)\\
  &\mid& (\key{void}) \\
  R_3 &::=& (\key{program} \; \Exp)
\end{array}
\]
\end{minipage}
}
\caption{The syntax of $R_3$, extending $R_2$
  (Figure~\ref{fig:r2-syntax}) with tuples.}
\label{fig:r3-syntax}
\end{figure}

Tuples are our first encounter with heap-allocated data, which raises
several interesting issues. First, variable binding performs a
shallow-copy when dealing with tuples, which means that different
variables can refer to the same tuple, i.e., different variables can
be \emph{aliases} for the same thing. Consider the following example
in which both \code{t1} and \code{t2} refer to the same tuple.  Thus,
the mutation through \code{t2} is visible when referencing the tuple
from \code{t1}, so the result of this program is \code{42}.
\begin{lstlisting}
    (let ([t1 (vector 3 7)])
      (let ([t2 t1])
        (let ([_ (vector-set! t2 0 42)])
          (vector-ref t1 0))))
\end{lstlisting}

The next issue concerns the lifetime of tuples. Of course, they are
created by the \code{vector} form, but when does their lifetime end?
Notice that the grammar in Figure~\ref{fig:r3-syntax} does not include
an operation for deleting tuples. Furthermore, the lifetime of a tuple
is not tied to any notion of static scoping. For example, the
following program returns \code{3} even though the variable \code{t}
goes out of scope prior to accessing the vector.
\begin{lstlisting}
    (vector-ref
      (let ([t (vector 3 7)])
        t)
      0)
\end{lstlisting}
From the perspective of programmer-observable behavior, tuples live
forever. Of course, if they really lived forever, then many programs
would run out of memory.\footnote{The $R_3$ language does not have
  looping or recursive function, so it is nigh impossible to write a
  program in $R_3$ that will run out of memory. However, we add
  recursive functions in the next Chapter!} A Racket implementation
must therefore perform automatic garbage collection.

Figure~\ref{fig:interp-R3} shows the definitional interpreter for the
$R_3$ language and Figure~\ref{fig:typecheck-R3} shows the type
checker. The additions to the interpreter are straightforward but the
updates to the type checker deserve some explanation.  As we shall see
in Section~\ref{sec:GC}, we need to know which variables are pointers
into the heap, that is, which variables are vectors. Also, when
allocating a vector, we shall need to know which elements of the
vector are pointers. We can obtain this information during type
checking and when we uncover local variables. The type checker in
Figure~\ref{fig:typecheck-R3} not only computes the type of an
expression, it also wraps every sub-expression $e$ with the form
$(\key{has-type}\; e\; T)$, where $T$ is $e$'s type. Subsequently, in
the \code{uncover-locals} pass (Section~\ref{sec:uncover-locals-r3})
this type information is propagated to all variables (including the
temporaries generated by \code{remove-complex-opera*}).

\begin{figure}[tbp]
\begin{lstlisting}
    (define primitives (set ... 'vector 'vector-ref 'vector-set!))

    (define (interp-op op)
      (match op
         ...
         ['vector vector]
         ['vector-ref vector-ref]
	 ['vector-set! vector-set!]
	 [else (error 'interp-op "unknown operator")]))

   (define (interp-R3 env)
     (lambda (e)
       (match e
         ...
         [else (error 'interp-R3 "unrecognized expression")]
         )))
\end{lstlisting}
\caption{Interpreter for the $R_3$ language.}
\label{fig:interp-R3}
\end{figure}

\begin{figure}[tbp]
\begin{lstlisting}
(define (type-check-exp env)
  (lambda (e)
    (define recur (type-check-exp env))
    (match e
      ...
      ['(void) (values '(has-type (void) Void) 'Void)]
      [`(vector ,es ...)
       (define-values (e* t*) (for/lists (e* t*) ([e es])
                                (recur e)))
       (let ([t `(Vector ,@t*)])
         (debug "vector/type-check-exp finished vector" t)
         (values `(has-type (vector ,@e*) ,t) t))]
      [`(vector-ref ,e ,i)
       (define-values (e^ t) (recur e))
       (match t
         [`(Vector ,ts ...)
          (unless (and (exact-nonnegative-integer? i) (< i (length ts)))
            (error 'type-check-exp "invalid index ~a" i))
          (let ([t (list-ref ts i)])
            (values `(has-type (vector-ref ,e^ (has-type ,i Integer)) ,t) 
                    t))]
         [else (error "expected a vector in vector-ref, not" t)])]
      [`(eq? ,arg1 ,arg2)
       (define-values (e1 t1) (recur arg1))
       (define-values (e2 t2) (recur arg2))
       (match* (t1 t2)
         [(`(Vector ,ts1 ...) `(Vector ,ts2 ...))
          (values `(has-type (eq? ,e1 ,e2) Boolean) 'Boolean)]
         [(other wise) ((super type-check-exp env) e)])]
      ...
      )))
\end{lstlisting}
\caption{Type checker for the $R_3$ language.}
\label{fig:typecheck-R3}
\end{figure}


\section{Garbage Collection}
\label{sec:GC}

Here we study a relatively simple algorithm for garbage collection
that is the basis of state-of-the-art garbage
collectors~\citep{Lieberman:1983aa,Ungar:1984aa,Jones:1996aa,Detlefs:2004aa,Dybvig:2006aa,Tene:2011kx}. In
particular, we describe a two-space copying
collector~\citep{Wilson:1992fk} that uses Cheney's algorithm to
perform the
copy~\citep{Cheney:1970aa}. Figure~\ref{fig:copying-collector} gives a
coarse-grained depiction of what happens in a two-space collector,
showing two time steps, prior to garbage collection on the top and
after garbage collection on the bottom. In a two-space collector, the
heap is divided into two parts, the FromSpace and the
ToSpace. Initially, all allocations go to the FromSpace until there is
not enough room for the next allocation request. At that point, the
garbage collector goes to work to make more room.


The garbage collector must be careful not to reclaim tuples that will
be used by the program in the future. Of course, it is impossible in
general to predict what a program will do, but we can overapproximate
the will-be-used tuples by preserving all tuples that could be
accessed by \emph{any} program given the current computer state.  A
program could access any tuple whose address is in a register or on
the procedure call stack. These addresses are called the \emph{root
  set}. In addition, a program could access any tuple that is
transitively reachable from the root set. Thus, it is safe for the
garbage collector to reclaim the tuples that are not reachable in this
way.

So the goal of the garbage collector is twofold:
\begin{enumerate}
\item preserve all tuple that are reachable from the root set via a
  path of pointers, that is, the \emph{live} tuples, and
\item reclaim the memory of everything else, that is, the
  \emph{garbage}.
\end{enumerate}
A copying collector accomplishes this by copying all of the live
objects from the FromSpace into the ToSpace and then performs a slight
of hand, treating the ToSpace as the new FromSpace and the old
FromSpace as the new ToSpace.  In the example of
Figure~\ref{fig:copying-collector}, there are three pointers in the
root set, one in a register and two on the stack.  All of the live
objects have been copied to the ToSpace (the right-hand side of
Figure~\ref{fig:copying-collector}) in a way that preserves the
pointer relationships. For example, the pointer in the register still
points to a 2-tuple whose first element is a 3-tuple and second
element is a 2-tuple.  There are four tuples that are not reachable
from the root set and therefore do not get copied into the ToSpace.
(The sitation in Figure~\ref{fig:copying-collector}, with a
cycle, cannot be created by a well-typed program in $R_3$. However,
creating cycles will be possible once we get to $R_6$.  We design
the garbage collector to deal with cycles to begin with, so we will
not need to revisit this issue.)

\begin{figure}[tbp]
\centering
\includegraphics[width=\textwidth]{figs/copy-collect-1} \\[5ex]
\includegraphics[width=\textwidth]{figs/copy-collect-2}
\caption{A copying collector in action.}
\label{fig:copying-collector}
\end{figure}

There are many alternatives to copying collectors (and their older
siblings, the generational collectors) when its comes to garbage
collection, such as mark-and-sweep and reference counting.  The
strengths of copying collectors are that allocation is fast (just a
test and pointer increment), there is no fragmentation, cyclic garbage
is collected, and the time complexity of collection only depends on
the amount of live data, and not on the amount of
garbage~\citep{Wilson:1992fk}. The main disadvantage of two-space
copying collectors is that they use a lot of space, though that
problem is ameliorated in generational collectors.  Racket and Scheme
programs tend to allocate many small objects and generate a lot of
garbage, so copying and generational collectors are a good fit.  Of
course, garbage collection is an active research topic, especially
concurrent garbage collection~\citep{Tene:2011kx}. Researchers are
continuously developing new techniques and revisiting old
trade-offs~\citep{Blackburn:2004aa,Jones:2011aa,Shahriyar:2013aa,Cutler:2015aa,Shidal:2015aa}.

\subsection{Graph Copying via Cheney's Algorithm}
\label{sec:cheney}

Let us take a closer look at how the copy works. The allocated objects
and pointers can be viewed as a graph and we need to copy the part of
the graph that is reachable from the root set. To make sure we copy
all of the reachable vertices in the graph, we need an exhaustive
graph traversal algorithm, such as depth-first search or breadth-first
search~\citep{Moore:1959aa,Cormen:2001uq}. Recall that such algorithms
take into account the possibility of cycles by marking which vertices
have already been visited, so as to ensure termination of the
algorithm. These search algorithms also use a data structure such as a
stack or queue as a to-do list to keep track of the vertices that need
to be visited. We shall use breadth-first search and a trick due to
\citet{Cheney:1970aa} for simultaneously representing the queue and
copying tuples into the ToSpace.

Figure~\ref{fig:cheney} shows several snapshots of the ToSpace as the
copy progresses. The queue is represented by a chunk of contiguous
memory at the beginning of the ToSpace, using two pointers to track
the front and the back of the queue. The algorithm starts by copying
all tuples that are immediately reachable from the root set into the
ToSpace to form the initial queue.  When we copy a tuple, we mark the
old tuple to indicate that it has been visited. (We discuss the
marking in Section~\ref{sec:data-rep-gc}.) Note that any pointers
inside the copied tuples in the queue still point back to the
FromSpace. Once the initial queue has been created, the algorithm
enters a loop in which it repeatedly processes the tuple at the front
of the queue and pops it off the queue.  To process a tuple, the
algorithm copies all the tuple that are directly reachable from it to
the ToSpace, placing them at the back of the queue. The algorithm then
updates the pointers in the popped tuple so they point to the newly
copied tuples. Getting back to Figure~\ref{fig:cheney}, in the first
step we copy the tuple whose second element is $42$ to the back of the
queue. The other pointer goes to a tuple that has already been copied,
so we do not need to copy it again, but we do need to update the
pointer to the new location. This can be accomplished by storing a
\emph{forwarding} pointer to the new location in the old tuple, back
when we initially copied the tuple into the ToSpace. This completes
one step of the algorithm. The algorithm continues in this way until
the front of the queue is empty, that is, until the front catches up
with the back.

\begin{figure}[tbp]
\centering \includegraphics[width=0.9\textwidth]{figs/cheney}
\caption{Depiction of the Cheney algorithm copying the live tuples.}
\label{fig:cheney}
\end{figure}


\subsection{Data Representation}
\label{sec:data-rep-gc}

The garbage collector places some requirements on the data
representations used by our compiler. First, the garbage collector
needs to distinguish between pointers and other kinds of data. There
are several ways to accomplish this.
\begin{enumerate}
\item Attached a tag to each object that identifies what type of
  object it is~\citep{McCarthy:1960dz}.
\item Store different types of objects in different
  regions~\citep{Steele:1977ab}.
\item Use type information from the program to either generate
  type-specific code for collecting or to generate tables that can
  guide the
  collector~\citep{Appel:1989aa,Goldberg:1991aa,Diwan:1992aa}.
\end{enumerate}
Dynamically typed languages, such as Lisp, need to tag objects
anyways, so option 1 is a natural choice for those languages.
However, $R_3$ is a statically typed language, so it would be
unfortunate to require tags on every object, especially small and
pervasive objects like integers and Booleans.  Option 3 is the
best-performing choice for statically typed languages, but comes with
a relatively high implementation complexity. To keep this chapter to a
2-week time budget, we recommend a combination of options 1 and 2,
with separate strategies used for the stack and the heap.

Regarding the stack, we recommend using a separate stack for
pointers~\citep{Siebert:2001aa,Henderson:2002aa,Baker:2009aa}, which
we call a \emph{root stack} (a.k.a. ``shadow stack''). That is, when a
local variable needs to be spilled and is of type \code{(Vector
  $\Type_1 \ldots \Type_n$)}, then we put it on the root stack instead
of the normal procedure call stack. Furthermore, we always spill
vector-typed variables if they are live during a call to the
collector, thereby ensuring that no pointers are in registers during a
collection. Figure~\ref{fig:shadow-stack} reproduces the example from
Figure~\ref{fig:copying-collector} and contrasts it with the data
layout using a root stack. The root stack contains the two pointers
from the regular stack and also the pointer in the second
register.

\begin{figure}[tbp]
\centering \includegraphics[width=0.7\textwidth]{figs/root-stack}
\caption{Maintaining a root stack to facilitate garbage collection.}
\label{fig:shadow-stack}
\end{figure}

The problem of distinguishing between pointers and other kinds of data
also arises inside of each tuple. We solve this problem by attaching a
tag, an extra 64-bits, to each tuple. Figure~\ref{fig:tuple-rep} zooms
in on the tags for two of the tuples in the example from
Figure~\ref{fig:copying-collector}. Note that we have drawn the bits
in a big-endian way, from right-to-left, with bit location 0 (the
least significant bit) on the far right, which corresponds to the
directionality of the x86 shifting instructions \key{salq} (shift
left) and \key{sarq} (shift right). Part of each tag is dedicated to
specifying which elements of the tuple are pointers, the part labeled
``pointer mask''. Within the pointer mask, a 1 bit indicates there is
a pointer and a 0 bit indicates some other kind of data. The pointer
mask starts at bit location 7. We have limited tuples to a maximum
size of 50 elements, so we just need 50 bits for the pointer mask. The
tag also contains two other pieces of information. The length of the
tuple (number of elements) is stored in bits location 1 through
6. Finally, the bit at location 0 indicates whether the tuple has yet
to be copied to the ToSpace.  If the bit has value 1, then this tuple
has not yet been copied.  If the bit has value 0 then the entire tag
is in fact a forwarding pointer. (The lower 3 bits of an pointer are
always zero anyways because our tuples are 8-byte aligned.)

\begin{figure}[tbp]
\centering \includegraphics[width=0.8\textwidth]{figs/tuple-rep}
\caption{Representation for tuples in the heap.}
\label{fig:tuple-rep}
\end{figure}

\subsection{Implementation of the Garbage Collector}
\label{sec:organize-gz}

The implementation of the garbage collector needs to do a lot of
bit-level data manipulation and we need to link it with our
compiler-generated x86 code. Thus, we recommend implementing the
garbage collector in C~\citep{Kernighan:1988nx} and putting the code
in the \code{runtime.c} file. Figure~\ref{fig:gc-header} shows the
interface to the garbage collector. The \code{initialize} function
creates the FromSpace, ToSpace, and root stack. The \code{initialize}
function is meant to be called near the beginning of \code{main},
before the rest of the program executes.  The \code{initialize}
function puts the address of the beginning of the FromSpace into the
global variable \code{free\_ptr}. The global \code{fromspace\_end}
points to the address that is 1-past the last element of the
FromSpace. (We use half-open intervals to represent chunks of
memory~\citep{Dijkstra:1982aa}.)  The \code{rootstack\_begin} global
points to the first element of the root stack.

As long as there is room left in the FromSpace, your generated code
can allocate tuples simply by moving the \code{free\_ptr} forward.
%
\margincomment{\tiny Should we dedicate a register to the free pointer? \\
--Jeremy}
%
The amount of room left in FromSpace is the difference between the
\code{fromspace\_end} and the \code{free\_ptr}.  The \code{collect}
function should be called when there is not enough room left in the
FromSpace for the next allocation.  The \code{collect} function takes
a pointer to the current top of the root stack (one past the last item
that was pushed) and the number of bytes that need to be
allocated. The \code{collect} function performs the copying collection
and leaves the heap in a state such that the next allocation will
succeed.

\begin{figure}[tbp]
\begin{lstlisting}
   void initialize(uint64_t rootstack_size, uint64_t heap_size);
   void collect(int64_t** rootstack_ptr, uint64_t bytes_requested);
   int64_t* free_ptr;
   int64_t* fromspace_begin;
   int64_t* fromspace_end;
   int64_t** rootstack_begin;
\end{lstlisting}
\caption{The compiler's interface to the garbage collector.}
\label{fig:gc-header}
\end{figure}

\begin{exercise}
  In the file \code{runtime.c} you will find the implementation of
  \code{initialize} and a partial implementation of \code{collect}.
  The \code{collect} function calls another function, \code{cheney},
  to perform the actual copy, and that function is left to the reader
  to implement. The following is the prototype for \code{cheney}.
\begin{lstlisting}
   static void cheney(int64_t** rootstack_ptr);
\end{lstlisting}
  The parameter \code{rootstack\_ptr} is a pointer to the top of the
  rootstack (which is an array of pointers).  The \code{cheney} function
  also communicates with \code{collect} through the global
  variables \code{fromspace\_begin} and \code{fromspace\_end}
  mentioned in Figure~\ref{fig:gc-header} as well as the pointers for
  the ToSpace:
\begin{lstlisting}
   static int64_t* tospace_begin;
   static int64_t* tospace_end;
\end{lstlisting}
  The job of the \code{cheney} function is to copy all the live
  objects (reachable from the root stack) into the ToSpace, update
  \code{free\_ptr} to point to the next unused spot in the ToSpace,
  update the root stack so that it points to the objects in the
  ToSpace, and finally to swap the global pointers for the FromSpace
  and ToSpace.
\end{exercise}


%% \section{Compiler Passes}
%% \label{sec:code-generation-gc}

The introduction of garbage collection has a non-trivial impact on our
compiler passes. We introduce one new compiler pass called
\code{expose-allocation} and make non-trivial changes to
\code{type-check}, \code{flatten}, \code{select-instructions},
\code{allocate-registers}, and \code{print-x86}.  The following
program will serve as our running example.  It creates two tuples, one
nested inside the other. Both tuples have length one. The example then
accesses the element in the inner tuple tuple via two vector
references.
% tests/s2_17.rkt
\begin{lstlisting}
    (vector-ref (vector-ref (vector (vector 42)) 0) 0))
\end{lstlisting}

Next we proceed to discuss the new \code{expose-allocation} pass.

\section{Expose Allocation}
\label{sec:expose-allocation}

The pass \code{expose-allocation} lowers the \code{vector} creation
form into a conditional call to the collector followed by the
allocation. We choose to place the \code{expose-allocation} pass
before \code{flatten} because \code{expose-allocation} introduces new
variables, which can be done locally with \code{let}, but \code{let}
is gone after \code{flatten}.  In the following, we show the
transformation for the \code{vector} form into let-bindings for the
intializing expressions, by a conditional \code{collect}, an
\code{allocate}, and the initialization of the vector.
(The \itm{len} is the length of the vector and \itm{bytes} is how many
total bytes need to be allocated for the vector, which is 8 for the
tag plus \itm{len} times 8.)

\begin{lstlisting}
  (has-type (vector |$e_0 \ldots e_{n-1}$|) |\itm{type}|)
|$\Longrightarrow$|
  (let ([|$x_0$| |$e_0$|]) ... (let ([|$x_{n-1}$| |$e_{n-1}$|])
  (let ([_ (if (< (+ (global-value free_ptr) |\itm{bytes}|)
                  (global-value fromspace_end))
               (void)
               (collect |\itm{bytes}|))])
  (let ([|$v$| (allocate |\itm{len}| |\itm{type}|)])
  (let ([_ (vector-set! |$v$| |$0$| |$x_0$|)]) ...
  (let ([_ (vector-set! |$v$| |$n-1$| |$x_{n-1}$|)])
     |$v$|) ... )))) ...)
\end{lstlisting}
(In the above, we suppressed all of the \code{has-type} forms in the
output for the sake of readability.)  The placement of the initializing
expressions $e_0,\ldots,e_{n-1}$ prior to the \code{allocate} and
the sequence of \code{vector-set!}'s is important, as those expressions
may trigger garbage collection and we do not want an allocated but
uninitialized tuple to be present during a garbage collection.

The output of \code{expose-allocation} is a language that extends
$R_3$ with the three new forms that we use above in the translation of
\code{vector}.
\[
\begin{array}{lcl}
  \Exp &::=& \cdots
      \mid (\key{collect} \,\itm{int})
      \mid (\key{allocate} \,\itm{int}\,\itm{type})
      \mid (\key{global-value} \,\itm{name})
\end{array}
\]

%% The \code{expose-allocation} inserts an \code{initialize} statement at
%% the beginning of the program which will instruct the garbage collector
%% to set up the FromSpace, ToSpace, and all the global variables.  The
%% two arguments of \code{initialize} specify the initial allocated space
%% for the root stack and for the heap.
%
%% The \code{expose-allocation} pass annotates all of the local variables
%% in the \code{program} form with their type.


Figure~\ref{fig:expose-alloc-output} shows the output of the
\code{expose-allocation} pass on our running example.

\begin{figure}[tbp]
\begin{lstlisting}
(program ()
 (vector-ref
  (vector-ref
   (let ((vecinit48
          (let ((vecinit44 42))
            (let ((collectret46
                   (if (<
                        (+ (global-value free_ptr) 16)
                        (global-value fromspace_end))
                     (void)
                     (collect 16))))
              (let ((alloc43 (allocate 1 (Vector Integer))))
                (let ((initret45 (vector-set! alloc43 0 vecinit44)))
                  alloc43))))))
     (let ((collectret50
            (if (< (+ (global-value free_ptr) 16)
                   (global-value fromspace_end))
              (void)
              (collect 16))))
       (let ((alloc47 (allocate 1 (Vector (Vector Integer)))))
         (let ((initret49 (vector-set! alloc47 0 vecinit48)))
           alloc47))))
   0)
  0))
\end{lstlisting}
\caption{Output of the \code{expose-allocation} pass, minus
  all of the \code{has-type} forms.}
\label{fig:expose-alloc-output}
\end{figure}


%\clearpage

\section{Explicate Control and the $C_2$ language}
\label{sec:explicate-control-r3}

\begin{figure}[tp]
\fbox{
\begin{minipage}{0.96\textwidth}
\[
\begin{array}{lcl}
\Arg &::=& \gray{ \Int \mid \Var \mid \key{\#t} \mid \key{\#f} }\\
\itm{cmp} &::= & \gray{  \key{eq?} \mid \key{<} } \\
\Exp &::= & \gray{ \Arg \mid (\key{read}) \mid (\key{-}\;\Arg) \mid (\key{+} \; \Arg\;\Arg)
      \mid (\key{not}\;\Arg) \mid (\itm{cmp}\;\Arg\;\Arg)  } \\
   &\mid& (\key{allocate} \,\itm{int}\,\itm{type})
   \mid (\key{vector-ref}\, \Arg\, \Int)  \\
   &\mid& (\key{vector-set!}\,\Arg\,\Int\,\Arg)
    \mid (\key{global-value} \,\itm{name}) \mid (\key{void}) \\
\Stmt &::=& \gray{ \ASSIGN{\Var}{\Exp} \mid \RETURN{\Exp} } 
       \mid (\key{collect} \,\itm{int}) \\
\Tail &::= & \gray{\RETURN{\Exp} \mid (\key{seq}\;\Stmt\;\Tail)} \\
      &\mid& \gray{(\key{goto}\,\itm{label})
       \mid \IF{(\itm{cmp}\, \Arg\,\Arg)}{(\key{goto}\,\itm{label})}{(\key{goto}\,\itm{label})}} \\
C_2 & ::= & (\key{program}\;\itm{info}\; ((\itm{label}\,\key{.}\,\Tail)^{+}))
\end{array}
\]
\end{minipage}
}
\caption{The $C_2$ language, extending $C_1$
  (Figure~\ref{fig:c1-syntax}) with vectors.}
\label{fig:c2-syntax}
\end{figure}

The output of \code{explicate-control} is a program in the
intermediate language $C_2$, whose syntax is defined in
Figure~\ref{fig:c2-syntax}.  The new forms of $C_2$ include the
\key{allocate}, \key{vector-ref}, and \key{vector-set!}, and
\key{global-value} expressions and the \code{collect} statement.  The
\code{explicate-control} pass can treat these new forms much like the
other forms.


\section{Uncover Locals}
\label{sec:uncover-locals-r3}

Recall that the \code{uncover-locals} function collects all of the
local variables so that it can store them in the $\itm{info}$ field of
the \code{program} form. Also recall that we need to know the types of
all the local variables for purposes of identifying the root set for
the garbage collector.  Thus, we change \code{uncover-locals} to
collect not just the variables, but the variables and their types in
the form of an association list. Thanks to the \code{has-type} forms,
the types are readily available. Figure~\ref{fig:uncover-locals-r3}
lists the output of the \code{uncover-locals} pass on the running
example.

\begin{figure}[tbp]
\begin{lstlisting}
(program
 ((locals . ((tmp54 . Integer) (tmp51 . Integer) (tmp53 . Integer)
             (alloc43 . (Vector Integer)) (tmp55 . Integer)
             (initret45 . Void) (alloc47 . (Vector (Vector Integer)))
             (collectret46 . Void) (vecinit48 . (Vector Integer))
             (tmp52 . Integer) (tmp57 . (Vector Integer))
             (vecinit44 . Integer) (tmp56 . Integer) (initret49 . Void)
             (collectret50 . Void))))
 ((block63 . (seq (collect 16) (goto block61)))
  (block62 . (seq (assign collectret46 (void)) (goto block61)))
  (block61 . (seq (assign alloc43 (allocate 1 (Vector Integer)))
               (seq (assign initret45 (vector-set! alloc43 0 vecinit44))
               (seq (assign vecinit48 alloc43)
               (seq (assign tmp54 (global-value free_ptr))
               (seq (assign tmp55 (+ tmp54 16))
               (seq (assign tmp56 (global-value fromspace_end))
               (if (< tmp55 tmp56) (goto block59) (goto block60)))))))))
  (block60 . (seq (collect 16) (goto block58)))
  (block59 . (seq (assign collectret50 (void)) (goto block58)))
  (block58 . (seq (assign alloc47 (allocate 1 (Vector (Vector Integer))))
               (seq (assign initret49 (vector-set! alloc47 0 vecinit48))
               (seq (assign tmp57 (vector-ref alloc47 0))
               (return (vector-ref tmp57 0))))))
  (start . (seq (assign vecinit44 42)
           (seq (assign tmp51 (global-value free_ptr))
           (seq (assign tmp52 (+ tmp51 16))
           (seq (assign tmp53 (global-value fromspace_end))
           (if (< tmp52 tmp53) (goto block62) (goto block63)))))))))
\end{lstlisting}
\caption{Output of \code{uncover-locals} for the running example.}
\label{fig:uncover-locals-r3}
\end{figure}

\clearpage

\section{Select Instructions}
\label{sec:select-instructions-gc}

%% void (rep as zero)
%% allocate
%% collect (callq collect)
%% vector-ref
%% vector-set!
%% global-value (postpone)

In this pass we generate x86 code for most of the new operations that
were needed to compile tuples, including \code{allocate},
\code{collect}, \code{vector-ref}, \code{vector-set!}, and
\code{(void)}. We postpone \code{global-value} to \code{print-x86}.

The \code{vector-ref} and \code{vector-set!} forms translate into
\code{movq} instructions with the appropriate \key{deref}.  (The
plus one is to get past the tag at the beginning of the tuple
representation.)
\begin{lstlisting}
   (assign |$\itm{lhs}$| (vector-ref |$\itm{vec}$| |$n$|))
   |$\Longrightarrow$|
   (movq |$\itm{vec}'$| (reg r11))
   (movq (deref r11 |$8(n+1)$|) |$\itm{lhs}$|)

   (assign |$\itm{lhs}$| (vector-set! |$\itm{vec}$| |$n$| |$\itm{arg}$|))
   |$\Longrightarrow$|
   (movq |$\itm{vec}'$| (reg r11))
   (movq |$\itm{arg}'$| (deref r11 |$8(n+1)$|))
   (movq (int 0) |$\itm{lhs}$|)
\end{lstlisting}
The $\itm{vec}'$ and $\itm{arg}'$ are obtained by recursively
processing $\itm{vec}$ and $\itm{arg}$.  The move of $\itm{vec}'$ to
register \code{r11} ensures that offsets are only performed with
register operands. This requires removing \code{r11} from
consideration by the register allocating.

We compile the \code{allocate} form to operations on the
\code{free\_ptr}, as shown below. The address in the \code{free\_ptr}
is the next free address in the FromSpace, so we move it into the
\itm{lhs} and then move it forward by enough space for the tuple being
allocated, which is $8(\itm{len}+1)$ bytes because each element is 8
bytes (64 bits) and we use 8 bytes for the tag. Last but not least, we
initialize the \itm{tag}. Refer to Figure~\ref{fig:tuple-rep} to see
how the tag is organized. We recommend using the Racket operations
\code{bitwise-ior} and \code{arithmetic-shift} to compute the tag.
The type annoation in the \code{vector} form is used to determine the
pointer mask region of the tag.
\begin{lstlisting}
   (assign |$\itm{lhs}$| (allocate |$\itm{len}$| (Vector |$\itm{type} \ldots$|)))
   |$\Longrightarrow$|
   (movq (global-value free_ptr) |$\itm{lhs}'$|)
   (addq (int |$8(\itm{len}+1)$|) (global-value free_ptr))
   (movq |$\itm{lhs}'$| (reg r11))
   (movq (int |$\itm{tag}$|) (deref r11 0))
\end{lstlisting}

The \code{collect} form is compiled to a call to the \code{collect}
function in the runtime. The arguments to \code{collect} are the top
of the root stack and the number of bytes that need to be allocated.
We shall use a dedicated register, \code{r15}, to store the pointer to
the top of the root stack. So \code{r15} is not available for use by
the register allocator.
\begin{lstlisting}
   (collect |$\itm{bytes}$|)
   |$\Longrightarrow$|
   (movq (reg r15) (reg rdi))
   (movq |\itm{bytes}| (reg rsi))
   (callq collect)
\end{lstlisting}


\begin{figure}[tp]
\fbox{
\begin{minipage}{0.96\textwidth}
\[
\begin{array}{lcl}
\Arg &::=&  \gray{  \INT{\Int} \mid \REG{\itm{register}}
    \mid (\key{deref}\,\itm{register}\,\Int) } \\
   &\mid& \gray{ (\key{byte-reg}\; \itm{register})  }
   \mid (\key{global-value}\; \itm{name}) \\
\itm{cc} & ::= & \gray{  \key{e} \mid \key{l} \mid \key{le} \mid \key{g} \mid \key{ge}  } \\
\Instr &::=& \gray{(\key{addq} \; \Arg\; \Arg) \mid
             (\key{subq} \; \Arg\; \Arg) \mid
             (\key{negq} \; \Arg) \mid (\key{movq} \; \Arg\; \Arg)} \\
      &\mid& \gray{(\key{callq} \; \mathit{label}) \mid
             (\key{pushq}\;\Arg) \mid
             (\key{popq}\;\Arg) \mid
             (\key{retq})} \\
       &\mid& \gray{  (\key{xorq} \; \Arg\;\Arg)
       \mid (\key{cmpq} \; \Arg\; \Arg) \mid (\key{set}\itm{cc} \; \Arg)  } \\
       &\mid& \gray{  (\key{movzbq}\;\Arg\;\Arg)
       \mid  (\key{jmp} \; \itm{label})
       \mid (\key{jmp-if}\itm{cc} \; \itm{label})}\\
       &\mid& \gray{(\key{label} \; \itm{label})  } \\
x86_2 &::= & \gray{  (\key{program} \;\itm{info} \;(\key{type}\;\itm{type})\; \Instr^{+})  }
\end{array}
\]
\end{minipage}
}
\caption{The x86$_2$ language (extends x86$_1$ of Figure~\ref{fig:x86-1}).}
\label{fig:x86-2}
\end{figure}

The syntax of the $x86_2$ language is defined in
Figure~\ref{fig:x86-2}.  It differs from $x86_1$ just in the addition
of the form for global variables.
%
Figure~\ref{fig:select-instr-output-gc} shows the output of the
\code{select-instructions} pass on the running example.

\begin{figure}[tbp]
\centering
\begin{minipage}{0.75\textwidth}
\begin{lstlisting}[basicstyle=\ttfamily\scriptsize]
(program
 ((locals . ((tmp54 . Integer) (tmp51 . Integer) (tmp53 . Integer)
             (alloc43 . (Vector Integer)) (tmp55 . Integer)
             (initret45 . Void) (alloc47 . (Vector (Vector Integer)))
             (collectret46 . Void) (vecinit48 . (Vector Integer))
             (tmp52 . Integer) (tmp57 Vector Integer) (vecinit44 . Integer)
             (tmp56 . Integer) (initret49 . Void) (collectret50 . Void))))
 ((block63 . (block ()
               (movq (reg r15) (reg rdi))
               (movq (int 16) (reg rsi))
               (callq collect)
               (jmp block61)))
  (block62 . (block () (movq (int 0) (var collectret46)) (jmp block61)))
  (block61 . (block ()
               (movq (global-value free_ptr) (var alloc43))
               (addq (int 16) (global-value free_ptr))
               (movq (var alloc43) (reg r11))
               (movq (int 3) (deref r11 0))
               (movq (var alloc43) (reg r11))
               (movq (var vecinit44) (deref r11 8))
               (movq (int 0) (var initret45))
               (movq (var alloc43) (var vecinit48))
               (movq (global-value free_ptr) (var tmp54))
               (movq (var tmp54) (var tmp55))
               (addq (int 16) (var tmp55))
               (movq (global-value fromspace_end) (var tmp56))
               (cmpq (var tmp56) (var tmp55))
               (jmp-if l block59)
               (jmp block60)))
  (block60 . (block ()
               (movq (reg r15) (reg rdi))
               (movq (int 16) (reg rsi))
               (callq collect)
               (jmp block58))
  (block59 . (block () 
               (movq (int 0) (var collectret50)) 
               (jmp block58)))
  (block58 . (block ()
               (movq (global-value free_ptr) (var alloc47))
               (addq (int 16) (global-value free_ptr))
               (movq (var alloc47) (reg r11))
               (movq (int 131) (deref r11 0))
               (movq (var alloc47) (reg r11))
               (movq (var vecinit48) (deref r11 8))
               (movq (int 0) (var initret49))
               (movq (var alloc47) (reg r11))
               (movq (deref r11 8) (var tmp57))
               (movq (var tmp57) (reg r11))
               (movq (deref r11 8) (reg rax))
               (jmp conclusion)))
  (start . (block ()
             (movq (int 42) (var vecinit44))
             (movq (global-value free_ptr) (var tmp51))
             (movq (var tmp51) (var tmp52))
             (addq (int 16) (var tmp52))
             (movq (global-value fromspace_end) (var tmp53))
             (cmpq (var tmp53) (var tmp52))
             (jmp-if l block62)
             (jmp block63))))))
\end{lstlisting}
\end{minipage}
\caption{Output of the \code{select-instructions} pass.}
\label{fig:select-instr-output-gc}
\end{figure}

\clearpage

\section{Register Allocation}
\label{sec:reg-alloc-gc}

As discussed earlier in this chapter, the garbage collector needs to
access all the pointers in the root set, that is, all variables that
are vectors. It will be the responsibility of the register allocator
to make sure that:
\begin{enumerate}
\item the root stack is used for spilling vector-typed variables, and
\item if a vector-typed variable is live during a call to the
  collector, it must be spilled to ensure it is visible to the
  collector.
\end{enumerate}

The later responsibility can be handled during construction of the
inference graph, by adding interference edges between the call-live
vector-typed variables and all the callee-saved registers. (They
already interfere with the caller-saved registers.)  The type
information for variables is in the \code{program} form, so we
recommend adding another parameter to the \code{build-interference}
function to communicate this association list.

The spilling of vector-typed variables to the root stack can be
handled after graph coloring, when choosing how to assign the colors
(integers) to registers and stack locations. The \code{program} output
of this pass changes to also record the number of spills to the root
stack.

% build-interference
%
% callq
%   extra parameter for var->type assoc. list
% update 'program' and 'if'

% allocate-registers
%    allocate spilled vectors to the rootstack

% don't change color-graph



\section{Print x86}
\label{sec:print-x86-gc}


\margincomment{\scriptsize We need to show the translation to x86 and what
  to do about global-value. \\ --Jeremy}

Figure~\ref{fig:print-x86-output-gc} shows the output of the
\code{print-x86} pass on the running example. In the prelude and
conclusion of the \code{main} function, we treat the root stack very
much like the regular stack in that we move the root stack pointer
(\code{r15}) to make room for all of the spills to the root stack,
except that the root stack grows up instead of down.  For the running
example, there was just one spill so we increment \code{r15} by 8
bytes. In the conclusion we decrement \code{r15} by 8 bytes.

One issue that deserves special care is that there may be a call to
\code{collect} prior to the initializing assignments for all the
variables in the root stack. We do not want the garbage collector to
accidentaly think that some uninitialized variable is a pointer that
needs to be followed. Thus, we zero-out all locations on the root
stack in the prelude of \code{main}. In
Figure~\ref{fig:print-x86-output-gc}, the instruction
%
\lstinline{movq $0, (%r15)}
%
accomplishes this task. The garbage collector tests each root to see
if it is null prior to dereferencing it.

\begin{figure}[htbp]
\begin{minipage}[t]{0.5\textwidth}
\begin{lstlisting}[basicstyle=\ttfamily\scriptsize]
_block58:
	movq	_free_ptr(%rip), %rcx
	addq	$16, _free_ptr(%rip)
	movq	%rcx, %r11
	movq	$131, 0(%r11)
	movq	%rcx, %r11
	movq	-8(%r15), %rax
	movq	%rax, 8(%r11)
	movq	$0, %rdx
	movq	%rcx, %r11
	movq	8(%r11), %rcx
	movq	%rcx, %r11
	movq	8(%r11), %rax
	jmp _conclusion
_block59:
	movq	$0, %rcx
	jmp _block58
_block62:
	movq	$0, %rcx
	jmp _block61
_block60:
	movq	%r15, %rdi
	movq	$16, %rsi
	callq	_collect
	jmp _block58
_block63:
	movq	%r15, %rdi
	movq	$16, %rsi
	callq	_collect
	jmp _block61
_start:
	movq	$42, %rbx
	movq	_free_ptr(%rip), %rdx
	addq	$16, %rdx
	movq	_fromspace_end(%rip), %rcx
	cmpq	%rcx, %rdx
	jl _block62
	jmp _block63
\end{lstlisting}
\end{minipage}
\begin{minipage}[t]{0.45\textwidth}
\begin{lstlisting}[basicstyle=\ttfamily\scriptsize]
_block61:
	movq	_free_ptr(%rip), %rcx
	addq	$16, _free_ptr(%rip)
	movq	%rcx, %r11
	movq	$3, 0(%r11)
	movq	%rcx, %r11
	movq	%rbx, 8(%r11)
	movq	$0, %rdx
	movq	%rcx, -8(%r15)
	movq	_free_ptr(%rip), %rcx
	addq	$16, %rcx
	movq	_fromspace_end(%rip), %rdx
	cmpq	%rdx, %rcx
	jl _block59
	jmp _block60

	.globl _main
_main:
	pushq	%rbp
	movq	%rsp, %rbp
	pushq	%r12
	pushq	%rbx
	pushq	%r13
	pushq	%r14
	subq	$0, %rsp
	movq $16384, %rdi
	movq $16, %rsi
	callq _initialize
	movq _rootstack_begin(%rip), %r15
	movq $0, (%r15)
	addq $8, %r15
	jmp _start
_conclusion:
	subq $8, %r15
	addq	$0, %rsp
	popq	%r14
	popq	%r13
	popq	%rbx
	popq	%r12
	popq	%rbp
	retq
\end{lstlisting}
\end{minipage}
\caption{Output of the \code{print-x86} pass.}
\label{fig:print-x86-output-gc}
\end{figure}


\margincomment{\scriptsize Suggest an implementation strategy
  in which the students first do the code gen and test that
  without GC (just use a big heap), then after that is debugged,
  implement the GC. \\ --Jeremy}


\begin{figure}[p]
\begin{tikzpicture}[baseline=(current  bounding  box.center)]
\node (R3) at (0,2)  {\large $R_3$};
\node (R3-2) at (3,2)  {\large $R_3$};
\node (R3-3) at (6,2)  {\large $R_3$};
\node (R3-4) at (9,2)  {\large $R_3$};
\node (R3-5) at (12,2)  {\large $R_3$};
\node (C2-4) at (3,0)  {\large $C_2$};
\node (C2-3) at (6,0)  {\large $C_2$};

\node (x86-2) at (3,-2)  {\large $\text{x86}^{*}_2$};
\node (x86-3) at (6,-2)  {\large $\text{x86}^{*}_2$};
\node (x86-4) at (9,-2) {\large $\text{x86}^{*}_2$};
\node (x86-5) at (9,-4) {\large $\text{x86}^{\dagger}_2$};

\node (x86-2-1) at (3,-4)  {\large $\text{x86}^{*}_2$};
\node (x86-2-2) at (6,-4)  {\large $\text{x86}^{*}_2$};

\path[->,bend left=15] (R3) edge [above] node {\ttfamily\footnotesize\color{red} typecheck} (R3-2);
\path[->,bend left=15] (R3-2) edge [above] node {\ttfamily\footnotesize uniquify} (R3-3);
\path[->,bend left=15] (R3-3) edge [above] node {\ttfamily\footnotesize\color{red} expose-alloc.} (R3-4);
\path[->,bend left=15] (R3-4) edge [above] node {\ttfamily\footnotesize remove-complex.} (R3-5);
\path[->,bend left=20] (R3-5) edge [right] node {\ttfamily\footnotesize explicate-control} (C2-3);
\path[->,bend right=15] (C2-3) edge [above] node {\ttfamily\footnotesize\color{red} uncover-locals} (C2-4);
\path[->,bend right=15] (C2-4) edge [left] node {\ttfamily\footnotesize\color{red} select-instr.} (x86-2);
\path[->,bend left=15] (x86-2) edge [right] node {\ttfamily\footnotesize uncover-live} (x86-2-1);
\path[->,bend right=15] (x86-2-1) edge [below] node {\ttfamily\footnotesize \color{red}build-inter.} (x86-2-2);
\path[->,bend right=15] (x86-2-2) edge [right] node {\ttfamily\footnotesize allocate-reg.} (x86-3);
\path[->,bend left=15] (x86-3) edge [above] node {\ttfamily\footnotesize patch-instr.} (x86-4);
\path[->,bend left=15] (x86-4) edge [right] node {\ttfamily\footnotesize\color{red} print-x86} (x86-5);
\end{tikzpicture}
\caption{Diagram of the passes for $R_3$, a language with tuples.}
\label{fig:R3-passes}
\end{figure}

Figure~\ref{fig:R3-passes} gives an overview of all the passes needed
for the compilation of $R_3$.


%%%%%%%%%%%%%%%%%%%%%%%%%%%%%%%%%%%%%%%%%%%%%%%%%%%%%%%%%%%%%%%%%%%%%%%%%%%%%%%%
\chapter{Functions}
\label{ch:functions}

This chapter studies the compilation of functions at the level of
abstraction of the C language. This corresponds to a subset of Typed
Racket in which only top-level function definitions are allowed. These
kind of functions are an important stepping stone to implementing
lexically-scoped functions in the form of \key{lambda} abstractions,
which is the topic of Chapter~\ref{ch:lambdas}.

\section{The $R_4$ Language}

The syntax for function definitions and function application is shown
in Figure~\ref{fig:r4-syntax}, where we define the $R_4$ language.
Programs in $R_4$ start with zero or more function definitions.  The
function names from these definitions are in-scope for the entire
program, including all other function definitions (so the ordering of
function definitions does not matter). The syntax for function
application does not include an explicit keyword, which is error prone
when using \code{match}. To alleviate this problem, we change the
syntax from $(\Exp \; \Exp^{*})$ to $(\key{app}\; \Exp \; \Exp^{*})$
during type checking.

Functions are first-class in the sense that a function pointer is data
and can be stored in memory or passed as a parameter to another
function.  Thus, we introduce a function type, written
\begin{lstlisting}
   (|$\Type_1$| |$\cdots$| |$\Type_n$| -> |$\Type_r$|)
\end{lstlisting}
for a function whose $n$ parameters have the types $\Type_1$ through
$\Type_n$ and whose return type is $\Type_r$. The main limitation of
these functions (with respect to Racket functions) is that they are
not lexically scoped. That is, the only external entities that can be
referenced from inside a function body are other globally-defined
functions. The syntax of $R_4$ prevents functions from being nested
inside each other.

\begin{figure}[tp]
\centering
\fbox{
\begin{minipage}{0.96\textwidth}
\[
\begin{array}{lcl}
  \Type &::=& \gray{ \key{Integer} \mid \key{Boolean}
         \mid (\key{Vector}\;\Type^{+}) \mid \key{Void}  } \mid (\Type^{*} \; \key{->}\; \Type) \\
\itm{cmp} &::= & \gray{  \key{eq?} \mid \key{<} \mid \key{<=} \mid \key{>} \mid \key{>=}  } \\
  \Exp &::=& \gray{ \Int \mid (\key{read}) \mid (\key{-}\;\Exp) \mid (\key{+} \; \Exp\;\Exp) \mid (\key{-}\;\Exp\;\Exp)}  \\
    &\mid&  \gray{ \Var \mid \LET{\Var}{\Exp}{\Exp} }\\
    &\mid& \gray{ \key{\#t} \mid \key{\#f} 
    \mid (\key{and}\;\Exp\;\Exp)
    \mid (\key{or}\;\Exp\;\Exp)
    \mid (\key{not}\;\Exp)} \\
   &\mid& \gray{(\itm{cmp}\;\Exp\;\Exp) \mid \IF{\Exp}{\Exp}{\Exp}} \\
  &\mid& \gray{(\key{vector}\;\Exp^{+}) \mid
    (\key{vector-ref}\;\Exp\;\Int)} \\
  &\mid& \gray{(\key{vector-set!}\;\Exp\;\Int\;\Exp)\mid (\key{void})} \\
      &\mid& (\Exp \; \Exp^{*}) \\
  \Def &::=& (\key{define}\; (\Var \; [\Var \key{:} \Type]^{*}) \key{:} \Type \; \Exp) \\
  R_4 &::=& (\key{program} \;\itm{info}\; \Def^{*} \; \Exp)
\end{array}
\]
\end{minipage}
}
\caption{Syntax of $R_4$, extending $R_3$ (Figure~\ref{fig:r3-syntax})
  with functions.}
\label{fig:r4-syntax}
\end{figure}

The program in Figure~\ref{fig:r4-function-example} is a
representative example of defining and using functions in $R_4$.  We
define a function \code{map-vec} that applies some other function
\code{f} to both elements of a vector (a 2-tuple) and returns a new
vector containing the results. We also define a function \code{add1}
that does what its name suggests. The program then applies
\code{map-vec} to \code{add1} and \code{(vector 0 41)}.  The result is
\code{(vector 1 42)}, from which we return the \code{42}.

\begin{figure}[tbp]
\begin{lstlisting}
(program ()
  (define (map-vec [f : (Integer -> Integer)]
                     [v : (Vector Integer Integer)])
          : (Vector Integer Integer)
    (vector (f (vector-ref v 0)) (f (vector-ref v 1))))
  (define (add1 [x : Integer]) : Integer
    (+ x 1))
  (vector-ref (map-vec add1 (vector 0 41)) 1)
  )
\end{lstlisting}
\caption{Example of using functions in $R_4$.}
\label{fig:r4-function-example}
\end{figure}

The definitional interpreter for $R_4$ is in
Figure~\ref{fig:interp-R4}. The case for the \code{program} form is
responsible for setting up the mutual recursion between the top-level
function definitions. We use the classic backpatching approach that
uses mutable variables and makes two passes over the function
definitions~\citep{Kelsey:1998di}.  In the first pass we set up the
top-level environment using a mutable cons cell for each function
definition. Note that the \code{lambda} value for each function is
incomplete; it does not yet include the environment.  Once the
top-level environment is constructed, we then iterate over it and
update the \code{lambda} value's to use the top-level environment.

\begin{figure}[tp]
\begin{lstlisting}
(define (interp-exp env)
  (lambda (e)
    (define recur (interp-exp env))
    (match e
      ...
      [`(,fun ,args ...)
       (define arg-vals (for/list ([e args]) (recur e)))
       (define fun-val (recur fun))
       (match fun-val
	 [`(lambda (,xs ...) ,body ,fun-env)
	  (define new-env (append (map cons xs arg-vals) fun-env))
	  ((interp-exp new-env) body)]
	 [else (error "interp-exp, expected function, not" fun-val)])]
      [else (error 'interp-exp "unrecognized expression")]
      )))

(define (interp-def d)
  (match d
    [`(define (,f [,xs : ,ps] ...) : ,rt ,body)
     (mcons f `(lambda ,xs ,body ()))]
    ))

(define (interp-R4 p)
  (match p
    [`(program ,ds ... ,body)
     (let ([top-level (for/list ([d ds]) (interp-def d))])
       (for/list ([b top-level])
         (set-mcdr! b (match (mcdr b)
                        [`(lambda ,xs ,body ())
                         `(lambda ,xs ,body ,top-level)])))
       ((interp-exp top-level) body))]
    ))
\end{lstlisting}
\caption{Interpreter for the $R_4$ language.}
\label{fig:interp-R4}
\end{figure}


\section{Functions in x86}
\label{sec:fun-x86}

\margincomment{\tiny Make sure callee-saved registers are discussed
   in enough depth, especially updating Fig 6.4 \\ --Jeremy }

\margincomment{\tiny Talk about the return address on the
   stack and what callq  and retq does.\\ --Jeremy }

The x86 architecture provides a few features to support the
implementation of functions. We have already seen that x86 provides
labels so that one can refer to the location of an instruction, as is
needed for jump instructions. Labels can also be used to mark the
beginning of the instructions for a function.  Going further, we can
obtain the address of a label by using the \key{leaq} instruction and
\key{rip}-relative addressing. For example, the following puts the
address of the \code{add1} label into the \code{rbx} register.
\begin{lstlisting}
   leaq add1(%rip), %rbx
\end{lstlisting}

In Section~\ref{sec:x86} we saw the use of the \code{callq}
instruction for jumping to a function whose location is given by a
label. Here we instead will be jumping to a function whose location is
given by an address, that is, we need to make an \emph{indirect
  function call}. The x86 syntax is to give the register name prefixed
with an asterisk.
\begin{lstlisting}
   callq *%rbx
\end{lstlisting}


\subsection{Calling Conventions}

The \code{callq} instruction provides partial support for implementing
functions, but it does not handle (1) parameter passing, (2) saving
and restoring frames on the procedure call stack, or (3) determining
how registers are shared by different functions. These issues require
coordination between the caller and the callee, which is often
assembly code written by different programmers or generated by
different compilers. As a result, people have developed
\emph{conventions} that govern how functions calls are performed.
Here we shall use the same conventions used by the \code{gcc}
compiler~\citep{Matz:2013aa}.

Regarding (1) parameter passing, the convention is to use the
following six registers: \code{rdi}, \code{rsi}, \code{rdx},
\code{rcx}, \code{r8}, and \code{r9}, in that order. If there are more
than six arguments, then the convention is to use space on the frame
of the caller for the rest of the arguments. However, to ease the
implementation of efficient tail calls (Section~\ref{sec:tail-call}),
we shall arrange to never have more than six arguments.
%
The register \code{rax} is for the return value of the function.

Regarding (2) frames and the procedure call stack, the convention is
that the stack grows down, with each function call using a chunk of
space called a frame. The caller sets the stack pointer, register
\code{rsp}, to the last data item in its frame. The callee must not
change anything in the caller's frame, that is, anything that is at or
above the stack pointer. The callee is free to use locations that are
below the stack pointer. 

Regarding (3) the sharing of registers between different functions,
recall from Section~\ref{sec:calling-conventions} that the registers
are divided into two groups, the caller-saved registers and the
callee-saved registers. The caller should assume that all the
caller-saved registers get overwritten with arbitrary values by the
callee. Thus, the caller should either 1) not put values that are live
across a call in caller-saved registers, or 2) save and restore values
that are live across calls. We shall recommend option 1).  On the flip
side, if the callee wants to use a callee-saved register, the callee
must save the contents of those registers on their stack frame and
then put them back prior to returning to the caller.  The base
pointer, register \code{rbp}, is used as a point-of-reference within a
frame, so that each local variable can be accessed at a fixed offset
from the base pointer.
%
Figure~\ref{fig:call-frames} shows the layout of the caller and callee
frames.
%% If we were to use stack arguments, they would be between the
%% caller locals and the callee return address. 



\begin{figure}[tbp]
\centering
\begin{tabular}{r|r|l|l} \hline
Caller View & Callee View & Contents       & Frame \\ \hline
8(\key{\%rbp})  & & return address & \multirow{5}{*}{Caller}\\
0(\key{\%rbp})  &  & old \key{rbp} \\
-8(\key{\%rbp}) &  & callee-saved $1$ \\
\ldots & & \ldots \\
$-8j$(\key{\%rbp}) &  & callee-saved $j$ \\
$-8(j+1)$(\key{\%rbp}) &  & local $1$ \\
\ldots & & \ldots \\
$-8(j+k)$(\key{\%rbp}) &  & local $k$ \\
 %% & &  \\
%% $8n-8$\key{(\%rsp)} & $8n+8$(\key{\%rbp})& argument $n$ \\
%% & \ldots           & \ldots \\
%% 0\key{(\%rsp)} & 16(\key{\%rbp})  & argument $1$   & \\
\hline
& 8(\key{\%rbp})   & return address & \multirow{5}{*}{Callee}\\
& 0(\key{\%rbp})   & old \key{rbp} \\
& -8(\key{\%rbp}) & callee-saved $1$ \\
& \ldots & \ldots \\
& $-8n$(\key{\%rbp})  & callee-saved $n$ \\
& $-8(n+1)$(\key{\%rbp})  & local $1$ \\
&  \ldots          & \ldots \\
& $-8(n+m)$(\key{\%rsp})   & local $m$\\ \hline
\end{tabular}
\caption{Memory layout of caller and callee frames.}
\label{fig:call-frames}
\end{figure}

%% Recall from Section~\ref{sec:x86} that the stack is also used for
%% local variables and for storing the values of callee-saved registers
%% (we shall refer to all of these collectively as ``locals''), and that
%% at the beginning of a function we move the stack pointer \code{rsp}
%% down to make room for them.
%% We recommend storing the local variables
%% first and then the callee-saved registers, so that the local variables
%% can be accessed using \code{rbp} the same as before the addition of
%% functions.
%% To make additional room for passing arguments, we shall
%% move the stack pointer even further down. We count how many stack
%% arguments are needed for each function call that occurs inside the
%% body of the function and find their maximum. Adding this number to the
%% number of locals gives us how much the \code{rsp} should be moved at
%% the beginning of the function. In preparation for a function call, we
%% offset from \code{rsp} to set up the stack arguments. We put the first
%% stack argument in \code{0(\%rsp)}, the second in \code{8(\%rsp)}, and
%% so on.

%% Upon calling the function, the stack arguments are retrieved by the
%% callee using the base pointer \code{rbp}. The address \code{16(\%rbp)}
%% is the location of the first stack argument, \code{24(\%rbp)} is the
%% address of the second, and so on. Figure~\ref{fig:call-frames} shows
%% the layout of the caller and callee frames. Notice how important it is
%% that we correctly compute the maximum number of arguments needed for
%% function calls; if that number is too small then the arguments and
%% local variables will smash into each other!

\subsection{Efficient Tail Calls}
\label{sec:tail-call}

In general, the amount of stack space used by a program is determined
by the longest chain of nested function calls. That is, if function
$f_1$ calls $f_2$, $f_2$ calls $f_3$, $\ldots$, and $f_{n-1}$ calls
$f_n$, then the amount of stack space is bounded by $O(n)$.  The depth
$n$ can grow quite large in the case of recursive or mutually
recursive functions. However, in some cases we can arrange to use only
constant space, i.e. $O(1)$, instead of $O(n)$.

If a function call is the last action in a function body, then that
call is said to be a \emph{tail call}. In such situations, the frame
of the caller is no longer needed, so we can pop the caller's frame
before making the tail call. With this approach, a recursive function
that only makes tail calls will only use $O(1)$ stack space.
Functional languages like Racket typically rely heavily on recursive
functions, so they typically guarantee that all tail calls will be
optimized in this way.

However, some care is needed with regards to argument passing in tail
calls.  As mentioned above, for arguments beyond the sixth, the
convention is to use space in the caller's frame for passing
arguments.  But here we've popped the caller's frame and can no longer
use it.  Another alternative is to use space in the callee's frame for
passing arguments. However, this option is also problematic because
the caller and callee's frame overlap in memory.  As we begin to copy
the arguments from their sources in the caller's frame, the target
locations in the callee's frame might overlap with the sources for
later arguments! We solve this problem by not using the stack for
parameter passing but instead use the heap, as we describe in the
Section~\ref{sec:limit-functions-r4}.

As mentioned above, for a tail call we pop the caller's frame prior to
making the tail call. The instructions for popping a frame are the
instructions that we usually place in the conclusion of a
function. Thus, we also need to place such code immediately before
each tail call. These instructions include restoring the callee-saved
registers, so it is good that the argument passing registers are all
caller-saved registers.

One last note regarding which instruction to use to make the tail
call. When the callee is finished, it should not return to the current
function, but it should return to the function that called the current
one. Thus, the return address that is already on the stack is the
right one, and we should not use \key{callq} to make the tail call, as
that would unnecessarily overwrite the return address. Instead we can
simply use the \key{jmp} instruction. Like the indirect function call,
we write an indirect jump with a register prefixed with an asterisk.
We recommend using \code{rax} to hold the jump target because the
preceeding ``conclusion'' overwrites just about everything else.
\begin{lstlisting}
   jmp *%rax
\end{lstlisting}

%% Now that we have a good understanding of functions as they appear in
%% $R_4$ and the support for functions in x86, we need to plan the
%% changes to our compiler, that is, do we need any new passes and/or do
%% we need to change any existing passes? Also, do we need to add new
%% kinds of AST nodes to any of the intermediate languages?

\section{Shrink $R_4$}
\label{sec:shrink-r4}

The \code{shrink} pass performs a couple minor modifications to the
grammar to ease the later passes. This pass adds an empty $\itm{info}$
field to each function definition:
\begin{lstlisting}
    (define (|$f$| [|$x_1 : \Type_1$| ...) : |$\Type_r$| |$\Exp$|)
|$\Rightarrow$|  (define (|$f$| [|$x_1 : \Type_1$| ...) : |$\Type_r$| () |$\Exp$|)
\end{lstlisting}
and introduces an explicit \code{main} function.\\
\begin{tabular}{lll}
\begin{minipage}{0.45\textwidth}
\begin{lstlisting}
    (program |$\itm{info}$| |$ds$| ... |$\Exp$|)
\end{lstlisting}
\end{minipage}
&
$\Rightarrow$
&
\begin{minipage}{0.45\textwidth}
\begin{lstlisting}
    (program |$\itm{info}$| |$ds'$| |$\itm{mainDef}$|)
\end{lstlisting}
\end{minipage}
\end{tabular}  \\
where $\itm{mainDef}$ is
\begin{lstlisting}
    (define (main) : Integer () |$\Exp'$|)
\end{lstlisting}


\section{Reveal Functions}
\label{sec:reveal-functions-r4}

Going forward, the syntax of $R_4$ is inconvenient for purposes of
compilation because it conflates the use of function names and local
variables. This is a problem because we need to compile the use of a
function name differently than the use of a local variable; we need to
use \code{leaq} to convert the function name (a label in x86) to an
address in a register.  Thus, it is a good idea to create a new pass
that changes function references from just a symbol $f$ to
\code{(fun-ref $f$)}. A good name for this pass is
\code{reveal-functions} and the output language, $F_1$, is defined in
Figure~\ref{fig:f1-syntax}.

\begin{figure}[tp]
\centering
\fbox{
\begin{minipage}{0.96\textwidth}
\[
\begin{array}{lcl}
  \Type &::=& \gray{ \key{Integer} \mid \key{Boolean}
         \mid (\key{Vector}\;\Type^{+}) \mid \key{Void}  \mid (\Type^{*} \; \key{->}\; \Type)} \\
  \Exp &::=& \gray{ \Int \mid (\key{read}) \mid (\key{-}\;\Exp) \mid (\key{+} \; \Exp\;\Exp)}  \\
     &\mid&  \gray{ \Var \mid \LET{\Var}{\Exp}{\Exp} }\\
  &\mid& \gray{ \key{\#t} \mid \key{\#f} \mid
      (\key{not}\;\Exp)} \mid \gray{(\itm{cmp}\;\Exp\;\Exp) \mid \IF{\Exp}{\Exp}{\Exp}} \\
  &\mid& \gray{(\key{vector}\;\Exp^{+}) \mid
    (\key{vector-ref}\;\Exp\;\Int)} \\
  &\mid& \gray{(\key{vector-set!}\;\Exp\;\Int\;\Exp)\mid (\key{void}) \mid
         (\key{app}\; \Exp \; \Exp^{*})} \\
      &\mid& (\key{fun-ref}\, \itm{label}) \\
  \Def &::=& \gray{(\key{define}\; (\itm{label} \; [\Var \key{:} \Type]^{*}) \key{:} \Type \; \Exp)} \\
  F_1 &::=& \gray{(\key{program}\;\itm{info} \; \Def^{*})}
\end{array}
\]
\end{minipage}
}
\caption{The $F_1$ language, an extension of $R_4$
  (Figure~\ref{fig:r4-syntax}).}
\label{fig:f1-syntax}
\end{figure}

%% Distinguishing between calls in tail position and non-tail position
%% requires the pass to have some notion of context. We recommend using
%% two mutually recursive functions, one for processing expressions in
%% tail position and another for the rest. 

Placing this pass after \code{uniquify} is a good idea, because it
will make sure that there are no local variables and functions that
share the same name. On the other hand, \code{reveal-functions} needs
to come before the \code{explicate-control} pass because that pass
will help us compile \code{fun-ref} into assignment statements.

\section{Limit Functions}
\label{sec:limit-functions-r4}

This pass transforms functions so that they have at most six
parameters and transforms all function calls so that they pass at most
six arguments.  A simple strategy for imposing an argument limit of
length $n$ is to take all arguments $i$ where $i \geq n$ and pack them
into a vector, making that subsequent vector the $n$th argument.

\begin{tabular}{lll}
\begin{minipage}{0.2\textwidth}
\begin{lstlisting}
  (|$f$| |$x_1$| |$\ldots$| |$x_n$|) 
\end{lstlisting}
\end{minipage}
&
$\Rightarrow$
&
\begin{minipage}{0.4\textwidth}
\begin{lstlisting}
(|$f$| |$x_1$| |$\ldots$| |$x_5$| (vector |$x_6$| |$\ldots$| |$x_n$|))
\end{lstlisting}
\end{minipage}
\end{tabular}

In the body of the function, all occurrances of the $i$th argument in
which $i>5$ must be replaced with a \code{vector-ref}.

\section{Remove Complex Operators and Operands}
\label{sec:rco-r4}

The primary decisions to make for this pass is whether to classify
\code{fun-ref} and \code{app} as either simple or complex
expressions. Recall that a simple expression will eventually end up as
just an ``immediate'' argument of an x86 instruction. Function
application will be translated to a sequence of instructions, so
\code{app} must be classified as complex expression. Regarding
\code{fun-ref}, as discussed above, the function label needs to
be converted to an address using the \code{leaq} instruction. Thus,
even though \code{fun-ref} seems rather simple, it needs to be
classified as a complex expression so that we generate an assignment
statement with a left-hand side that can serve as the target of the
\code{leaq}.

\section{Explicate Control and the $C_3$ language}
\label{sec:explicate-control-r4}

Figure~\ref{fig:c3-syntax} defines the syntax for $C_3$, the output of
\key{explicate-control}. The three mutually recursive functions for
this pass, for assignment, tail, and predicate contexts, must all be
updated with cases for \code{fun-ref} and \code{app}. In
assignment and predicate contexts, \code{app} becomes \code{call},
whereas in tail position \code{app} becomes \code{tailcall}.  We
recommend defining a new function for processing function definitions.
This code is similar to the case for \code{program} in $R_3$.  The
top-level \code{explicate-control} function that handles the
\code{program} form of $R_4$ can then apply this new function to all
the function definitions.

\begin{figure}[tp]
\fbox{
\begin{minipage}{0.96\textwidth}
\[
\begin{array}{lcl}
\Arg &::=& \gray{ \Int \mid \Var \mid \key{\#t} \mid \key{\#f} }
  \\
\itm{cmp} &::= & \gray{  \key{eq?} \mid \key{<} } \\
\Exp &::= & \gray{ \Arg \mid (\key{read}) \mid (\key{-}\;\Arg) \mid (\key{+} \; \Arg\;\Arg)
      \mid (\key{not}\;\Arg) \mid (\itm{cmp}\;\Arg\;\Arg)  } \\
   &\mid& \gray{  (\key{allocate}\,\Int\,\Type)
   \mid (\key{vector-ref}\, \Arg\, \Int)  } \\
   &\mid& \gray{  (\key{vector-set!}\,\Arg\,\Int\,\Arg) \mid (\key{global-value} \,\itm{name}) \mid (\key{void}) } \\
   &\mid& (\key{fun-ref}\,\itm{label}) \mid (\key{call} \,\Arg\,\Arg^{*}) \\
\Stmt &::=& \gray{ \ASSIGN{\Var}{\Exp} \mid \RETURN{\Exp} 
       \mid (\key{collect} \,\itm{int}) }\\
\Tail &::= & \gray{\RETURN{\Exp} \mid (\key{seq}\;\Stmt\;\Tail)} \\
      &\mid& \gray{(\key{goto}\,\itm{label})
       \mid \IF{(\itm{cmp}\, \Arg\,\Arg)}{(\key{goto}\,\itm{label})}{(\key{goto}\,\itm{label})}} \\
      &\mid& (\key{tailcall} \,\Arg\,\Arg^{*}) \\
  \Def &::=& (\key{define}\; (\itm{label} \; [\Var \key{:} \Type]^{*}) \key{:} \Type \; ((\itm{label}\,\key{.}\,\Tail)^{+})) \\
C_3 & ::= & (\key{program}\;\itm{info}\;\Def^{*})
\end{array}
\]
\end{minipage}
}
\caption{The $C_3$ language, extending $C_2$ (Figure~\ref{fig:c2-syntax}) with functions.}
\label{fig:c3-syntax}
\end{figure}

\section{Uncover Locals}
\label{sec:uncover-locals-r4}

The function for processing $\Tail$ should be updated with a case for
\code{tailcall}. We also recommend creating a new function for
processing function definitions. Each function definition in $C_3$ has
its own set of local variables, so the code for function definitions
should be similar to the case for the \code{program} form in $C_2$.

\section{Select Instructions}
\label{sec:select-r4}

The output of select instructions is a program in the x86$_3$
language, whose syntax is defined in Figure~\ref{fig:x86-3}.

\begin{figure}[tp]
\fbox{
\begin{minipage}{0.96\textwidth}
\[
\begin{array}{lcl}
\Arg &::=&  \gray{  \INT{\Int} \mid \REG{\itm{register}}
    \mid (\key{deref}\,\itm{register}\,\Int) } \\
   &\mid& \gray{ (\key{byte-reg}\; \itm{register}) 
    \mid   (\key{global-value}\; \itm{name})  } \\
   &\mid& (\key{fun-ref}\; \itm{label})\\
\itm{cc} & ::= & \gray{  \key{e} \mid \key{l} \mid \key{le} \mid \key{g} \mid \key{ge}  } \\
\Instr &::=& \gray{  (\key{addq} \; \Arg\; \Arg) \mid
             (\key{subq} \; \Arg\; \Arg) \mid
             (\key{negq} \; \Arg) \mid (\key{movq} \; \Arg\; \Arg)  } \\
      &\mid& \gray{  (\key{callq} \; \mathit{label}) \mid
             (\key{pushq}\;\Arg) \mid
             (\key{popq}\;\Arg) \mid
             (\key{retq})  } \\
       &\mid& \gray{  (\key{xorq} \; \Arg\;\Arg)
       \mid (\key{cmpq} \; \Arg\; \Arg) \mid (\key{set}\itm{cc} \; \Arg)  } \\
       &\mid& \gray{  (\key{movzbq}\;\Arg\;\Arg)
       \mid  (\key{jmp} \; \itm{label})
       \mid (\key{j}\itm{cc} \; \itm{label})
       \mid (\key{label} \; \itm{label})  } \\
     &\mid& (\key{indirect-callq}\;\Arg ) \mid (\key{tail-jmp}\;\Arg) \\
     &\mid& (\key{leaq}\;\Arg\;\Arg)\\
\Block &::= & \gray{(\key{block} \;\itm{info}\; \Instr^{+})} \\
\Def &::= & (\key{define} \; (\itm{label}) \;\itm{info}\; ((\itm{label} \,\key{.}\, \Block)^{+}))\\
x86_3 &::= & (\key{program} \;\itm{info} \;\Def^{*})
\end{array}
\]
\end{minipage}
}
\caption{The x86$_3$ language (extends x86$_2$ of Figure~\ref{fig:x86-2}).}
\label{fig:x86-3}
\end{figure}

An assignment of \code{fun-ref} becomes a \code{leaq} instruction
as follows: \\
\begin{tabular}{lll}
\begin{minipage}{0.45\textwidth}
\begin{lstlisting}
  (assign |$\itm{lhs}$| (fun-ref |$f$|))
\end{lstlisting}
\end{minipage}
&
$\Rightarrow$
&
\begin{minipage}{0.4\textwidth}
\begin{lstlisting}
(leaq (fun-ref |$f$|) |$\itm{lhs}$|)
\end{lstlisting}
\end{minipage}
\end{tabular} \\


Regarding function definitions, we need to remove their parameters and
instead perform parameter passing in terms of the conventions
discussed in Section~\ref{sec:fun-x86}. That is, the arguments will be
in the argument passing registers, and inside the function we should
generate a \code{movq} instruction for each parameter, to move the
argument value from the appropriate register to a new local variable
with the same name as the old parameter.

Next, consider the compilation of function calls, which have the
following form upon input to \code{select-instructions}.
\begin{lstlisting}
  (assign |\itm{lhs}| (call |\itm{fun}| |\itm{args}| |$\ldots$|))
\end{lstlisting}
In the mirror image of handling the parameters of function
definitions, the arguments \itm{args} need to be moved to the argument
passing registers.
%
Once the instructions for parameter passing have been generated, the
function call itself can be performed with an indirect function call,
for which I recommend creating the new instruction
\code{indirect-callq}. Of course, the return value from the function
is stored in \code{rax}, so it needs to be moved into the \itm{lhs}.
\begin{lstlisting}
  (indirect-callq |\itm{fun}|)
  (movq (reg rax) |\itm{lhs}|)
\end{lstlisting}

Regarding tail calls, the parameter passing is the same as non-tail
calls: generate instructions to move the arguments into to the
argument passing registers.  After that we need to pop the frame from
the procedure call stack.  However, we do not yet know how big the
frame is; that gets determined during register allocation. So instead
of generating those instructions here, we invent a new instruction
that means ``pop the frame and then do an indirect jump'', which we
name \code{tail-jmp}.

Recall that in Section~\ref{sec:explicate-control-r1} we recommended
using the label \code{start} for the initial block of a program, and
in Section~\ref{sec:select-r1} we recommended labelling the conclusion
of the program with \code{conclusion}, so that $(\key{return}\;\Arg)$
can be compiled to an assignment to \code{rax} followed by a jump to
\code{conclusion}. With the addition of function definitions, we will
have a starting block and conclusion for each function, but their
labels need to be unique. We recommend prepending the function's name
to \code{start} and \code{conclusion}, respectively, to obtain unique
labels. (Alternatively, one could \code{gensym} labels for the start
and conclusion and store them in the $\itm{info}$ field of the
function definition.)


\section{Uncover Live}

%% The rest of the passes need only minor modifications to handle the new
%% kinds of AST nodes: \code{fun-ref}, \code{indirect-callq}, and
%% \code{leaq}. 

Inside \code{uncover-live}, when computing the $W$ set (written
variables) for an \code{indirect-callq} instruction, we recommend
including all the caller-saved registers, which will have the affect
of making sure that no caller-saved register actually needs to be
saved.

\section{Build Interference Graph}

With the addition of function definitions, we compute an interference
graph for each function (not just one for the whole program).

Recall that in Section~\ref{sec:reg-alloc-gc} we discussed the need to
spill vector-typed variables that are live during a call to the
\code{collect}.  With the addition of functions to our language, we
need to revisit this issue. Many functions will perform allocation and
therefore have calls to the collector inside of them. Thus, we should
not only spill a vector-typed variable when it is live during a call
to \code{collect}, but we should spill the variable if it is live
during any function call. Thus, in the \code{build-interference} pass,
we recommend adding interference edges between call-live vector-typed
variables and the callee-saved registers (in addition to the usual
addition of edges between call-live variables and the caller-saved
registers).

\section{Patch Instructions}

In \code{patch-instructions}, you should deal with the x86
idiosyncrasy that the destination argument of \code{leaq} must be a
register. Additionally, you should ensure that the argument of
\code{tail-jmp} is \itm{rax}, our reserved register---this is to make
code generation more convenient, because we will be trampling many
registers before the tail call (as explained below).

\section{Print x86}

For the \code{print-x86} pass, we recommend the following translations:
\begin{lstlisting}
  (fun-ref |\itm{label}|) |$\Rightarrow$| |\itm{label}|(%rip)
  (indirect-callq |\itm{arg}|) |$\Rightarrow$| callq *|\itm{arg}|
\end{lstlisting}
Handling \code{tail-jmp} requires a bit more care. A straightforward
translation of \code{tail-jmp} would be \code{jmp *$\itm{arg}$}, which
is what we will want to do, but before the jump we need to pop the
current frame. So we need to restore the state of the registers to the
point they were at when the current function was called.  This
sequence of instructions is the same as the code for the conclusion of
a function.

Note that your \code{print-x86} pass needs to add the code for saving
and restoring callee-saved registers, if you have not already
implemented that. This is necessary when generating code for function
definitions.

\section{An Example Translation}

Figure~\ref{fig:add-fun} shows an example translation of a simple
function in $R_4$ to x86. The figure also includes the results of the
\code{explicate-control} and \code{select-instructions} passes.  We
have ommited the \code{has-type} AST nodes for readability.  Can you
see any ways to improve the translation?

\begin{figure}[tbp]
\begin{tabular}{ll}
\begin{minipage}{0.45\textwidth}
% s3_2.rkt
\begin{lstlisting}[basicstyle=\ttfamily\scriptsize]
(program
 (define (add [x : Integer]
                [y : Integer])
    : Integer (+ x y))
 (add 40 2))
\end{lstlisting}
$\Downarrow$
\begin{lstlisting}[basicstyle=\ttfamily\scriptsize]
(program ()
 (define (add86 [x87 : Integer]
           [y88 : Integer]) : Integer ()
   ((add86start . (return (+ x87 y88)))))
 (define (main) : Integer ()
   ((mainstart . 
      (seq (assign tmp89 (fun-ref add86))
           (tailcall tmp89 40 2))))))
\end{lstlisting}
\end{minipage}
&
$\Rightarrow$
\begin{minipage}{0.5\textwidth}
\begin{lstlisting}[basicstyle=\ttfamily\scriptsize]
(program ()
 (define (add86)
   ((locals (x87 . Integer) (y88 . Integer)) 
    (num-params . 2))
   ((add86start .
      (block ()
        (movq (reg rcx) (var x87))
        (movq (reg rdx) (var y88))
        (movq (var x87) (reg rax))
        (addq (var y88) (reg rax))
        (jmp add86conclusion)))))
 (define (main)
   ((locals . ((tmp89 . (Integer Integer -> Integer))))
    (num-params . 0))
   ((mainstart .
      (block ()
        (leaq (fun-ref add86) (var tmp89))
        (movq (int 40) (reg rcx))
        (movq (int 2) (reg rdx))
        (tail-jmp (var tmp89))))))
\end{lstlisting}
$\Downarrow$
\end{minipage}
\end{tabular}
\begin{tabular}{lll}
\begin{minipage}{0.3\textwidth}
\begin{lstlisting}[basicstyle=\ttfamily\scriptsize]
_add90start:
	movq	%rcx, %rsi
	movq	%rdx, %rcx
	movq	%rsi, %rax
	addq	%rcx, %rax
	jmp _add90conclusion
	.globl _add90
	.align 16
_add90:
	pushq	%rbp
	movq	%rsp, %rbp
	pushq	%r12
	pushq	%rbx
	pushq	%r13
	pushq	%r14
	subq	$0, %rsp
	jmp _add90start
_add90conclusion:
	addq	$0, %rsp
	popq	%r14
	popq	%r13
	popq	%rbx
	popq	%r12
	subq $0, %r15
	popq	%rbp
	retq
\end{lstlisting}
\end{minipage}
&
\begin{minipage}{0.3\textwidth}
\begin{lstlisting}[basicstyle=\ttfamily\scriptsize]
_mainstart:
	leaq	_add90(%rip), %rsi
	movq	$40, %rcx
	movq	$2, %rdx
	movq	%rsi, %rax
	addq	$0, %rsp
	popq	%r14
	popq	%r13
	popq	%rbx
	popq	%r12
	subq $0, %r15
	popq	%rbp
	jmp	*%rax

	.globl _main
	.align 16
_main:
	pushq	%rbp
	movq	%rsp, %rbp
	pushq	%r12
	pushq	%rbx
	pushq	%r13
	pushq	%r14
	subq	$0, %rsp
	movq $16384, %rdi
	movq $16, %rsi
	callq _initialize
	movq _rootstack_begin(%rip), %r15
	jmp _mainstart
\end{lstlisting}
\end{minipage}
&
\begin{minipage}{0.3\textwidth}
\begin{lstlisting}[basicstyle=\ttfamily\scriptsize]
_mainconclusion:
	addq	$0, %rsp
	popq	%r14
	popq	%r13
	popq	%rbx
	popq	%r12
	subq $0, %r15
	popq	%rbp
	retq
\end{lstlisting}
\end{minipage}
\end{tabular}
\caption{Example compilation of a simple function to x86.}
\label{fig:add-fun}
\end{figure}

\begin{exercise}\normalfont
Expand your compiler to handle $R_4$ as outlined in this section.
Create 5 new programs that use functions, including examples that pass
functions and return functions from other functions and including
recursive functions. Test your compiler on these new programs and all
of your previously created test programs.
\end{exercise}

\begin{figure}[p]
\begin{tikzpicture}[baseline=(current  bounding  box.center)]
\node (R4) at (0,2)  {\large $R_4$};
\node (R4-2) at (3,2)  {\large $R_4$};
\node (R4-3) at (6,2)  {\large $R_4$};
\node (F1-1) at (12,0)  {\large $F_1$};
\node (F1-2) at (9,0)  {\large $F_1$};
\node (F1-3) at (6,0)  {\large $F_1$};
\node (F1-4) at (3,0)  {\large $F_1$};
\node (C3-1) at (6,-2)  {\large $C_3$};
\node (C3-2) at (3,-2)  {\large $C_3$};

\node (x86-2) at (3,-4)  {\large $\text{x86}^{*}_3$};
\node (x86-3) at (6,-4)  {\large $\text{x86}^{*}_3$};
\node (x86-4) at (9,-4) {\large $\text{x86}^{*}_3$};
\node (x86-5) at (9,-6) {\large $\text{x86}^{\dagger}_3$};

\node (x86-2-1) at (3,-6)  {\large $\text{x86}^{*}_3$};
\node (x86-2-2) at (6,-6)  {\large $\text{x86}^{*}_3$};

\path[->,bend left=15] (R4) edge [above] node
     {\ttfamily\footnotesize\color{red} typecheck} (R4-2);
\path[->,bend left=15] (R4-2) edge [above] node
     {\ttfamily\footnotesize uniquify} (R4-3);
\path[->,bend left=15] (R4-3) edge [right] node
     {\ttfamily\footnotesize\color{red} reveal-functions} (F1-1);
\path[->,bend left=15] (F1-1) edge [below] node
     {\ttfamily\footnotesize\color{red} limit-functions} (F1-2);
\path[->,bend right=15] (F1-2) edge [above] node
     {\ttfamily\footnotesize expose-alloc.} (F1-3);
\path[->,bend right=15] (F1-3) edge [above] node
     {\ttfamily\footnotesize\color{red} remove-complex.} (F1-4);
\path[->,bend left=15] (F1-4) edge [right] node
     {\ttfamily\footnotesize\color{red} explicate-control} (C3-1);
\path[->,bend left=15] (C3-1) edge [below] node
     {\ttfamily\footnotesize\color{red} uncover-locals} (C3-2);
\path[->,bend right=15] (C3-2) edge [left] node
     {\ttfamily\footnotesize\color{red} select-instr.} (x86-2);
\path[->,bend left=15] (x86-2) edge [left] node
     {\ttfamily\footnotesize\color{red} uncover-live} (x86-2-1);
\path[->,bend right=15] (x86-2-1) edge [below] node 
     {\ttfamily\footnotesize \color{red}build-inter.} (x86-2-2);
\path[->,bend right=15] (x86-2-2) edge [left] node
     {\ttfamily\footnotesize allocate-reg.} (x86-3);
\path[->,bend left=15] (x86-3) edge [above] node
     {\ttfamily\footnotesize\color{red} patch-instr.} (x86-4);
\path[->,bend right=15] (x86-4) edge [left] node {\ttfamily\footnotesize\color{red} print-x86} (x86-5);
\end{tikzpicture}
\caption{Diagram of the passes for $R_4$, a language with functions.}
\label{fig:R4-passes}
\end{figure}

Figure~\ref{fig:R4-passes} gives an overview of the passes needed for
the compilation of $R_4$.


%%%%%%%%%%%%%%%%%%%%%%%%%%%%%%%%%%%%%%%%%%%%%%%%%%%%%%%%%%%%%%%%%%%%%%%%%%%%%%%%
\chapter{Lexically Scoped Functions}
\label{ch:lambdas}

This chapter studies lexically scoped functions as they appear in
functional languages such as Racket. By lexical scoping we mean that a
function's body may refer to variables whose binding site is outside
of the function, in an enclosing scope.
%
Consider the example in Figure~\ref{fig:lexical-scoping} featuring an
anonymous function defined using the \key{lambda} form.  The body of
the \key{lambda}, refers to three variables: \code{x}, \code{y}, and
\code{z}. The binding sites for \code{x} and \code{y} are outside of
the \key{lambda}. Variable \code{y} is bound by the enclosing
\key{let} and \code{x} is a parameter of \code{f}. The \key{lambda} is
returned from the function \code{f}. Below the definition of \code{f},
we have two calls to \code{f} with different arguments for \code{x},
first \code{5} then \code{3}. The functions returned from \code{f} are
bound to variables \code{g} and \code{h}. Even though these two
functions were created by the same \code{lambda}, they are really
different functions because they use different values for
\code{x}. Finally, we apply \code{g} to \code{11} (producing
\code{20}) and apply \code{h} to \code{15} (producing \code{22}) so
the result of this program is \code{42}.

\begin{figure}[btp]
% s4_6.rkt
\begin{lstlisting}
   (define (f [x : Integer]) : (Integer -> Integer)
      (let ([y 4])
         (lambda: ([z : Integer]) : Integer
            (+ x (+ y z)))))

   (let ([g (f 5)])
     (let ([h (f 3)])
       (+ (g 11) (h 15))))
\end{lstlisting}
\caption{Example of a lexically scoped function.}
\label{fig:lexical-scoping}
\end{figure}


\section{The $R_5$ Language}

The syntax for this language with anonymous functions and lexical
scoping, $R_5$, is defined in Figure~\ref{fig:r5-syntax}. It adds the
\key{lambda} form to the grammar for $R_4$, which already has syntax
for function application.  In this chapter we shall descibe how to
compile $R_5$ back into $R_4$, compiling lexically-scoped functions
into a combination of functions (as in $R_4$) and tuples (as in
$R_3$).

\begin{figure}[tp]
\centering
\fbox{
\begin{minipage}{0.96\textwidth}
\[
\begin{array}{lcl}
  \Type &::=& \gray{\key{Integer} \mid \key{Boolean}
     \mid (\key{Vector}\;\Type^{+}) \mid \key{Void}
     \mid (\Type^{*} \; \key{->}\; \Type)} \\
  \Exp &::=& \gray{\Int \mid (\key{read}) \mid (\key{-}\;\Exp)
     \mid (\key{+} \; \Exp\;\Exp) \mid (\key{-} \; \Exp\;\Exp)}  \\
    &\mid&  \gray{\Var \mid \LET{\Var}{\Exp}{\Exp}}\\
    &\mid& \gray{\key{\#t} \mid \key{\#f} 
     \mid (\key{and}\;\Exp\;\Exp) 
     \mid (\key{or}\;\Exp\;\Exp) 
     \mid (\key{not}\;\Exp) } \\
    &\mid& \gray{(\key{eq?}\;\Exp\;\Exp) \mid \IF{\Exp}{\Exp}{\Exp}} \\
    &\mid& \gray{(\key{vector}\;\Exp^{+}) \mid
          (\key{vector-ref}\;\Exp\;\Int)} \\
    &\mid& \gray{(\key{vector-set!}\;\Exp\;\Int\;\Exp)\mid (\key{void})} \\
    &\mid& \gray{(\Exp \; \Exp^{*})} \\
    &\mid& (\key{lambda:}\; ([\Var \key{:} \Type]^{*}) \key{:} \Type \; \Exp) \\
  \Def &::=& \gray{(\key{define}\; (\Var \; [\Var \key{:} \Type]^{*}) \key{:} \Type \; \Exp)} \\
  R_5 &::=& \gray{(\key{program} \; \Def^{*} \; \Exp)}
\end{array}
\]
\end{minipage}
}
\caption{Syntax of $R_5$, extending $R_4$ (Figure~\ref{fig:r4-syntax}) 
  with \key{lambda}.}
\label{fig:r5-syntax}
\end{figure}

To compile lexically-scoped functions to top-level function
definitions, the compiler will need to provide special treatment to
variable occurences such as \code{x} and \code{y} in the body of the
\code{lambda} of Figure~\ref{fig:lexical-scoping}, for the functions
of $R_4$ may not refer to variables defined outside the function. To
identify such variable occurences, we review the standard notion of
free variable.

\begin{definition}
A variable is \emph{free with respect to an expression} $e$ if the
variable occurs inside $e$ but does not have an enclosing binding in
$e$.
\end{definition}

For example, the variables \code{x}, \code{y}, and \code{z} are all
free with respect to the expression \code{(+ x (+ y z))}.  On the
other hand, only \code{x} and \code{y} are free with respect to the
following expression becuase \code{z} is bound by the \code{lambda}.
\begin{lstlisting}
   (lambda: ([z : Integer]) : Integer
      (+ x (+ y z)))
\end{lstlisting}

Once we have identified the free variables of a \code{lambda}, we need
to arrange for some way to transport, at runtime, the values of those
variables from the point where the \code{lambda} was created to the
point where the \code{lambda} is applied. Referring again to
Figure~\ref{fig:lexical-scoping}, the binding of \code{x} to \code{5}
needs to be used in the application of \code{g} to \code{11}, but the
binding of \code{x} to \code{3} needs to be used in the application of
\code{h} to \code{15}. An efficient solution to the problem, due to
\citet{Cardelli:1983aa}, is to bundle into a vector the values of the
free variables together with the function pointer for the lambda's
code, an arrangement called a \emph{flat closure} (which we shorten to
just ``closure'') . Fortunately, we have all the ingredients to make
closures, Chapter~\ref{ch:tuples} gave us vectors and
Chapter~\ref{ch:functions} gave us function pointers. The function
pointer shall reside at index $0$ and the values for free variables
will fill in the rest of the vector. Figure~\ref{fig:closures} depicts
the two closures created by the two calls to \code{f} in
Figure~\ref{fig:lexical-scoping}.  Because the two closures came from
the same \key{lambda}, they share the same function pointer but differ
in the values for the free variable \code{x}.

\begin{figure}[tbp]
\centering \includegraphics[width=0.6\textwidth]{figs/closures}
\caption{Example closure representation for the \key{lambda}'s
  in Figure~\ref{fig:lexical-scoping}.}
\label{fig:closures}
\end{figure}


\section{Interpreting $R_5$}

Figure~\ref{fig:interp-R5} shows the definitional interpreter for
$R_5$. The clause for \key{lambda} saves the current environment
inside the returned \key{lambda}. Then the clause for \key{app} uses
the environment from the \key{lambda}, the \code{lam-env}, when
interpreting the body of the \key{lambda}.  The \code{lam-env}
environment is extended with the mapping of parameters to argument
values.

\begin{figure}[tbp]
\begin{lstlisting}
(define (interp-exp env)
  (lambda (e)
    (define recur (interp-exp env))
    (match e
      ...
      [`(lambda: ([,xs : ,Ts] ...) : ,rT ,body)
       `(lambda ,xs ,body ,env)]
      [`(app ,fun ,args ...)
       (define fun-val ((interp-exp env) fun))
       (define arg-vals (map (interp-exp env) args))
       (match fun-val
	 [`(lambda (,xs ...) ,body ,lam-env)
	  (define new-env (append (map cons xs arg-vals) lam-env))
	  ((interp-exp new-env) body)]
	 [else (error "interp-exp, expected function, not" fun-val)])]
      [else (error 'interp-exp "unrecognized expression")]
      )))
\end{lstlisting}
\caption{Interpreter for $R_5$.}
\label{fig:interp-R5}
\end{figure}

\section{Type Checking $R_5$}

Figure~\ref{fig:typecheck-R5} shows how to type check the new
\key{lambda} form. The body of the \key{lambda} is checked in an
environment that includes the current environment (because it is
lexically scoped) and also includes the \key{lambda}'s parameters.  We
require the body's type to match the declared return type.

\begin{figure}[tbp]
\begin{lstlisting}
(define (typecheck-R5 env)
  (lambda (e)
    (match e
      [`(lambda: ([,xs : ,Ts] ...) : ,rT ,body)
       (define new-env (append (map cons xs Ts) env))
       (define bodyT ((typecheck-R5 new-env) body))
       (cond [(equal? rT bodyT)
              `(,@Ts -> ,rT)]
             [else
              (error "mismatch in return type" bodyT rT)])]
      ...
      )))
\end{lstlisting}
\caption{Type checking the \key{lambda}'s in $R_5$.}
\label{fig:typecheck-R5}
\end{figure}


\section{Closure Conversion}

The compiling of lexically-scoped functions into top-level function
definitions is accomplished in the pass \code{convert-to-closures}
that comes after \code{reveal-functions} and before
\code{limit-functions}. 

As usual, we shall implement the pass as a recursive function over the
AST. All of the action is in the clauses for \key{lambda} and
\key{app}. We transform a \key{lambda} expression into an expression
that creates a closure, that is, creates a vector whose first element
is a function pointer and the rest of the elements are the free
variables of the \key{lambda}.  The \itm{name} is a unique symbol
generated to identify the function.

\begin{tabular}{lll}
\begin{minipage}{0.4\textwidth}
\begin{lstlisting}
(lambda: (|\itm{ps}| ...) : |\itm{rt}| |\itm{body}|)
\end{lstlisting}
\end{minipage}
&
$\Rightarrow$
&
\begin{minipage}{0.4\textwidth}
\begin{lstlisting}
(vector |\itm{name}| |\itm{fvs}| ...)
\end{lstlisting}
\end{minipage}
\end{tabular}  \\
%
In addition to transforming each \key{lambda} into a \key{vector}, we
must create a top-level function definition for each \key{lambda}, as
shown below.\\
\begin{minipage}{0.8\textwidth}
  \begin{lstlisting}
   (define (|\itm{name}| [clos : (Vector _ |\itm{fvts}| ...)] |\itm{ps}| ...)
      (let ([|$\itm{fvs}_1$| (vector-ref clos 1)])
        ...
        (let ([|$\itm{fvs}_n$| (vector-ref clos |$n$|)])
          |\itm{body'}|)...))
\end{lstlisting}
\end{minipage}\\
The \code{clos} parameter refers to the closure. The $\itm{ps}$
parameters are the normal parameters of the \key{lambda}. The types
$\itm{fvts}$ are the types of the free variables in the lambda and the
underscore is a dummy type because it is rather difficult to give a
type to the function in the closure's type, and it does not matter.
The sequence of \key{let} forms bind the free variables to their
values obtained from the closure.

We transform function application into code that retreives the
function pointer from the closure and then calls the function, passing
in the closure as the first argument. We bind $e'$ to a temporary
variable to avoid code duplication.

\begin{tabular}{lll}
\begin{minipage}{0.3\textwidth}
\begin{lstlisting}
(app |$e$| |\itm{es}| ...)
\end{lstlisting}
\end{minipage}
&
$\Rightarrow$
&
\begin{minipage}{0.5\textwidth}
\begin{lstlisting}
(let ([|\itm{tmp}| |$e'$|])
  (app (vector-ref |\itm{tmp}| 0) |\itm{tmp}| |\itm{es'}|))
\end{lstlisting}
\end{minipage}
\end{tabular}  \\

There is also the question of what to do with top-level function
definitions. To maintain a uniform translation of function
application, we turn function references into closures.

\begin{tabular}{lll}
\begin{minipage}{0.3\textwidth}
\begin{lstlisting}
(fun-ref |$f$|)
\end{lstlisting}
\end{minipage}
&
$\Rightarrow$
&
\begin{minipage}{0.5\textwidth}
\begin{lstlisting}
(vector (fun-ref |$f$|))
\end{lstlisting}
\end{minipage}
\end{tabular}  \\
%
The top-level function definitions need to be updated as well to take
an extra closure parameter.

\section{An Example Translation}
\label{sec:example-lambda}

Figure~\ref{fig:lexical-functions-example} shows the result of closure
conversion for the example program demonstrating lexical scoping that
we discussed at the beginning of this chapter.


\begin{figure}[h]
\begin{minipage}{0.8\textwidth}
\begin{lstlisting}%[basicstyle=\ttfamily\footnotesize]
(program
 (define (f [x : Integer]) : (Integer -> Integer)
    (let ([y 4])
       (lambda: ([z : Integer]) : Integer
          (+ x (+ y z)))))
 (let ([g (f 5)])
   (let ([h (f 3)])
     (+ (g 11) (h 15)))))
\end{lstlisting}
$\Downarrow$
\begin{lstlisting}%[basicstyle=\ttfamily\footnotesize]
(program (type Integer)
  (define (f (x : Integer)) : (Integer -> Integer)
    (let ((y 4))
       (lambda: ((z : Integer)) : Integer
         (+ x (+ y z)))))
   (let ((g (app (fun-ref f) 5)))
      (let ((h (app (fun-ref f) 3)))
         (+ (app g 11) (app h 15)))))
\end{lstlisting}
$\Downarrow$
\begin{lstlisting}%[basicstyle=\ttfamily\footnotesize]
(program (type Integer)
  (define (f (clos.1 : _) (x : Integer)) : (Integer -> Integer)
     (let ((y 4))
        (vector (fun-ref lam.1) x y)))
  (define (lam.1 (clos.2 : _) (z : Integer)) : Integer
     (let ((x (vector-ref clos.2 1)))
        (let ((y (vector-ref clos.2 2)))
           (+ x (+ y z)))))
   (let ((g (let ((t.1 (vector (fun-ref f))))
              (app (vector-ref t.1 0) t.1 5))))
      (let ((h (let ((t.2 (vector  (fun-ref f))))
                 (app (vector-ref t.2 0) t.2 3))))
         (+ (let ((t.3 g)) (app (vector-ref t.3 0) t.3 11))
            (let ((t.4 h)) (app (vector-ref t.4 0) t.4 15))))))
\end{lstlisting}
\end{minipage}

\caption{Example of closure conversion.}
\label{fig:lexical-functions-example}
\end{figure}



\begin{figure}[p]
\begin{tikzpicture}[baseline=(current  bounding  box.center)]
\node (R4) at (0,2)  {\large $R_4$};
\node (R4-2) at (3,2)  {\large $R_4$};
\node (R4-3) at (6,2)  {\large $R_4$};
\node (F1-1) at (12,0)  {\large $F_1$};
\node (F1-2) at (9,0)  {\large $F_1$};
\node (F1-3) at (6,0)  {\large $F_1$};
\node (F1-4) at (3,0)  {\large $F_1$};
\node (F1-5) at (0,0)  {\large $F_1$};
\node (C3-1) at (6,-2)  {\large $C_3$};
\node (C3-2) at (3,-2)  {\large $C_3$};

\node (x86-2) at (3,-4)  {\large $\text{x86}^{*}_3$};
\node (x86-3) at (6,-4)  {\large $\text{x86}^{*}_3$};
\node (x86-4) at (9,-4) {\large $\text{x86}^{*}_3$};
\node (x86-5) at (9,-6) {\large $\text{x86}^{\dagger}_3$};

\node (x86-2-1) at (3,-6)  {\large $\text{x86}^{*}_3$};
\node (x86-2-2) at (6,-6)  {\large $\text{x86}^{*}_3$};

\path[->,bend left=15] (R4) edge [above] node
     {\ttfamily\footnotesize\color{red} typecheck} (R4-2);
\path[->,bend left=15] (R4-2) edge [above] node
     {\ttfamily\footnotesize uniquify} (R4-3);
\path[->] (R4-3) edge [right] node
     {\ttfamily\footnotesize reveal-functions} (F1-1);
\path[->,bend left=15] (F1-1) edge [below] node
     {\ttfamily\footnotesize\color{red} convert-to-clos.} (F1-2);
\path[->,bend right=15] (F1-2) edge [above] node
     {\ttfamily\footnotesize limit-functions} (F1-3);
\path[->,bend right=15] (F1-3) edge [above] node
     {\ttfamily\footnotesize expose-alloc.} (F1-4);
\path[->,bend right=15] (F1-4) edge [above] node
     {\ttfamily\footnotesize remove-complex.} (F1-5);
\path[->] (F1-5) edge [left] node
     {\ttfamily\footnotesize explicate-control} (C3-1);
\path[->,bend left=15] (C3-1) edge [below] node
     {\ttfamily\footnotesize uncover-locals} (C3-2);
\path[->,bend right=15] (C3-2) edge [left] node
     {\ttfamily\footnotesize select-instr.} (x86-2);
\path[->,bend left=15] (x86-2) edge [left] node
     {\ttfamily\footnotesize uncover-live} (x86-2-1);
\path[->,bend right=15] (x86-2-1) edge [below] node 
     {\ttfamily\footnotesize build-inter.} (x86-2-2);
\path[->,bend right=15] (x86-2-2) edge [left] node
     {\ttfamily\footnotesize allocate-reg.} (x86-3);
\path[->,bend left=15] (x86-3) edge [above] node
     {\ttfamily\footnotesize patch-instr.} (x86-4);
\path[->,bend right=15] (x86-4) edge [left] node {\ttfamily\footnotesize print-x86} (x86-5);
\end{tikzpicture}
  \caption{Diagram of the passes for $R_5$, a language with lexically-scoped
  functions.}
\label{fig:R5-passes}
\end{figure}

Figure~\ref{fig:R5-passes} provides an overview of all the passes needed
for the compilation of $R_5$.



%%%%%%%%%%%%%%%%%%%%%%%%%%%%%%%%%%%%%%%%%%%%%%%%%%%%%%%%%%%%%%%%%%%%%%%%%%%%%%%%
\chapter{Dynamic Typing}
\label{ch:type-dynamic}

In this chapter we discuss the compilation of a dynamically typed
language, named $R_7$, that is a subset of the Racket
language. (Recall that in the previous chapters we have studied
subsets of the \emph{Typed} Racket language.) In dynamically typed
languages, an expression may produce values of differing
type. Consider the following example with a conditional expression
that may return a Boolean or an integer depending on the input to the
program.
\begin{lstlisting}
   (not (if (eq? (read) 1) #f 0))
\end{lstlisting}
Languages that allow expressions to produce different kinds of values
are called \emph{polymorphic}. There are many kinds of polymorphism,
such as subtype polymorphism and parametric
polymorphism~\citep{Cardelli:1985kx}. The kind of polymorphism are
talking about here does not have a special name, but it is the usual
kind that arrises in dynamically typed languages.

Another characteristic of dynamically typed languages is that
primitive operations, such as \code{not}, are often defined to operate
on many different types of values. In fact, in Racket, the \code{not}
operator produces a result for any kind of value: given \code{\#f} it
returns \code{\#t} and given anything else it returns \code{\#f}.
Furthermore, even when primitive operations restrict their inputs to
values of a certain type, this restriction is enforced at runtime
instead of during compilation. For example, the following vector
reference results in a run-time contract violation.
\begin{lstlisting}
   (vector-ref (vector 42) #t)
\end{lstlisting}

\begin{figure}[tp]
\centering
\fbox{
\begin{minipage}{0.97\textwidth}
\[
\begin{array}{rcl}
  \itm{cmp} &::= & \key{eq?} \mid \key{<} \mid \key{<=} \mid \key{>} \mid \key{>=} \\
\Exp &::=& \Int \mid (\key{read}) \mid (\key{-}\;\Exp) 
      \mid (\key{+} \; \Exp\;\Exp) \mid (\key{-} \; \Exp\;\Exp)  \\
     &\mid&  \Var \mid \LET{\Var}{\Exp}{\Exp} \\
     &\mid& \key{\#t} \mid \key{\#f} 
      \mid (\key{and}\;\Exp\;\Exp) 
      \mid (\key{or}\;\Exp\;\Exp) 
      \mid (\key{not}\;\Exp) \\
     &\mid& (\itm{cmp}\;\Exp\;\Exp) \mid \IF{\Exp}{\Exp}{\Exp} \\
     &\mid& (\key{vector}\;\Exp^{+}) \mid
      (\key{vector-ref}\;\Exp\;\Exp) \\
     &\mid& (\key{vector-set!}\;\Exp\;\Exp\;\Exp) \mid (\key{void}) \\
     &\mid& (\Exp \; \Exp^{*}) \mid (\key{lambda}\; (\Var^{*}) \; \Exp) \\
     & \mid & (\key{boolean?}\;\Exp) \mid (\key{integer?}\;\Exp)\\
     & \mid & (\key{vector?}\;\Exp) \mid (\key{procedure?}\;\Exp) \mid (\key{void?}\;\Exp) \\
  \Def &::=& (\key{define}\; (\Var \; \Var^{*}) \; \Exp) \\
R_7  &::=& (\key{program} \; \Def^{*}\; \Exp)
\end{array}
\]
\end{minipage}
}
\caption{Syntax of $R_7$, an untyped language (a subset of Racket).}
\label{fig:r7-syntax}
\end{figure}

The syntax of $R_7$, our subset of Racket, is defined in
Figure~\ref{fig:r7-syntax}.
%
The definitional interpreter for $R_7$ is given in
Figure~\ref{fig:interp-R7}.

\begin{figure}[tbp]
\begin{lstlisting}[basicstyle=\ttfamily\footnotesize]
(define (get-tagged-type v) (match v [`(tagged ,v1 ,ty) ty]))

(define (valid-op? op) (member op '(+ - and or not)))

(define (interp-r7 env)
  (lambda (ast)
    (define recur (interp-r7 env))
    (match ast
      [(? symbol?) (lookup ast env)]
      [(? integer?) `(inject ,ast Integer)]
      [#t `(inject #t Boolean)]
      [#f `(inject #f Boolean)]
      [`(read) `(inject ,(read-fixnum) Integer)]
      [`(lambda (,xs ...) ,body)
       `(inject (lambda ,xs ,body ,env) (,@(map (lambda (x) 'Any) xs) -> Any))]
      [`(define (,f ,xs ...) ,body)
       (mcons f `(lambda ,xs ,body))]
      [`(program ,ds ... ,body)
       (let ([top-level (for/list ([d ds]) ((interp-r7 '()) d))])
         (for/list ([b top-level])
           (set-mcdr! b (match (mcdr b)
                          [`(lambda ,xs ,body)
                           `(inject (lambda ,xs ,body ,top-level)
                                    (,@(map (lambda (x) 'Any) xs) -> Any))])))
         ((interp-r7 top-level) body))]
      [`(vector ,(app recur elts) ...)
       (define tys (map get-tagged-type elts))
       `(inject ,(apply vector elts) (Vector ,@tys))]
      [`(vector-set! ,(app recur v1) ,n ,(app recur v2))
         (match v1
           [`(inject ,vec ,ty)
             (vector-set! vec n v2)
            `(inject (void) Void)])]
      [`(vector-ref ,(app recur v) ,n)
       (match v [`(inject ,vec ,ty) (vector-ref vec n)])]
      [`(let ([,x ,(app recur v)]) ,body)
       ((interp-r7 (cons (cons x v) env)) body)]
      [`(,op ,es ...) #:when (valid-op? op)
       (interp-r7-op op (for/list ([e es]) (recur e)))]
      [`(eq? ,(app recur l) ,(app recur r))
       `(inject ,(equal? l r) Boolean)]
      [`(if ,(app recur q) ,t ,f)
       (match q
         [`(inject #f Boolean) (recur f)]
         [else (recur t)])]
      [`(,(app recur f-val) ,(app recur vs) ...)
       (match f-val
         [`(inject (lambda (,xs ...) ,body ,lam-env) ,ty)
          (define new-env (append (map cons xs vs) lam-env))
          ((interp-r7 new-env) body)]
         [else (error "interp-r7, expected function, not" f-val)])])))
\end{lstlisting}
\caption{Interpreter for the $R_7$ language. UPDATE ME -Jeremy}
\label{fig:interp-R7}
\end{figure}


Let us consider how we might compile $R_7$ to x86, thinking about the
first example above. Our bit-level representation of the Boolean
\code{\#f} is zero and similarly for the integer \code{0}.  However,
\code{(not \#f)} should produce \code{\#t} whereas \code{(not 0)}
should produce \code{\#f}. Furthermore, the behavior of \code{not}, in
general, cannot be determined at compile time, but depends on the
runtime type of its input, as in the example above that depends on the
result of \code{(read)}.

The way around this problem is to include information about a value's
runtime type in the value itself, so that this information can be
inspected by operators such as \code{not}.  In particular, we shall
steal the 3 right-most bits from our 64-bit values to encode the
runtime type.  We shall use $001$ to identify integers, $100$ for
Booleans, $010$ for vectors, $011$ for procedures, and $101$ for the
void value. We shall refer to these 3 bits as the \emph{tag} and we
define the following auxilliary function.
\begin{align*}
\itm{tagof}(\key{Integer}) &= 001 \\
\itm{tagof}(\key{Boolean}) &= 100 \\
\itm{tagof}((\key{Vector} \ldots)) &= 010 \\
\itm{tagof}((\key{Vectorof} \ldots)) &= 010 \\
\itm{tagof}((\ldots \key{->} \ldots)) &= 011 \\
\itm{tagof}(\key{Void}) &= 101
\end{align*}
(We shall say more about the new \key{Vectorof} type shortly.)
This stealing of 3 bits comes at some
price: our integers are reduced to ranging from $-2^{60}$ to
$2^{60}$. The stealing does not adversely affect vectors and
procedures because those values are addresses, and our addresses are
8-byte aligned so the rightmost 3 bits are unused, they are always
$000$. Thus, we do not lose information by overwriting the rightmost 3
bits with the tag and we can simply zero-out the tag to recover the
original address.

In some sense, these tagged values are a new kind of value.  Indeed,
we can extend our \emph{typed} language with tagged values by adding a
new type to classify them, called \key{Any}, and with operations for
creating and using tagged values, yielding the $R_6$ language that we
define in Section~\ref{sec:r6-lang}. The $R_6$ language provides the
fundamental support for polymorphism and runtime types that we need to
support dynamic typing.

There is an interesting interaction between tagged values and garbage
collection.  A variable of type \code{Any} might refer to a vector and
therefore it might be a root that needs to be inspected and copied
during garbage collection. Thus, we need to treat variables of type
\code{Any} in a similar way to variables of type \code{Vector} for
purposes of register allocation, which we discuss in
Section~\ref{sec:register-allocation-r6}. One concern is that, if a
variable of type \code{Any} is spilled, it must be spilled to the root
stack.  But this means that the garbage collector needs to be able to
differentiate between (1) plain old pointers to tuples, (2) a tagged
value that points to a tuple, and (3) a tagged value that is not a
tuple. We enable this differentiation by choosing not to use the tag
$000$. Instead, that bit pattern is reserved for identifying plain old
pointers to tuples. On the other hand, if one of the first three bits
is set, then we have a tagged value, and inspecting the tag can
differentiation between vectors ($010$) and the other kinds of values.

We shall implement our untyped language $R_7$ by compiling it to $R_6$
(Section~\ref{sec:compile-r7}), but first we describe the how to
extend our compiler to handle the new features of $R_6$
(Sections~\ref{sec:shrink-r6}, \ref{sec:select-r6}, and
\ref{sec:register-allocation-r6}).

\section{The $R_6$ Language: Typed Racket $+$ \key{Any}}
\label{sec:r6-lang}

\begin{figure}[tp]
\centering
\fbox{
\begin{minipage}{0.97\textwidth}
\[
\begin{array}{lcl}
  \Type &::=& \gray{\key{Integer} \mid \key{Boolean}
     \mid (\key{Vector}\;\Type^{+}) \mid (\key{Vectorof}\;\Type) \mid \key{Void}} \\
    &\mid& \gray{(\Type^{*} \; \key{->}\; \Type)} \mid \key{Any} \\
  \FType &::=& \key{Integer} \mid \key{Boolean} \mid \key{Void} \mid (\key{Vectorof}\;\key{Any}) \mid (\key{Vector}\; \key{Any}^{*}) \\
    &\mid& (\key{Any}^{*} \; \key{->}\; \key{Any})\\
  \itm{cmp} &::= & \key{eq?} \mid \key{<} \mid \key{<=} \mid \key{>} \mid \key{>=} \\
  \Exp &::=& \gray{\Int \mid (\key{read}) \mid (\key{-}\;\Exp)
     \mid (\key{+} \; \Exp\;\Exp) \mid (\key{-} \; \Exp\;\Exp)}  \\
    &\mid&  \gray{\Var \mid \LET{\Var}{\Exp}{\Exp}} \\
    &\mid& \gray{\key{\#t} \mid \key{\#f} 
     \mid (\key{and}\;\Exp\;\Exp) 
     \mid (\key{or}\;\Exp\;\Exp) 
     \mid (\key{not}\;\Exp)} \\
    &\mid& \gray{(\itm{cmp}\;\Exp\;\Exp) \mid \IF{\Exp}{\Exp}{\Exp}} \\
    &\mid& \gray{(\key{vector}\;\Exp^{+}) \mid
          (\key{vector-ref}\;\Exp\;\Int)} \\
    &\mid& \gray{(\key{vector-set!}\;\Exp\;\Int\;\Exp)\mid (\key{void})} \\
    &\mid& \gray{(\Exp \; \Exp^{*})
    \mid (\key{lambda:}\; ([\Var \key{:} \Type]^{*}) \key{:} \Type \; \Exp)} \\
  & \mid & (\key{inject}\; \Exp \; \FType) \mid (\key{project}\;\Exp\;\FType) \\
  & \mid & (\key{boolean?}\;\Exp) \mid (\key{integer?}\;\Exp)\\
  & \mid & (\key{vector?}\;\Exp) \mid (\key{procedure?}\;\Exp) \mid (\key{void?}\;\Exp) \\
  \Def &::=& \gray{(\key{define}\; (\Var \; [\Var \key{:} \Type]^{*}) \key{:} \Type \; \Exp)} \\
  R_6 &::=& \gray{(\key{program} \; \Def^{*} \; \Exp)}
\end{array}
\]
\end{minipage}
}
\caption{Syntax of $R_6$, extending $R_5$ (Figure~\ref{fig:r5-syntax})
  with \key{Any}.}
\label{fig:r6-syntax}
\end{figure}

The syntax of $R_6$ is defined in Figure~\ref{fig:r6-syntax}.  The
$(\key{inject}\; e\; T)$ form converts the value produced by
expression $e$ of type $T$ into a tagged value.  The
$(\key{project}\;e\;T)$ form converts the tagged value produced by
expression $e$ into a value of type $T$ or else halts the program if
the type tag is equivalent to $T$. We treat
$(\key{Vectorof}\;\key{Any})$ as equivalent to
$(\key{Vector}\;\key{Any}\;\ldots)$.

Note that in both \key{inject} and
\key{project}, the type $T$ is restricted to the flat types $\FType$,
which simplifies the implementation and corresponds with what is
needed for compiling untyped Racket. The type predicates,
$(\key{boolean?}\,e)$ etc., expect a tagged value and return \key{\#t}
if the tag corresponds to the predicate, and return \key{\#t}
otherwise.
%
Selections from the type checker for $R_6$ are shown in
Figure~\ref{fig:typecheck-R6} and the interpreter for $R_6$ is in
Figure~\ref{fig:interp-R6}.

\begin{figure}[btp]
\begin{lstlisting}[basicstyle=\ttfamily\footnotesize]
(define (flat-ty? ty) ...)

(define (typecheck-R6 env)
  (lambda (e)
    (define recur (typecheck-R6 env))
    (match e
       [`(inject ,e ,ty)
        (unless (flat-ty? ty)
          (error "may only inject a value of flat type, not ~a" ty))
        (define-values (new-e e-ty) (recur e))
        (cond
         [(equal? e-ty ty)
          (values `(inject ,new-e ,ty) 'Any)]
         [else
          (error "inject expected ~a to have type ~a" e ty)])]
       [`(project ,e ,ty)
        (unless (flat-ty? ty)
          (error "may only project to a flat type, not ~a" ty))
        (define-values (new-e e-ty) (recur e))
        (cond
         [(equal? e-ty 'Any)
          (values `(project ,new-e ,ty) ty)]
         [else
          (error "project expected ~a to have type Any" e)])]
       [`(vector-ref ,e ,i)
        (define-values (new-e e-ty) (recur e))
        (match e-ty
          [`(Vector ,ts ...) ...]
          [`(Vectorof ,ty)
           (unless (exact-nonnegative-integer? i)
             (error 'type-check "invalid index ~a" i))
           (values `(vector-ref ,new-e ,i) ty)]
          [else (error "expected a vector in vector-ref, not" e-ty)])]
        ...
      )))
\end{lstlisting}
\caption{Type checker for parts of the $R_6$ language.}
\label{fig:typecheck-R6}
\end{figure}

% to do: add rules for vector-ref, etc. for Vectorof
%Also, \key{eq?} is extended to operate on values of type \key{Any}.

\begin{figure}[btp]
\begin{lstlisting}
(define primitives (set 'boolean? ...))

(define (interp-op op)
  (match op
     ['boolean? (lambda (v)
                  (match v
                     [`(tagged ,v1 Boolean) #t]
                     [else #f]))]
     ...))

;; Equavalence of flat types
(define (tyeq? t1 t2)
  (match `(,t1 ,t2)
    [`((Vectorof Any) (Vector ,t2s ...))
     (for/and ([t2 t2s]) (eq? t2 'Any))]
    [`((Vector ,t1s ...) (Vectorof Any))
     (for/and ([t1 t1s]) (eq? t1 'Any))]
    [else (equal? t1 t2)]))

(define (interp-R6 env)
  (lambda (ast)
    (match ast
       [`(inject ,e ,t)
        `(tagged ,((interp-R6 env) e) ,t)]
       [`(project ,e ,t2)
        (define v ((interp-R6 env) e))
        (match v
           [`(tagged ,v1 ,t1)
            (cond [(tyeq? t1 t2)
                   v1]
                  [else
                   (error "in project, type mismatch" t1 t2)])]
           [else
            (error "in project, expected tagged value" v)])]
       ...)))
\end{lstlisting}
\caption{Interpreter for $R_6$.}
\label{fig:interp-R6}
\end{figure}

%\clearpage

\section{Shrinking $R_6$}
\label{sec:shrink-r6}

In the \code{shrink} pass we recommend compiling \code{project} into
an explicit \code{if} expression that uses three new operations:
\code{tag-of-any}, \code{value-of-any}, and \code{exit}.  The
\code{tag-of-any} operation retrieves the type tag from a tagged value
of type \code{Any}.  The \code{value-of-any} retrieves the underlying
value from a tagged value. Finally, the \code{exit} operation ends the
execution of the program by invoking the operating system's
\code{exit} function. So the translation for \code{project} is as
follows. (We have ommitted the \code{has-type} AST nodes to make this
output more readable.)

\begin{tabular}{lll}
\begin{minipage}{0.3\textwidth}
\begin{lstlisting}
   (project |$e$| |$\Type$|)
\end{lstlisting}
\end{minipage}
&
$\Rightarrow$
&
\begin{minipage}{0.5\textwidth}
\begin{lstlisting}
(let ([|$\itm{tmp}$| |$e'$|])
  (if (eq? (tag-of-any |$\itm{tmp}$|) |$\itm{tag}$|)
      (value-of-any |$\itm{tmp}$|)
      (exit)))
\end{lstlisting}
\end{minipage}
\end{tabular}  \\

Similarly, we recommend translating the type predicates
(\code{boolean?}, etc.) into uses of \code{tag-of-any} and \code{eq?}.

\section{Instruction Selection for $R_6$}
\label{sec:select-r6}

\paragraph{Inject}

We recommend compiling an \key{inject} as follows if the type is
\key{Integer} or \key{Boolean}.  The \key{salq} instruction shifts the
destination to the left by the number of bits specified its source
argument (in this case $3$, the length of the tag) and it preserves
the sign of the integer. We use the \key{orq} instruction to combine
the tag and the value to form the tagged value.  \\
\begin{tabular}{lll}
\begin{minipage}{0.4\textwidth}
\begin{lstlisting}
(assign |\itm{lhs}| (inject |$e$| |$T$|))
\end{lstlisting}
\end{minipage}
&
$\Rightarrow$
&
\begin{minipage}{0.5\textwidth}
\begin{lstlisting}
(movq |$e'$| |\itm{lhs}'|)
(salq (int 3) |\itm{lhs}'|)
(orq (int |$\itm{tagof}(T)$|) |\itm{lhs}'|)
\end{lstlisting}
\end{minipage}
\end{tabular}  \\
The instruction selection for vectors and procedures is different
because their is no need to shift them to the left. The rightmost 3
bits are already zeros as described above. So we just combine the
value and the tag using \key{orq}.  \\
\begin{tabular}{lll}
\begin{minipage}{0.4\textwidth}
\begin{lstlisting}
(assign |\itm{lhs}| (inject |$e$| |$T$|))
\end{lstlisting}
\end{minipage}
&
$\Rightarrow$
&
\begin{minipage}{0.5\textwidth}
\begin{lstlisting}
(movq |$e'$| |\itm{lhs}'|)
(orq (int |$\itm{tagof}(T)$|) |\itm{lhs}'|)
\end{lstlisting}
\end{minipage}
\end{tabular} 

\paragraph{Tag of Any}

Recall that the \code{tag-of-any} operation extracts the type tag from
a value of type \code{Any}. The type tag is the bottom three bits, so
we obtain the tag by taking the bitwise-and of the value with $111$
($7$ in decimal).

\begin{tabular}{lll}
\begin{minipage}{0.4\textwidth}
\begin{lstlisting}
(assign |\itm{lhs}| (tag-of-any |$e$|))
\end{lstlisting}
\end{minipage}
&
$\Rightarrow$
&
\begin{minipage}{0.5\textwidth}
\begin{lstlisting}
(movq |$e'$| |\itm{lhs}'|)
(andq (int 7) |\itm{lhs}'|)
\end{lstlisting}
\end{minipage}
\end{tabular}  

\paragraph{Value of Any}

Like \key{inject}, the instructions for \key{value-of-any} are
different depending on whether the type $T$ is a pointer (vector or
procedure) or not (Integer or Boolean). The following shows the
instruction selection for Integer and Boolean.  We produce an untagged
value by shifting it to the right by 3 bits.
%
\\
\begin{tabular}{lll}
\begin{minipage}{0.4\textwidth}
\begin{lstlisting}
(assign |\itm{lhs}| (project |$e$| |$T$|))
\end{lstlisting}
\end{minipage}
&
$\Rightarrow$
&
\begin{minipage}{0.5\textwidth}
\begin{lstlisting}
(movq |$e'$| |\itm{lhs}'|)
(sarq (int 3) |\itm{lhs}'|)
\end{lstlisting}
\end{minipage}
\end{tabular}  \\
%
In the case for vectors and procedures, there is no need to
shift. Instead we just need to zero-out the rightmost 3 bits. We
accomplish this by creating the bit pattern $\ldots 0111$ ($7$ in
decimal) and apply \code{bitwise-not} to obtain $\ldots 1000$ which we
\code{movq} into the destination $\itm{lhs}$.  We then generate
\code{andq} with the tagged value to get the desired result. \\
%
\begin{tabular}{lll}
\begin{minipage}{0.4\textwidth}
\begin{lstlisting}
(assign |\itm{lhs}| (project |$e$| |$T$|))
\end{lstlisting}
\end{minipage}
&
$\Rightarrow$
&
\begin{minipage}{0.5\textwidth}
\begin{lstlisting}
(movq (int |$\ldots 1000$|) |\itm{lhs}'|)
(andq |$e'$| |\itm{lhs}'|)
\end{lstlisting}
\end{minipage}
\end{tabular}  

%% \paragraph{Type Predicates} We leave it to the reader to
%% devise a sequence of instructions to implement the type predicates
%% \key{boolean?}, \key{integer?}, \key{vector?}, and \key{procedure?}.

\section{Register Allocation for $R_6$}
\label{sec:register-allocation-r6}

As mentioned above, a variable of type \code{Any} might refer to a
vector. Thus, the register allocator for $R_6$ needs to treat variable
of type \code{Any} in the same way that it treats variables of type
\code{Vector} for purposes of garbage collection. In particular,
\begin{itemize}
\item If a variable of type \code{Any} is live during a function call,
  then it must be spilled. One way to accomplish this is to augment
  the pass \code{build-interference} to mark all variables that are
  live after a \code{callq} as interfering with all the registers.

\item If avariable of type \code{Any} is spilled, it must be spilled
  to the root stack instead of the normal procedure call stack.
\end{itemize}


\section{Compiling $R_7$ to $R_6$}
\label{sec:compile-r7}

Figure~\ref{fig:compile-r7-r6} shows the compilation of many of the
$R_7$ forms into $R_6$. An important invariant of this pass is that
given a subexpression $e$ of $R_7$, the pass will produce an
expression $e'$ of $R_6$ that has type \key{Any}. For example, the
first row in Figure~\ref{fig:compile-r7-r6} shows the compilation of
the Boolean \code{\#t}, which must be injected to produce an
expression of type \key{Any}.
%
The second row of Figure~\ref{fig:compile-r7-r6}, the compilation of
addition, is representative of compilation for many operations: the
arguments have type \key{Any} and must be projected to \key{Integer}
before the addition can be performed.

The compilation of \key{lambda} (third row of
Figure~\ref{fig:compile-r7-r6}) shows what happens when we need to
produce type annotations: we simply use \key{Any}.
%
The compilation of \code{if} and \code{eq?}  demonstrate how this pass
has to account for some differences in behavior between $R_7$ and
$R_6$. The $R_7$ language is more permissive than $R_6$ regarding what
kind of values can be used in various places. For example, the
condition of an \key{if} does not have to be a Boolean. For \key{eq?},
the arguments need not be of the same type (but in that case, the
result will be \code{\#f}).

\begin{figure}[btp]
\centering
\begin{tabular}{|lll|} \hline
\begin{minipage}{0.25\textwidth}
\begin{lstlisting}
#t
\end{lstlisting}
\end{minipage}
&
$\Rightarrow$
&
\begin{minipage}{0.6\textwidth}
\begin{lstlisting}
(inject #t Boolean)
\end{lstlisting}
\end{minipage}
\\[2ex]\hline
\begin{minipage}{0.25\textwidth}
\begin{lstlisting}
(+ |$e_1$| |$e_2$|)
\end{lstlisting}
\end{minipage}
&
$\Rightarrow$
&
\begin{minipage}{0.6\textwidth}
\begin{lstlisting}
(inject
   (+ (project |$e'_1$| Integer)
      (project |$e'_2$| Integer))
   Integer)
\end{lstlisting}
\end{minipage}
\\[2ex]\hline
\begin{minipage}{0.25\textwidth}
\begin{lstlisting}
(lambda (|$x_1 \ldots$|) |$e$|)
\end{lstlisting}
\end{minipage}
&
$\Rightarrow$
&
\begin{minipage}{0.6\textwidth}
\begin{lstlisting}
(inject (lambda: ([|$x_1$|:Any]|$\ldots$|):Any |$e'$|)
        (Any|$\ldots$|Any -> Any))
\end{lstlisting}
\end{minipage}
\\[2ex]\hline
\begin{minipage}{0.25\textwidth}
\begin{lstlisting}
(app |$e_0$| |$e_1 \ldots e_n$|)
\end{lstlisting}
\end{minipage}
&
$\Rightarrow$
&
\begin{minipage}{0.6\textwidth}
\begin{lstlisting}
(app (project |$e'_0$| (Any|$\ldots$|Any -> Any))
   |$e'_1 \ldots e'_n$|)
\end{lstlisting}
\end{minipage}
\\[2ex]\hline
\begin{minipage}{0.25\textwidth}
\begin{lstlisting}
(vector-ref |$e_1$| |$e_2$|)
\end{lstlisting}
\end{minipage}
&
$\Rightarrow$
&
\begin{minipage}{0.6\textwidth}
\begin{lstlisting}
(let ([tmp1 (project |$e'_1$| (Vectorof Any))])
  (let ([tmp2 (project |$e'_2$| Integer)])
     (vector-ref tmp1 tmp2)))
\end{lstlisting}
\end{minipage}
\\[2ex]\hline
\begin{minipage}{0.25\textwidth}
\begin{lstlisting}
(if |$e_1$| |$e_2$| |$e_3$|)
\end{lstlisting}
\end{minipage}
&
$\Rightarrow$
&
\begin{minipage}{0.6\textwidth}
\begin{lstlisting}
(if (eq? |$e'_1$| (inject #f Boolean))
   |$e'_3$|
   |$e'_2$|)
\end{lstlisting}
\end{minipage}
\\[2ex]\hline
\begin{minipage}{0.25\textwidth}
\begin{lstlisting}
(eq? |$e_1$| |$e_2$|)
\end{lstlisting}
\end{minipage}
&
$\Rightarrow$
&
\begin{minipage}{0.6\textwidth}
\begin{lstlisting}
(inject (eq? |$e'_1$| |$e'_2$|) Boolean)
\end{lstlisting}
\end{minipage}
\\[2ex]\hline
\end{tabular} 

\caption{Compiling $R_7$ to $R_6$.}
\label{fig:compile-r7-r6}
\end{figure}

%%%%%%%%%%%%%%%%%%%%%%%%%%%%%%%%%%%%%%%%%%%%%%%%%%%%%%%%%%%%%%%%%%%%%%%%%%%%%%%%
\chapter{Gradual Typing}
\label{ch:gradual-typing}

This chapter will be based on the ideas of \citet{Siek:2006bh}.

%%%%%%%%%%%%%%%%%%%%%%%%%%%%%%%%%%%%%%%%%%%%%%%%%%%%%%%%%%%%%%%%%%%%%%%%%%%%%%%%
\chapter{Parametric Polymorphism}
\label{ch:parametric-polymorphism}

This chapter may be based on ideas from \citet{Cardelli:1984aa},
\citet{Leroy:1992qb}, \citet{Shao:1997uj}, or \citet{Harper:1995um}.





%%%%%%%%%%%%%%%%%%%%%%%%%%%%%%%%%%%%%%%%%%%%%%%%%%%%%%%%%%%%%%%%%%%%%%%%%%%%%%%%
\chapter{High-level Optimization}
\label{ch:high-level-optimization}

This chapter will present a procedure inlining pass based on the
algorithm of \citet{Waddell:1997fk}.


%%%%%%%%%%%%%%%%%%%%%%%%%%%%%%%%%%%%%%%%%%%%%%%%%%%%%%%%%%%%%%%%%%%%%%%%%%%%%%%%
\chapter{Appendix}

\section{Interpreters}
\label{appendix:interp}

We provide several interpreters in the \key{interp.rkt} file.  The
\key{interp-scheme} function takes an AST in one of the Racket-like
languages considered in this book ($R_1, R_2, \ldots$) and interprets
the program, returning the result value.  The \key{interp-C} function
interprets an AST for a program in one of the C-like languages ($C_0,
C_1, \ldots$), and the \code{interp-x86} function interprets an AST
for an x86 program.

\section{Utility Functions}
\label{appendix:utilities}

The utility function described in this section can be found in the
\key{utilities.rkt} file.

The \key{read-program} function takes a file path and parses that file
(it must be a Racket program) into an abstract syntax tree (as an
S-expression) with a \key{program} AST at the top.

The \key{assert} function displays the error message \key{msg} if the
Boolean \key{bool} is false.
\begin{lstlisting}
(define (assert msg bool) ...)
\end{lstlisting}

The \key{lookup} function takes a key and an association list (a list
of key-value pairs), and returns the first value that is associated
with the given key, if there is one. If not, an error is triggered.
The association list may contain both immutable pairs (built with
\key{cons}) and mutable mapirs (built with \key{mcons}).

The \key{map2} function ...


%% \subsection{Graphs}

%% \begin{itemize}
%% \item The \code{make-graph} function takes a list of vertices
%%   (symbols) and returns a graph.

%% \item The \code{add-edge} function takes a graph and two vertices and
%%   adds an edge to the graph that connects the two vertices. The graph
%%   is updated in-place. There is no return value for this function.

%% \item The \code{adjacent} function takes a graph and a vertex and
%%   returns the set of vertices that are adjacent to the given
%%   vertex. The return value is a Racket \code{hash-set} so it can be
%%   used with functions from the \code{racket/set} module.

%% \item The \code{vertices} function takes a graph and returns the list
%%   of vertices in the graph.
%% \end{itemize}

\subsection{Testing}

The \key{interp-tests} function takes a compiler name (a string), a
description of the passes, an interpreter for the source language, a
test family name (a string), and a list of test numbers, and runs the
compiler passes and the interpreters to check whether the passes
correct. The description of the passes is a list with one entry per
pass.  An entry is a list with three things: a string giving the name
of the pass, the function that implements the pass (a translator from
AST to AST), and a function that implements the interpreter (a
function from AST to result value) for the language of the output of
the pass.  The interpreters from Appendix~\ref{appendix:interp} make a
good choice.  The \key{interp-tests} function assumes that the
subdirectory \key{tests} has a bunch of Scheme programs whose names
all start with the family name, followed by an underscore and then the
test number, ending in \key{.scm}. Also, for each Scheme program there
is a file with the same number except that it ends with \key{.in} that
provides the input for the Scheme program.
\begin{lstlisting}
(define (interp-tests name passes test-family test-nums) ...
\end{lstlisting}

The compiler-tests function takes a compiler name (a string) a
description of the passes (see the comment for \key{interp-tests}) a
test family name (a string), and a list of test numbers (see the
comment for interp-tests), and runs the compiler to generate x86 (a
\key{.s} file) and then runs gcc to generate machine code.  It runs
the machine code and checks that the output is 42.
\begin{lstlisting}
(define (compiler-tests name passes test-family test-nums) ...)
\end{lstlisting}

The compile-file function takes a description of the compiler passes
(see the comment for \key{interp-tests}) and returns a function that,
given a program file name (a string ending in \key{.scm}), applies all
of the passes and writes the output to a file whose name is the same
as the program file name but with \key{.scm} replaced with \key{.s}.
\begin{lstlisting}
(define (compile-file passes)
  (lambda (prog-file-name) ...))
\end{lstlisting}

\section{x86 Instruction Set Quick-Reference}
\label{sec:x86-quick-reference}


Table~\ref{tab:x86-instr} lists some x86 instructions and what they
do. We write $A \to B$ to mean that the value of $A$ is written into
location $B$.  Address offsets are given in bytes. The instruction
arguments $A, B, C$ can be immediate constants (such as $\$4$),
registers (such as $\%rax$), or memory references (such as
$-4(\%ebp)$). Most x86 instructions only allow at most one memory
reference per instruction.  Other operands must be immediates or
registers.


\begin{table}[tbp]
  \centering
\begin{tabular}{l|l}
\textbf{Instruction} & \textbf{Operation} \\ \hline
\texttt{addq} $A$, $B$ &  $A + B \to B$\\
\texttt{negq} $A$ & $- A \to A$ \\
\texttt{subq} $A$, $B$ &  $B - A \to B$\\
\texttt{callq} $L$ & Pushes the return address and jumps to label $L$ \\
\texttt{callq} *$A$ & Calls the function at the address $A$. \\
%\texttt{leave} & $\texttt{ebp} \to \texttt{esp};$ \texttt{popl \%ebp} \\
\texttt{retq} & Pops the return address and jumps to it \\
\texttt{popq} $A$ & $*\mathtt{rsp} \to A; \mathtt{rsp} + 8 \to \mathtt{rsp}$ \\
\texttt{pushq} $A$ & $\texttt{rsp} - 8 \to \texttt{rsp}; A \to *\texttt{rsp}$\\
\texttt{leaq} $A$,$B$ & $A \to B$ ($C$ must be a register) \\
\texttt{cmpq} $A$, $B$ & compare $A$ and $B$ and set the flag register \\
\texttt{je} $L$ & \multirow{5}{3.7in}{Jump to label $L$ if the flag register
  matches the condition code of the instruction, otherwise go to the
  next instructions. The condition codes are \key{e} for ``equal'',
  \key{l} for ``less'', \key{le} for ``less or equal'', \key{g}
  for ``greater'', and \key{ge} for ``greater or equal''.} \\
\texttt{jl} $L$ & \\
\texttt{jle} $L$ & \\
\texttt{jg} $L$ & \\
\texttt{jge} $L$ & \\
\texttt{jmp} $L$ & Jump to label $L$ \\
\texttt{movq} $A$, $B$ &  $A \to B$ \\
\texttt{movzbq} $A$, $B$ &
  \multirow{3}{3.7in}{$A \to B$, \text{where } $A$ is a single-byte register
  (e.g., \texttt{al} or \texttt{cl}), $B$ is a 8-byte register,
  and the extra bytes of $B$ are set to zero.} \\
 & \\
 & \\
\texttt{notq} $A$ & $\sim A \to A$ \qquad (bitwise complement)\\
\texttt{orq} $A$, $B$ & $A | B \to B$ \qquad (bitwise-or)\\
\texttt{andq} $A$, $B$ & $A \& B \to B$ \qquad (bitwise-and)\\
\texttt{salq} $A$, $B$ & $B$ \texttt{<<} $A \to B$ (arithmetic shift left, where $A$ is a constant)\\
\texttt{sarq} $A$, $B$ & $B$ \texttt{>>} $A \to B$ (arithmetic shift right, where $A$ is a constant)\\
\texttt{sete} $A$ & \multirow{5}{3.7in}{If the flag matches the condition code,
   then $1 \to A$, else $0 \to A$. Refer to \texttt{je} above for the
   description of the condition codes. $A$ must be a single byte register
   (e.g., \texttt{al} or \texttt{cl}).} \\
\texttt{setl} $A$ & \\
\texttt{setle} $A$ & \\
\texttt{setg} $A$ & \\
\texttt{setge} $A$ &
\end{tabular}
\vspace{5pt}
  \caption{Quick-reference for the x86 instructions used in this book.}
  \label{tab:x86-instr}
\end{table}



\bibliographystyle{plainnat}
\bibliography{all}

\end{document}

%%  LocalWords:  Dybvig Waddell Abdulaziz Ghuloum Dipanwita Sussman
%%  LocalWords:  Sarkar lcl Matz aa representable Chez Ph Dan's nano
%%  LocalWords:  fk bh Siek plt uq Felleisen Bor Yuh ASTs AST Naur eq
%%  LocalWords:  BNF fixnum datatype arith prog backquote quasiquote
%%  LocalWords:  ast sexp Reynold's reynolds interp cond fx evaluator
%%  LocalWords:  quasiquotes pe nullary unary rcl env lookup gcc rax
%%  LocalWords:  addq movq callq rsp rbp rbx rcx rdx rsi rdi subq nx
%%  LocalWords:  negq pushq popq retq globl Kernighan uniquify lll ve
%%  LocalWords:  allocator gensym alist subdirectory scm rkt tmp lhs
%%  LocalWords:  runtime Liveness liveness undirected Balakrishnan je
%%  LocalWords:  Rosen DSATUR SDO Gebremedhin Omari morekeywords cnd
%%  LocalWords:  fullflexible vertices Booleans Listof Pairof thn els
%%  LocalWords:  boolean typecheck notq cmpq sete movzbq jmp al xorq
%%  LocalWords:  EFLAGS thns elss elselabel endlabel Tuples tuples os
%%  LocalWords:  tuple args lexically leaq Polymorphism msg bool nums
%%  LocalWords:  macosx unix Cormen vec callee xs maxStack numParams
%%  LocalWords:  arg bitwise XOR'd thenlabel immediates optimizations
%%  LocalWords:  deallocating Ungar Detlefs Tene kx FromSpace ToSpace
%%  LocalWords:  Appel Diwan Siebert ptr  fromspace rootstack typedef
%%  LocalWords:  len prev rootlen heaplen setl lt
